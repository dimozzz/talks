\begin{frame}{Системы доказательств}

	Язык~--- подмножество $\{0, 1\}^*$.

    \begin{definition}[Cook, Reckhow 79]
        Система доказательств для языка $L$~--- такой полиномиальный по времени алгоритм $\Pi: \{0,
        1\}^* \times \{0, 1\}^* \rightarrow \{0, 1\}$, что:
        \begin{itemize}
            \item (полнота) $x \in L \Rightarrow \exists w ~ \Pi(x, w) = 1$;
            \item (корректность) $\exists w ~ \Pi(x, w) = 1 \Rightarrow x \in L$.
        \end{itemize}
    \end{definition}

    \pause

    Мера сложности~--- длина $|w|$.

    \pause

    Основной язык~--- язык невыполнимых пропозициональных формул в КНФ ($\UNSAT$).

\end{frame}


\begin{frame}{Примеры}

    \begin{itemize}
        \item Резолюция.
            \pause
            Оперирует с дизъюнктами по следующим правилам:
            \begin{itemize}
                \item $\frac{A \lor x ~~~ B \lor \neg x}{A \lor B} ~~~~~ \frac{A}{A \lor z}$
            \end{itemize}
            Доказательство: вывод пустого дизъюнкта из дизъюнктов исходной формулы.
        \pause
        \item Cutting Planes.
            \pause
            Оперирует с неравенствами над $\mathbb{Z}$ по следующим правилам:
            \begin{itemize}
                \item $\frac{A \ge a ~~~ B \ge b}{\alpha A + \beta B \ge \alpha a + \beta b} ~~~~~ \frac{ka_1 x_1 + ka_2 x_2
                    + \dots \ge c}{a_1 x_1 + a_2 x_2 + \dots \ge \lceil \frac{c}{k} \rceil}$
            \end{itemize}
            Доказательство: вывод неравенства $0 \ge 1$ из неравенств, кодирующих дизъюнкты формулы.
        \pause
        \item Nullstellensatz.
            \pause
            $\mathbb{K}$~--- произвольное алгебраически замкнутое поле. Истинность дизъюнкта $C_i$ кодируем в виде
            полиномиального уравнения $p_i(x_1, x_2, \dots) = 0$ над полем $\mathbb{K}$.
            \pause
            
            Доказательство: такой набор полиномов $h_i$, что $\sum\limits_{i} h_i p_i = 1$.
    \end{itemize}

\end{frame}


\begin{frame}{Мотивация}

    \begin{itemize}
        \item Программа Кука по разделению $\NP$ и $\coNP$ ($\P$ и $\NP$): доказывать оценки для {\color{blue} все более
            сильных} систем, пока не удастся обобщить методы на произвольную систему доказательств.
        \pause
        \item Нижние оценки на {\color{blue} слабые} (резолюции и аналоги) системы доказательств дают оценки на алгоритмы для
            решения задачи выполнимости (покрывают большинство применяемых эвристик).
        \pause
        \item Нижние оценки на {\color{blue} сильные} системы доказательств дали бы нижние оценки на другие модели вычислений
            (например алгебраические схемы).
    \end{itemize}
\end{frame}


\begin{frame}{История}

    \begin{itemize}
        \item Оценки на ослабленную версию резолюций (Цейтин 68).
        \pause
        \item Оценки на резолюции (Haken 85; Kraj{\'{\i}}{\v{c}}ek 97; Ben-Sasson, Wigderson 02; ...; Разборов 16).
        \pause
        \item Оценки на Cutting Planes (Pudl{\'{a}}k 95; Haken, Cook 98).
        \pause
        \item Оценки на Nullstellensatz и усиления (Разборов 98; Алехнович, Разборов 2003; Nordstrom, Mik{\v{s}}a 15).
        \pause
        \item Оценки на ограниченные версии систем через коммуникационную сложность (Beame, Impagliazzo, Pitassi 94; Beame и
            др. 08; Nordstrom, Huynh 12; Goos, Pitassi 14)
    \end{itemize}

\end{frame}

\begin{frame}{Результаты}

    \begin{itemize}
        \item Нижние оценки на время работы алгоритмов расщепления (Ицыксон, С 11, 12, 14).
        \pause
        \item Система $Res(\oplus)$ (обобщение резолюций): верхние оценки, нижние оценки на ограниченную версию (Ицыксон,
            С 14).
        \pause
        \item Нижние оценки на размер доказательства в резолюциях для формул, кодирующий существование совершенного
            паросочетания в графах (Ицыксон, Слабодкин, С 16).
        \pause
        \item Нижние оценки на $\OBDD$-алгоритмы и $\OBDD$-системы доказательств (Ицыксон, Кноп, Ромащенко, С 17).
        \pause
        \item Нижняя оценка на {\color{blue} random} Cutting Planes (усиление $\CP$) при помощи коммуникационной сложности (C
            17).
    \end{itemize}

\end{frame}


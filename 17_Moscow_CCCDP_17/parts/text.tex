\begin{frame}{Proof systems}

    \begin{definition}[Cook, Reckhow 79]
        Let $\varphi$ be a boolean formula in CNF. A proof system for $\UNSAT$ is a polynomial-time algorithm $\Pi: \{0,
        1\}^* \times \{0, 1\}^* \rightarrow \{0, 1\}$, such that: 
        \begin{itemize}
            \item (completeness) $\varphi \in \UNSAT \Rightarrow \exists w ~ \Pi(\varphi, w) = 1$;
            \item (soundness) $\exists w ~ \Pi(x, w) = 1 \Rightarrow x \in \UNSAT$.
        \end{itemize}
    \end{definition}

    The complexity measure is the lenght of minimal $w$.
    \pause

    \vspace{0.3cm}

    \begin{itemize}
        \item Resolution. The proof is a sequence of clauses $D_1, \dots, D_s$ such that:
            \begin{itemize}
                \item $D_s = \emptyset$;
                \item either $D_i$ is a clause of $\varphi$
                \item or $D_i$ can be obtain by the following rule from previous clauses $\frac{A \lor x ~~~ B \lor \neg x}{A
                    \lor B} ~~~~~ \frac{A}{A \lor z}$.
            \end{itemize}
    \end{itemize}
\end{frame}

\begin{frame}{Proof systems 2}
    \begin{itemize}
        \item Resolution. The proof is a sequence of clauses $D_1, \dots, D_s$ such that:
            \begin{itemize}
                \item $D_s = \emptyset$;
                \item either $D_i$ is a clause of $\varphi$
                \item or $D_i$ can be obtain by the following rule from previous clauses $\frac{A \lor x ~~~ B \lor \neg x}{A
                    \lor B} ~~~~~ \frac{A}{A \lor z}$.
            \end{itemize}
        \pause
        \item Cutting Planes. The proof is a sequence of linear inequalities $D_1, \dots, D_s$ over integers such that:
            \begin{itemize}
                \item $D_s = 0 \ge 1$;
                \item either $D_i$ encodes a clause of $\varphi$
                \item or $D_i$ can be obtain by the following rule from previous clauses $\frac{A \ge a ~~~ B \ge b}{\alpha A
                    + \beta B \ge \alpha a + \beta b} ~~~~~ \frac{ka_1 x_1 + ka_2 x_2 + \dots \ge c}{a_1 x_1 + a_2 x_2 +
                    \dots \ge \lceil \frac{c}{k} \rceil}$.
            \end{itemize}
        \pause
        \item Generalization (semantic proof system). The proof is a sequence of constraints $D_1, \dots, D_s$ such that:
            \begin{itemize}
                \item $D_s$ is trivially false constraint;
                \item either $D_i$ encodes a clause of $\varphi$
                \item or $D_i$ is semantically followed from two previous constraints.
            \end{itemize}
    \end{itemize}
\end{frame}


\begin{frame}{Motivation}

    Копия из СПБГУ.
\end{frame}


\begin{frame}
    Теорема Крейга.

    Извлечение схемы.
\end{frame}

\begin{frame}

	Результаты Крайчека и Пудлака.
\end{frame}

\begin{frame}

	Когда интерполяции нет. + OBDD.
\end{frame}


\begin{frame}{Canonical search problem $\Search_{\varphi}$ (Impagliazzo et al. 1994)}

	
    $\varphi(x, y)$ is an unsatisfiable CNF formula:
    \begin{itemize}
        \item Alice receives a substitution to the variables $x$, Bob receives a substitution to the
            variables $y$;
        \item goal is to find a clause $C \in \varphi$ that is unsatisfied by this substitution.
    \end{itemize}

    \pause

    \begin{theorem}[Beame, Pitassi, Segerlind 2007. Informal]
        If there is a {\color{blue} tree-like} proof in proof system $\Pi$ ($\Pi$ from some \textit{huge
          class}) of size $S$ then there is a communication protocol for $\Search_{\varphi}$ of depth
        $poly(\log(S))$.
    \end{theorem}
    
\end{frame}


\begin{frame}{Proofs and games}
	$D_1, D_2, \dots, D_{17}$ is a semantic proof of of $\varphi(x, y)$.

    \begin{center}
    	\tikzstyle{inner} = [thin, circle, minimum size = 0.6cm, draw, inner sep = 0.1pt, black, font = \scriptsize]
\tikzstyle{inner_g} = [thin, circle, minimum size = 0.6cm, draw, inner sep = 0.1pt, black, fill = green]
\tikzstyle{inner_r} = [thin, circle, minimum size = 0.6cm, draw, inner sep = 0.1pt, black, fill = red]
\tikzstyle{inner_b} = [
	thin, circle, minimum size = 0.6cm, draw, inner sep = 0.1pt, black, fill = blue!30!white, font =
    \scriptsize]
\tikzstyle{ed} = [thick, ->, draw, black]

    
\begin{tikzpicture}

    \only<-1>{
        \node[inner] (a) at (0, 0) {$D_{17}$};
	}
    \only<2->{
        \node[inner_b] (a) at (0, 0) {$D_{17}$};
    }

    \only<-2>{
        \node[inner] (b) at (-1.5, -1) {$D_{16}$};
	}
    \only<3->{
        \node[inner_b] (b) at (-1.5, -1) {$D_{16}$};
    }
  

    \node[inner] (c) at (1.5, -1) {$D_{15}$};

    \node[inner] (d) at (-3, -2) {$D_{14}$};

    \only<-3>{
        \node[inner] (e) at (0, -2) {$D_{13}$};
	}
    \only<4->{
        \node[inner_b] (e) at (0, -2) {$D_{13}$};
    }

    \node[inner] (f) at (3, -2) {$D_{12}$};

    \node[inner] (g) at (-4.5, -3) {$D_{11}$};

    \only<-4>{
        \node[inner] (h) at (-1.5, -3) {$D_{10}$};
	}
    \only<5->{
        \node[inner_b] (h) at (-1.5, -3) {$D_{10}$};
    }

    \node[inner] (i) at (1.0, -3) {$D_9$};

    \only<-6>{
        \node[inner] (j) at (2.0, -3) {$D_8$};
	}
    \only<7->{
        \node[inner_b] (j) at (2.0, -3) {$D_8$};
    }

    \node[inner] (k) at (4.0, -3) {$D_7$};

    \only<-5>{
        \node[inner] (l) at (-2.5, -4) {$D_6$};
	}
    \only<6->{
        \node[inner_b] (l) at (-2.5, -4) {$D_6$};
    }

    \node[inner] (m) at (-0.5, -4) {$D_5$};

    \only<-7>{
        \node[inner] (n) at (1.2, -4) {$D_4$};
	}
    \only<8->{
        \node[inner_b] (n) at (1.2, -4) {$D_4$};
    }

    \node[inner] (o) at (2.8, -4) {$D_3$};

    \node[inner] (p) at (0.4, -5) {$D_2$};
    
    \only<-8>{
        \node[inner] (q) at (2, -5) {$D_1$};
	}
    \only<9->{
        \node[inner_b] (q) at (2, -5) {$D_1$};
    }


    
    \path (a) edge[ed] (b);
    \path (a) edge[ed] (c);
    \path (b) edge[ed] (d);
    \path (b) edge[ed] (e);
    \path (c) edge[ed] (e);
    \path (c) edge[ed] (f);
    \path (d) edge[ed] (g);
    \path (d) edge[ed] (h);
    \path (e) edge[ed] (h);
    \path (e) edge[ed] (i);
    \path (f) edge[ed] (j);
    \path (f) edge[ed] (k);
    \path (h) edge[ed] (l);
    \path (h) edge[ed] (m);
    \path (j) edge[ed] (n);
    \path (j) edge[ed] (o);
    \path (n) edge[ed] (p);
    \path (n) edge[ed] (q);
\end{tikzpicture}    
    \end{center}

\end{frame}

\begin{frame}
    Игры и схемы. Monbit.
\end{frame}
    


\begin{frame}

    Сложность констрейнта.

    Игры и схемы. Результаты

    Прямые оценки на игры.
\end{frame}


\begin{frame}

	Случайные формулы.  Pit and Pud.
\end{frame}
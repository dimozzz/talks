\begin{frame}{Proof systems}

    \begin{definition}[Cook, Reckhow 79]
        Let $\varphi$ be a boolean formula in CNF. A proof system for $\UNSAT$ is a polynomial-time algorithm $\Pi: \{0,
        1\}^* \times \{0, 1\}^* \rightarrow \{0, 1\}$, such that: 
        \begin{itemize}
            \item (completeness) $\varphi \in \UNSAT \Rightarrow \exists w ~ \Pi(\varphi, w) = 1$;
            \item (soundness) $\exists w ~ \Pi(x, w) = 1 \Rightarrow x \in \UNSAT$.
        \end{itemize}
    \end{definition}

    The complexity measure is the lenght of minimal $w$.
    \pause

    \vspace{0.3cm}

    \begin{itemize}
        \item Resolution. The proof is a sequence of clauses $D_1, \dots, D_s$ such that:
            \begin{itemize}
                \item $D_s = \emptyset$;
                \item either $D_i$ is a clause of $\varphi$
                \item or $D_i$ can be obtain by the following rule from previous clauses $\frac{A \lor x ~~~ B \lor \neg x}{A
                    \lor B} ~~~~~ \frac{A}{A \lor z}$.
            \end{itemize}
    \end{itemize}
\end{frame}

\begin{frame}{Proof systems 2}
    \begin{itemize}
        \item Resolution. The proof is a sequence of clauses $D_1, \dots, D_s$ such that:
            \begin{itemize}
                \item $D_s = \emptyset$;
                \item either $D_i$ is a clause of $\varphi$
                \item or $D_i$ can be obtain by the following rule from previous clauses $\frac{A \lor x ~~~ B \lor \neg x}{A
                    \lor B} ~~~~~ \frac{A}{A \lor z}$.
            \end{itemize}
        \pause
        \item Cutting Planes. The proof is a sequence of linear inequalities $D_1, \dots, D_s$ over integers such that:
            \begin{itemize}
                \item $D_s = 0 \ge 1$;
                \item either $D_i$ encodes a clause of $\varphi$
                \item or $D_i$ can be obtain by the following rule from previous clauses $\frac{A \ge a ~~~ B \ge b}{\alpha A
                    + \beta B \ge \alpha a + \beta b} ~~~~~ \frac{ka_1 x_1 + ka_2 x_2 + \dots \ge c}{a_1 x_1 + a_2 x_2 +
                    \dots \ge \lceil \frac{c}{k} \rceil}$.
            \end{itemize}
        \pause
        \item Generalization (semantic proof system). The proof is a sequence of constraints $D_1, \dots, D_s$ such that:
            \begin{itemize}
                \item $D_s$ is trivially false constraint;
                \item either $D_i$ encodes a clause of $\varphi$
                \item or $D_i$ is semantically followed from two previous constraints.
            \end{itemize}
    \end{itemize}
\end{frame}


\begin{frame}{Motivation}
    \begin{itemize}
        \item Cook's program for separation $\NP$ and $\coNP$: to prove lower bounds for {\color{blue}
            stronger and stronger} systems until one can generalize techiques for all systems.
        \pause
        \item Lower bounds on {\color{blue} weak} (resolution and similar) systems give us the lower
            bounds on algorithms for satifiability problems.
        \pause
        \item Lower bounds on {\color{blue} strong} proof system would give us lower bounds on other
            computational models (for example algebraic circuits).
    \end{itemize}
\end{frame}
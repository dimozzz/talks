\begin{frame}{Формулы и функции}

    <<Слабые>> системы доказательств простая модель, а схемы <<сложная>>. 

    \pause

    \begin{itemize}
        \item $\varphi(x) \coloneqq \bigvee C_j$ ~--- формула от $n$ переменных;
        \item $g\colon \{0, 1\}^k \times \{0, 1\}^{\ell} \to \{0, 1\}$;
        \pause
        \item $\varphi \circ g \coloneqq \varphi(g(y_1, z_1), g(y_2, z_2), g(y_3, z_3), \dots)$.
    \end{itemize}

    \pause
    $\psi(y, z) \coloneqq \varphi \circ g = \bigvee\limits_{j = 1}^{m} D_j$, определим функцию
    $F_{\psi}\colon \{0, 1\}^m \to \{0, 1\}$.
    \pause
    \vspace{-0.2cm}
    \begin{center}
        \tikzstyle{ops} = [alt=<{#1}>{opacity = 1}{opacity = 0}]

\begin{tikzpicture}
    \draw[thick, rounded corners = 2pt] (0, 0) rectangle (4, 3);
    \node at (-1, 1.5) {$Y \coloneqq \{0, 1\}^{n k}$};
    \node at (5, 1.5) {};
    \node at (2, 3.3) {$Z \coloneqq \{0, 1\}^{n \ell}$};

    \draw[red!30, fill = red!10, rounded corners = 3pt] (0.3, 0.1) rectangle (1, 2.9)
        node[midway, red!80] {$D_1$};
    \draw[green!50!black, fill = green!30, rounded corners = 3pt, opacity = 0.5] (0.5, 0.4) rectangle
        (3.5, 0.9) node[midway, green!20!black] {$D_2$};
    \draw[blue!50!black, fill = blue!30, rounded corners = 3pt, opacity = 0.5] (0.7, 2) rectangle
        (3.4, 2.85) node[midway, blue!20!black] {$D_3$};
    \draw[ops = 6, very thick, red] (-1, 0.5) -- (5, 0.5);
    \draw[ops = 7, very thick, red] (3, -0.5) -- (3, 3.5);
\end{tikzpicture}
    \end{center}

    \pause
    $F_{\psi}(1, 1, 0, \dots) \coloneqq 1$\pause, ~~~$F_{\psi}(1, 0, 0, \dots) \coloneqq 0$
\end{frame}

\begin{frame}{Формулы и функции}

    \pause
    \begin{itemize}
        \item $F_{\psi}$ определена частично;
            \pause
        \item $F_{\psi}$ можно дополнить до монотонной.
    \end{itemize}

    \begin{theorem}
        Если $\varphi$ сложна для резолюционной системы $\Rightarrow$ $F_{\varphi \circ g}$ требует
        большой \alert{монотонной} схемы, где $g$~--- <<хорошая>> функция.
    \end{theorem}

    \pause

    <<Обратная>> теорема тоже существует.
\end{frame}

\begin{frame}{Иерархия}

    
\tikzset{
    vert/.style = {
        draw,
        ellipse
    },
    tikzart-fire/.pic = {
        \draw[fill = red!60] (0, 0) .. controls (0.3, 0) and (0.6, 0.1) .. (0.7, 0.3)
            .. controls (0.8, 0.5) and (0.85, 0.6) .. (0.8, 0.9)
            .. controls (0.75, 1.1) and (0.7, 1.2) .. (0.6, 1.4)
            .. controls (0.65, 1.2) and (0.6, 1.05) .. (0.5, 0.9)
            .. controls (0.5, 1.2) and (0.2, 1.3) .. (0.1, 1.6)
            .. controls (0.05, 1.75) and (0.1, 2) .. (0.2, 2.1)
            .. controls (-0.1, 2) and (-0.2, 1.85) .. (-0.3, 1.7)
            .. controls (-0.4, 1.5) and (-0.45, 1.3) .. (-0.4, 1.1)
            .. controls (-0.5, 1.2) and (-0.51, 1.4) .. (-0.5, 1.5)
            .. controls (-0.75, 1.2) and (-0.8, 0.7) .. (-0.7, 0.5)
            .. controls (-0.6, 0.28) and (-0.4, 0) .. (0, 0);
            \fill[white] (0, 0) .. controls (0.3, 0) and (0.52, 0.34) .. (0.37, 0.61)
            .. controls (0.4, 0.54) and (0.32, 0.32) .. (0.25, 0.25)
            .. controls (0.3, 0.35) and (0.25, 0.5) .. (0.2, 0.6)
            .. controls (0.1, 0.8) and (-0.05, 1) .. (0, 1.2)
            .. controls (-0.32, 1) and (-0.3, 0.72) .. (-0.2, 0.47)
            .. controls (-0.3, 0.51) and (-0.31, 0.6) .. (-0.33, 0.7)
            .. controls (-0.4, 0.6) and (-0.4, 0.5) .. (-0.4, 0.4)
            .. controls (-0.35, 0.18) and (-0.2, 0) .. (0, 0);
    },
    semisim/.style = {
        ->,
        blue,
        dashed,
        decorate,
        decoration = {
            snake,
            amplitude = 0.5,
            segment length = 2
        }
    },        
}


    
\begin{tikzpicture}[xscale = 1.3, xshift = -1]
    \node[vert] (res) at (1, 0) {$\Res$};
    \node[vert] (ns) at (-3, 0) {$\NS$};
    \node[vert] (cp) at (3, 1) {$\CP$};
    \node[vert] (resk) at (1.2, 1.4) {$\Res(k)$};
    \node[vert] (acf) at (1.3, 2.4) {$\AC_0$-Frege};
    \node[vert] (resl) at (-1.3, 2.7) {$\ResL$};
    \node[vert] (acfp) at (0.5, 3.8) {$\AC_0[p]$-Frege};
    \node[vert] (fre) at (0.5, 5) {Frege};
    \node[vert] (ips) at (-2, 6) {$\PrSys{IPS}$};
    \node[vert] (pcr) at (-2.8, 1.9) {$\PCR[]$};
    \node[vert] (sos) at (-4, 2.5) {$\SOS$};
    
    \node[vert] (cps) at (-4, 6.5) {$\PrSys{CPS}$};

    

    \draw[->, semisim] (pcr) -- (sos);
    \draw[->] (res) -- (cp);
    \draw[->] (cp) to[out = 90, in = -20] (fre);
    \draw[->] (res) -- (resl);
    \draw[->] (res) -- (resk);
    \draw[->] (resk) -- (acf);
    \draw[->] (res) to[out = 138, in = -30] (pcr);
    \draw[->] (ns) -- (pcr);
    \draw[->] (resl) -- (acfp);
    \draw[->] (acf) -- (acfp);
    \draw[->] (acfp) -- (fre);
    \draw[->] (fre) -- (ips);
    \draw[->, semisim] (ips) -- (cps);

    \draw[->] (pcr) -- (ips);
    \draw[->] (sos) -- (cps);

    \node at (0, 6.9) {};
    

    \begin{scope}[on background layer]
        \draw[ultra thick, fill = black!5] (-4, -1) to[out = 110, in = 220] (-4.4, 3)
            to[out = 40, in = 180] (-1, 2) to[out = 0, in = 125] (2.3, 3) to[out = -55, in = 90]
            (1.5, -1);
    \end{scope}
    \node[blue] at (-1.5, 0.95) {Restriction};


    \begin{scope}[on background layer]
        \fill[orange!5] (-4, -1) to[out = 90, in = 180] (-2.3, 0.75) to[out = 0, in = 180] (0.5, 0.7)
            to[out = 0, in = 200] (3.5, 2) -- (3.5, -1) -- cycle;
        \draw[ultra thick, orange] (-4, -1) to[out = 90, in = 180] (-2.3, 0.75) to[out = 0, in = 180]
            (0.5, 0.7) to[out = 0, in = 200] (3.5, 2);
        \draw[ultra thick] (-4, -1) to[out = 110, in = 220] (-4.4, 3) to[out = 40, in = 180] (-1, 2)
            to[out = 0, in = 125] (2.3, 3) to[out = -55, in = 90] (1.5, -1);
    \end{scope}
    \node[blue] at (0, -0.7) {Mon. Interpolation};
\end{tikzpicture}
    
\end{frame}

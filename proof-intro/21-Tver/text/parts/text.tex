Один из наиболее естественных вопросов математической логики: по заданному истинному утверждению оценить
длину кратчайшего доказательства в какой-нибудь аксиоматической системе. Мы сосредоточимся на этом
вопросе для пропозициональной логики, основном объекте теории сложности доказательств. Для удобства
вместо истинных утверждений (тавтологий) вы перейдем ко всюду ложным утверждениям и будет рассматривать
<<доказательства>> невыполнимости пропозициональных формул. Следуя терминологии теории сложности, мы
будем изучать язык $\UNSAT$ невыполнимых пропозициональных формул в КНФ.

Начнем с основного определения, которое было сформулировано в работе \cite{CookRec79}. 
\begin{definition*}
    Системой доказательств для языка $\UNSAT$ будем называть такую полиномиально вычислимую функцию
    $\Pi\colon \{0, 1\}^* \times \{0, 1\}^* \to \{0, 1\}$, что:
    \begin{itemize}
        \item если $\varphi \in \UNSAT$, то существует такое $w \in \{0, 1\}^*$, что $\Pi(\varphi, w) = 1$
            (будем говорить, что $w$~--- это доказательство для $\varphi$);
        \item если $\varphi \notin \UNSAT$, то для всех $w \in \{0, 1\}^*, \Pi(\varphi, w) = 0$.
    \end{itemize}
\end{definition*}

Про системы доказательств можно думать в терминах игр. Рассмотрим двух игроков: Оптимист и Пессимист,
которые получают формулу $\varphi$ в КНФ от $n$ переменных $x_1, \dots, x_n$. Оптимист считает, что есть
некоторый выполняющие набор для формулы $\varphi$, т.е. такой набор значений $a_1, a_2, \dots, a_n \in
\{0, 1\}$, что $\varphi(a_1, a_2, \dots, a_n) = 1$. А Пессимист считает что такого набора нет и пытается
убедить в этом Оптимиста, предъявив некоторое доказательство $w$. При этом система доказательств
определяется тем, какие доказательства Оптимист считает корректными.

Одним из классических примеров систем доказательств является резолюционная система. В данной системе
доказательство $\pi$ для формулы $\varphi$ представляет собой такую упорядоченную последовательность
дизъюнктов $\pi \coloneqq D_1, \dots, D_{s}$, что $D_{s} = \emptyset$ пустой дизъюнкт и для каждого $i
\in [s]$ либо $D_i$ это дизъюнкт $\varphi$, либо найдутся такие $j, k < i$, что $D_i$ получен из $D_j$ и
$D_k$ путем применения резолюционного правила
\begin{prooftree}
    \AxiomC{$C \lor x$}
    \AxiomC{$D \lor \neg x$}
    \BinaryInfC{$C \lor D$}
\end{prooftree}
или правила ослабления
\begin{prooftree}
    \AxiomC{$C$}
    \RightLabel{, где $[C \subseteq D]$.}
    \UnaryInfC{$D$}
\end{prooftree}
В терминах нашего определения $\Pi$~--- это алгоритм, который получает вместо формулу $\varphi$, и в
качестве $w$ он получает резолюционное доказательство, которое, ему необходимо проверить.

Как мы уже замечали в начале, основной задачей сложности доказательств является оценка размера кратчайших
доказательств невыполнимости формул в различных системах. И сложность доказательств~--- один из немногих
разделов теории сложности, где удается получить безусловные нижние оценки. В частности: на резолюционную
систему доказательств \cite{Haken85}, на ряд систем алгебраического вывода
\cite{BIPRS97, CEI96, IPS99}. Однако, для некоторых систем вопрос о нижних оценках по прежнему открыт,
например для системы Фреге (Гильбертовская система).

Нижние и верхние оценки на сложность доказательств в различных системах часто удается переносить на
другие модели вычислений.
\begin{enumerate}
    \item Если бы нам удалось привести пример формул для которых кратчайшее доказательство имеет
        суперполиномиальный размер (от длины формулы) во всех системах, то это бы повлекло за собой
        неравенство классов $\NP \neq \coNP$ \cite{CookRec79}, и, в частности, классов $\P \neq \NP$, что
        является одной из <<задач тысячелетия>>. 
    \item Даже если мы сосредоточимся на резолюционной системе доказательств, то нижние оценки размер
        доказательств в ней влекут нижние оценки на время работы популярных алгоритмов для решения задачи
        выполнимости булевых формул (что является $\NP$-полной задачей) \cite{DP60, AHI05}.
    \item Также, известных нижних оценок (на резолюционную систему, а также на ряд систем, основанных на
        алгебраическом выводе) хватает для получения сильных нижних оценок на монотонные модели
        вычисления, в частности на монотонные схемы и, так называемые, монотонные \textit{span programs}
        \cite{PR18, GGKS18}.
\end{enumerate}

В докладе мы сосредоточимся на применения теории сложности, а также обсудим основные задачи и современные
проблемы данной теории.

\begin{frame}{Системы доказательств}

	Язык: $L \subseteq \{0, 1\}^*$. $\UNSAT$: язык невыполнимых пропозициональных формул в КНФ.
    \pause

    \begin{definition}[Cook, Reckhow 79]
        Система доказательств для языка $L$~--- такой полиномиальный по времени алгоритм $\Pi\colon \{0,
        1\}^* \times \{0, 1\}^* \rightarrow \{0, 1\}$, что:
        \begin{itemize}
            \item (полнота) $x \in L \Rightarrow \exists w ~ \Pi(x, w) = 1$;
            \item (корректность) $\exists w ~ \Pi(x, w) = 1 \Rightarrow x \in L$.
        \end{itemize}
    \end{definition}

    Мера сложности~--- длина $|w|$.

    \pause

    \begin{block}{Программа Кука}
        Будем доказывать оценки для \textcolor{blue}{все более сильных} систем, пока не удастся обобщить
        методы на произвольную систему доказательств.

        Цель: показать, что язык $\UNSAT$ сложный ($\NP \neq \coNP, \P \neq \NP$).
    \end{block}

\end{frame}


\begin{frame}{Примеры}

    $$
        \varphi \coloneqq (x \lor y \lor \neg z) \land (\neg w \land u) \land (\neg x \land \neg u) \land \dots
    $$

    \pause
    \begin{itemize}
        \item Резолюция. $A, B$~--- дизъюнкты.
            \begin{itemize}
                \item $\frac{A \lor x ~~~ B \lor \neg x}{A \lor B} ~~~~~ \frac{A}{A \lor z}$
            \end{itemize}
            Доказательство: вывод пустого дизъюнкта из дизъюнктов исходной формулы.
        \pause
        \item Cutting Planes. $(x \lor y \lor \neg z) \Rightarrow x + y + (1 - z) \ge 1$.
            \begin{itemize}
                \item $\frac{A \ge a ~~~ B \ge b}{\alpha A + \beta B \ge \alpha a + \beta b} ~~~~~
                    \frac{ka_1 x_1 + ka_2 x_2 + \dots \ge c}{a_1 x_1 + a_2 x_2 + \dots \ge \lceil
                    \frac{c}{k} \rceil}$
            \end{itemize}
            Доказательство: вывод неравенства $0 \ge 1$ из неравенств, кодирующих дизъюнкты формулы.
        \pause
        \item Nullstellensatz.
            \begin{itemize}
                \item $\field$~--- поле;
                \item $(x \lor y \lor \neg z) \Rightarrow f_i \coloneqq (1 - x) (1 - y) z = 0$;
                \item $x^2 - x = 0$.
            \end{itemize}
            Доказательство: такой набор полиномов $h_i$, что $\sum\limits_{i} h_i f_i = 1$.
    \end{itemize}

\end{frame}


\begin{frame}{Pebbling}

    \begin{center}
        \tikzstyle{vert} = [
    circle,
    draw,
    inner sep = 0pt,
    minimum size = 0.45cm,
    fill = LEIorange!5
]
\tikzstyle{pstyle} = [alt=<{#1}>{fill = LEIorange!50}{}]
    
\tikzset{
    xcenter around/.style 2 args = {
        execute at end picture = {%
            \useasboundingbox let \p0 = (current bounding box.south west),
            \p1 = (current bounding box.north east),
            \p2 = (#1),
            \p3 = (#2) in ({min(\x2 + \x3 - \x1, \x0)}, \y0) rectangle ({max(\x3 + \x2 - \x0, \x1)},\y1);
        }
    }
}

\begin{tikzpicture}[>=latex, xcenter around = {-2.1, -2.1}{2.1, 0.1}]
    \node at (3.6, -0.6) {\includegraphics[scale = 0.09]{pics/utia-rest.png}};
    
    \node[vert, pstyle = 2] (a) at (0, 0) {$r$};
    \node[vert, pstyle = 4] (b) at (-1, -1) {$x$};
    \node[vert] (c) at (1, -1) {$y$};
    \node[vert, pstyle = {3-4}] (d) at (-2, -2) {$z$};
    \node[vert, pstyle = {3-4}] (e) at (0, -2) {$u$};
    \node[vert, pstyle = 3] (f) at (2, -2) {$w$};

    \draw[->] (b) -- (a);
    \draw[->] (c) -- (a);
    \draw[->] (d) -- (b);
    \draw[->] (e) -- (b);
    \draw[->] (e) -- (c);
    \draw[->] (f) -- (c);
\end{tikzpicture}        
    \end{center}

    \pause
    \begin{itemize}
        \item $(\neg r)$;
            \pause
        \item $(z), (u), (w)$;
            \pause
        \item $(\neg z \lor \neg u \lor x), (\neg w \lor \neg u \lor y), (\neg x \lor \neg y \lor r)$.
    \end{itemize}

    \pause
    \tikzset{
    pr-vert/.style = {
        draw,
        rounded rectangle,
        minimum width = 1cm,
        minimum height = 0.4cm,
        outer sep = 0pt,
        fill = #1
    },
    pr-vert/.default = LEIblue!10
}

\tikzstyle{ops} = [alt = <{#1-}>{opacity = 1}{opacity = 0}]

\begin{tikzpicture}
    \node[pr-vert, ops = 6] (a) at (0, 0) {$u$};
    \node[pr-vert, ops = 6] (b) at (0, -1) {$z$};
    \node[pr-vert, ops = 6] (c) at (0, -2) {$\neg z \lor \neg u \lor x$};
    
    \node[pr-vert = LEIorange!10, ops = 7] (d) at (2, -1.9) {$\neg u \lor x$};
    \node[pr-vert = LEIorange!10, ops = 7] (e) at (3, -1.25) {$x$};

    \node[pr-vert, ops = 8] (f) at (4, 0) {$\neg u \lor \neg w \lor y$};
    \node[pr-vert, ops = 8] (g) at (6, 0) {$w$};
    
    \node[pr-vert = LEIorange!10, ops = 8] (h) at (5, -0.75) {$\neg w \lor y$};
    \node[pr-vert = LEIorange!10, ops = 8] (i) at (6.5, -1) {$y$};

    \node[pr-vert, ops = 9] (j) at (5, -2) {$\neg x \lor \neg y \lor r$};

    \node[pr-vert = LEIorange!10, ops = 9] (k) at (7.5, -2) {$\neg y \lor r$};
    \node[pr-vert = LEIorange!10, ops = 9] (l) at (8.5, -1.2) {$r$};

    \node[pr-vert, ops = 10] (m) at (8.5, 0) {$\neg r$};

    \node[pr-vert = red!30, ops = 10] (n) at (9.5, -0.6) {$\emptyset$};

    \draw[->, ops = 7] (b) -- (d);
    \draw[->, ops = 7] (c) -- (d);
    \draw[->, ops = 7] (a) -- (e);
    \draw[->, ops = 7] (d) -- (e);

    \draw[->, ops = 8] (a) -- (h);
    \draw[->, ops = 8] (f) -- (h);
    \draw[->, ops = 8] (h) -- (i);
    \draw[->, ops = 8] (g) -- (i);

    \draw[->, ops = 9] (e) -- (k);
    \draw[->, ops = 9] (j) -- (k);
    \draw[->, ops = 9] (k) -- (l);
    \draw[->, ops = 9] (i) -- (l);

    \draw[->, ops = 10] (l) -- (n);
    \draw[->, ops = 10] (m) -- (n);
\end{tikzpicture}
\end{frame}

\begin{frame}{Принцип Дирихле (Pigeonhole Principle), $\PHP_{n}^{m}$}

    Переменные: $x_{i, j}$, $i \in \{1, 2, \dots, m\}$, $j \in \{1, 2, \dots, n\}$.
    \vspace{0.1cm}

    \pause
    \begin{itemize}
        \item $\bigvee\limits_{j = 1}^{n} x_{i, j}$;
        \item $\neg x_{i, j} \lor \neg x_{i', j}$.
    \end{itemize}

    \pause

    \vspace{0.2cm}
    \begin{minipage}{0.3\linewidth}
        \centering
        \tikzstyle{undir} = [thick]
\tikzstyle{dir} = [thick, ->, bend left = 12]
\tikzstyle{ver} = [thick, ->, draw, circle]

\begin{tikzpicture}[black, >=stealth']

    \node[ver] (A) at (0, 0) {};
    \node[ver] (B) at (2, 2) {};
    \node[ver] (C) at (3, 0) {};
    \node[ver] (D) at (4, 1) {};
    \node[ver] (E) at (3.5, 3.5) {};
    \node[ver] (F) at (-0.5, 3) {};

    \only<1>{
        \draw[undir] (A) to (B);
        \draw[undir] (A) to (C);
        \draw[undir] (B) to (C);
        \draw[undir] (C) to (D);
        \draw[undir] (B) to (D);
        \draw[undir] (D) to (E);
        \draw[undir] (E) to (F);
        \draw[undir] (F) to (A);
        \draw[undir] (B) to (F);
    }
    
	\only<2->{
        \draw[dir] (A) to (B);
        \draw[dir] (A) to (C);
        \draw[dir] (B) to (C);
        \draw[dir] (C) to (D);
        \draw[dir] (B) to (D);
        \draw[dir] (D) to (E);
        \draw[dir] (E) to (F);
        \draw[dir] (F) to (A);
        \draw[dir] (B) to (F);

        \draw[dir] (B) to (A);
        \draw[dir] (C) to (A);
        \draw[dir] (C) to (B);
        \draw[dir] (D) to (C);
        \draw[dir] (D) to (B);
        \draw[dir] (E) to (D);
        \draw[dir] (F) to (E);
        \draw[dir] (A) to (F);
        \draw[dir] (F) to (B);
    }
\end{tikzpicture}

    \end{minipage}
    \pause
    \begin{minipage}{0.68\linewidth}
        \begin{theorem}[Haken 85]
            Любое резолюционное доказательство $\PHP_{n}^{n + 1}$ имеет размер $2^{\Omega(n)}$. 
        \end{theorem}
        \pause
        \begin{theorem}[Raz 04; Razborov 03]
            Любое резолюционное доказательство $\PHP_{n}^{\infty}$ имеет размер $2^{\Omega(n^{1 / 3})}$.
        \end{theorem}
        \pause
        \begin{minipage}{0.3\linewidth}
            \centering
            \includegraphics[width = 0.8\textwidth]{pics/pigeon3.png}
        \end{minipage}
        \begin{minipage}{0.67\linewidth}
            Открытый вопрос: улучшить нижнюю оценку на размер доказательств $\PHP_{n}^{\infty}$ до
            $2^{\Omega(n^{1 / 2})}$.
        \end{minipage}
    \end{minipage}
    
\end{frame}
\begin{frame}{Немного о задачах}

    Как понять, что объект достоен изучения?

    \pause
    \begin{itemize}
        \item Объект интересен лично вам.
        \pause
        \item Объект появился хотя бы \alert{дважды} в различных задачах независимым образом.
    \end{itemize}
\end{frame}


\begin{frame}{Немного о <<теории сложности>>}

    $f\colon X^n \to Y$

    Есть ли простое описание у функции $f$? \pause $\Leftrightarrow$ Выразима ли функция $f$, как
    композиция <<простых>>?

    \pause

    \begin{itemize}
        \item{} [Теорема Абеля] Корни многочленов степени $5$ невыразимы в виде композиции <<простых>>
            функций.
            \pause
        \item{} [13-я проблема Гильберта] Можно ли выразить корни многочленов степени $7$ в виде
            композиции функций от двух переменных?
            \pause
        \item{} [Теорема Колмогорова -- Арнольда] Любую непрерывную функцию, можно выразить в виде
            композиции функций от двух переменных.      
    \end{itemize}

    \vspace{0.5cm}
    \pause
    $X \coloneqq \{0, 1\}$
    \begin{itemize}
        \item Интерполяционный полином.
        \pause
        \item{} <<Можно ли выразить?>> $\Rightarrow$ <<Насколько сложно выразить?>>
    \end{itemize}
\end{frame}


\begin{frame}{Схемы}

    \begin{center}
        \tikzstyle{inner} = [thin, circle, minimum size = 0.4cm, draw, inner sep = 0.1pt, black, opacity = 1]
\tikzstyle{inner_g} = [thin, circle, minimum size = 0.4cm, draw, inner sep = 0.1pt, black, fill =
    green, opacity = 1]
\tikzstyle{ed} = [thick, <-, draw, black, opacity = 1]

    
\begin{tikzpicture}[>=stealth', yscale = -0.98, xscale = 1.5]
    \node[inner] (a) at (0, 0) {\scriptsize $\land$};
    \node[inner] (b) at (-0.9, -1.2) {\scriptsize $\oplus$};
    \node[inner] (c) at (0.9, -1.2) {\scriptsize $\lor$};
    \node[inner_g] (d) at (-1.5, -2.4) {\scriptsize $x_1$};
    \node[inner] (e) at (-0.3, -2.4) {\scriptsize $\lor$};
    \node[inner] (f) at (1, -2.4) {\scriptsize $\oplus$};
    \node[inner_g] (g) at (-1.5, -4.3) {\scriptsize $x_2$};
    \node[inner_g] (h) at (-0.25, -4.3) {\scriptsize $x_3$};
    \node[inner_g] (g2) at (1.5, -4.3) {\scriptsize $x_3$};
    \node[inner_g] (h2) at (0.25, -4.3) {\scriptsize $x_2$};
    
    \path (a) edge[ed] (b);
    \path (a) edge[ed] (c);
    \path (b) edge[ed] (d);
    \path (b) edge[ed] (e);
    \path (c) edge[ed] (e);
    \path (c) edge[ed] (f);
    \path (e) edge[ed] (g);
    \path (e) edge[ed] (h);
    \path (f) edge[ed] (g2);
    \path (f) edge[ed] (h2);
\end{tikzpicture}

    \end{center}

    \pause

    \vspace{-0.7cm}
    \begin{theorem}
        Если алгоритм может посчитать функцию за время $t$, то есть и схема для функции размера
        $\bigO{t \log t}$. 
    \end{theorem}

    \pause
    \vspace{0.1cm}
    \begin{itemize}
        \item Криптография;
        \item классификация вычислительных задач;
        \item $\ldots$
    \end{itemize}
\end{frame}


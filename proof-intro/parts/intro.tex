\begin{frame}{Немного о <<теории сложности>>}

    Сколько <<вычислительных ресурсов>> требует задача?

    \pause
    \begin{itemize}
        \item Простые задачи:
            \begin{itemize}
                \item сложение;
                \item умножение;
                \item поиск кратчейшего маршрута в городе;
            \end{itemize}
        \pause
        \item Сложные задачи:
            \begin{itemize}
                \item составление расписания;
                \item системы уравнений в $\{0, 1\}$ переменных.
            \end{itemize}
        \pause
        \item Невычислимые задачи:
            \begin{itemize}
                \item задача об остановке алгоритма;
                \item системы уравнений в целых числах.
            \end{itemize}
    \end{itemize}
    
\end{frame}

\begin{frame}{Не только алгоритмы}

    \pause
    \begin{enumerate}
        \item Длины <<доказательств>>.
            \pause
        \item Размеры деревьев решений.
            \pause
        \item Размеры коммуникационных протоколов.
        \item И т.д.
    \end{enumerate}
\end{frame}
\begin{frame}{Системы доказательств}

	Язык: $L \subseteq \{0, 1\}^*$. $\UNSAT$: язык невыполнимых пропозициональных формул в КНФ.
    \pause

    \begin{definition}[Cook, Reckhow 79]
        Система доказательств для языка $L$~--- такой полиномиальный по времени алгоритм $\Pi\colon \{0,
        1\}^* \times \{0, 1\}^* \rightarrow \{0, 1\}$, что:
        \begin{itemize}
            \item (полнота) $x \in L \Rightarrow \exists w ~ \Pi(x, w) = 1$;
            \item (корректность) $\exists w ~ \Pi(x, w) = 1 \Rightarrow x \in L$.
        \end{itemize}
    \end{definition}

    Мера сложности~--- длина $|w|$.

    \pause

    \begin{block}{Программа Кука}
        Будем доказывать оценки для \textcolor{blue}{все более сильных} систем, пока не удастся обобщить
        методы на произвольную систему доказательств.

        Цель: показать, что язык $\UNSAT$ сложный.
    \end{block}

\end{frame}


\begin{frame}{Примеры}

    $$
        \varphi \coloneqq (x \lor y \lor \neg z) \land (\neg w \land u) \land (\neg x \land \neg u) \land \dots
    $$

    \pause
    \begin{itemize}
        \item Резолюция. $A, B$~--- дизъюнкты.
            \begin{itemize}
                \item $\frac{A \lor x ~~~ B \lor \neg x}{A \lor B} ~~~~~ \frac{A}{A \lor z}$
            \end{itemize}
            Доказательство: вывод пустого дизъюнкта из дизъюнктов исходной формулы.
        \pause
        \item Cutting Planes. $(x \lor y \lor \neg z) \Rightarrow x + y + (1 - z) \ge 1$.
            \begin{itemize}
                \item $\frac{A \ge a ~~~ B \ge b}{\alpha A + \beta B \ge \alpha a + \beta b} ~~~~~
                    \frac{ka_1 x_1 + ka_2 x_2 + \dots \ge c}{a_1 x_1 + a_2 x_2 + \dots \ge \lceil
                    \frac{c}{k} \rceil}$
            \end{itemize}
            Доказательство: вывод неравенства $0 \ge 1$ из неравенств, кодирующих дизъюнкты формулы.
        \pause
        \item Nullstellensatz.
            \begin{itemize}
                \item $\field$~--- поле;
                \item $(x \lor y \lor \neg z) \Rightarrow f_i \coloneqq (1 - x) (1 - y) z = 0$;
                \item $x^2 - x = 0$.
            \end{itemize}
            Доказательство: такой набор полиномов $h_i$, что $\sum\limits_{i} h_i f_i = 1$.
    \end{itemize}

\end{frame}
Основы сложности пропозициональных доказательств были заложены в работе \cite{!!!}. Рассмотрим язык $\UNSAT$

Рассмотрим двух игроков: Оптимист и Пессимист, у которых есть некоторый набор уравнений $\mathcal{H}
\coloneqq \{h_1, \dots, h_m\}$ на $n$ переменных $x_1, \dots, x_n \in \Omega$, где $\Omega$~--- некоторое
конечное пространство. Обычно мы предполагаем, что $\Omega \coloneqq \{0, 1\}$ и $h_i$ представляет собой
дизъюнкции наборов переменных или полиномиальные уравнения. Оптимист верит, что есть некоторая точка $a
\in \Omega^n$, удовлетворяющая всем ограничениям $h_i$, а Пессимист в противоположный факт.

Если прав Оптимист, то он может показать точку $a \in \Omega^n$, удовлетворяющую всем условиям,
Пессимисту. Предполагая, что Пессимист может вычислить значения $h_i$ в точке $a$ могут удостовериться,
что система $\mathcal{H}$ совместна. Если же прав Пессимист, то на текущий момент мы не знаем
<<простого>> способа убедить в этом оптимиста (предполагая, что вычислительная мощность Оптимиста
ограничена). Дадим формальное определение.

Системой доказательств для языка $L \subseteq \{0, 1\}^*$ будем называть такую полиномиально вычислимую
функцию $\Pi\colon \{0, 1\}^* \times \{0, 1\}^∗ \to \{0, 1\}$, что:
\begin{itemize}
    \item если $x \in L$, то существует такой $y \in {0, 1}^∗$, что $\Pi(x, y) = 1$ (будем говорить, что
        такой y~--- доказательство для x);
    \item если $x \notin L$, то для всех $y \in \{0, 1\}^∗, \Pi(x, y) = 0$.        
\end{itemize}

Основной задачей сложности доказательств является оценка размера кратчайших доказательств принадлежности 

The main task of proof complexity is to quantify the size of the smallest proof required to
prove that some given system $\mathcal{H}$ is unsatisfiable.

Let us focus on the main case: $\Omega \coloneqq \{0, 1\}$, $h_i$ is a disjunction and Optimist is a
polynomial-time Turing machine.


There are several motivations to pay attention to the sizes of proofs.
\begin{enumerate}
    \item Without limitations of the power of Optimist any superpolynomial lower bound on the sizes of proofs
        will imply that $\NP \neq \coNP$ \cite{CookRec79}. 
    \item Even if Optimist is limited to check the application of resolution rule:
        $$
            \frac{A \lor x ~~~~~ B \lor \neg x}{A \lor B},
        $$
        lower bounds on the size of proof will give us lower bounds on the running time of the
        popular algorithms for some $\NP$-complete problems \cite{DP60, AHI05}.
    \item Again if the focus on \textit{weak} Optimist models like Resolution or so-called
        Nullstellensatz, lower bounds will imply lower bounds on the strong enough computational models
        line \textit{monotone span programs} and \textit{monotone circuits} \cite{PR18, GGKS18}.
\end{enumerate}

In this talk we focus on the \textit{Restriction Technique} that is the most popular technique for
proving lower bounds on the size of proofs of unsatisfiability. And we discuss the limitations of this
technique and give some open problems.



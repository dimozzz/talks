Suppose we have two players Optimist and Pessimist that have a family of constraints $\mathcal{H}
\coloneqq \{h_1, \dots, h_m\}$ on $n$ variables $x_1, \dots, x_n \in \Omega$ where $\Omega$ is some
finite space. Usually we assume that $\Omega \coloneqq \{0, 1\}$ and $h_i$ is a disjunction of some subset
of variables or some small polynomial equation. Optimist believes that there is a point $a \in \Omega^n$
that satisfy all constraints and Pessimist believes in the opposite fact.

If Optimist is right then he can give a point $a \in \Omega^n$ that satisfy al constraints to
Pessimist. Assuming that Pessimist may check the feasibility of $a$ both players agreed that system
$\mathcal{H}$ is satisfiable. But if the Pessimist is right we do not know any simple way to convince the
Optimist that there is no solution of $\mathcal{H}$ (assuming that Optimist is a polynomial-time Turing
machine). The main task of proof complexity is to quantify the size of the smallest proof required to
prove that some given system $\mathcal{H}$ is unsatisfiable.

Let us focus on the main case: $\Omega \coloneqq \{0, 1\}$, $h_i$ is a disjunction and Optimist is a
polynomial-time Turing machine. There are several motivations to pay attention to the sizes of proofs.
\begin{enumerate}
    \item Without limitations of the power of Optimist any superpolynomial lower bound on the sizes of proofs
        will imply that $\NP \neq \coNP$ \cite{CookRec79}. 
    \item Even if Optimist is limited to check the application of resolution rule:
        $$
            \frac{A \lor x ~~~~~ B \lor \neg x}{A \lor B},
        $$
        lower bounds on the size of proof will give us lower bounds on the running time of the
        popular algorithms for some $\NP$-complete problems \cite{DP60, AHI05}.
    \item Again if the focus on \textit{weak} Optimist models like Resolution or so-called
        Nullstellensatz, lower bounds will imply lower bounds on the strong enough computational models
        line \textit{monotone span programs} and \textit{monotone circuits} \cite{PR18, GGKS18}.
\end{enumerate}

In this talk we focus on the \textit{Restriction Technique} that is the most popular technique for
proving lower bounds on the size of proofs of unsatisfiability. And we discuss the limitations of this
technique and give some open problems.

\begin{frame}{Notation}
    $$
        (x \lor y \lor z \lor \bar{w}) \land
        (x \lor \bar{y}) \land
        (\bar{x} \lor z) \land
        (w)   
    $$
        
    \pause
    \begin{minipage}{0.35\linewidth}
        \centering
        $
        \begin{cases}
            x \lor y \lor z \lor \bar{w} \\
            x \lor \bar{y} \\
            \bar{x} \lor z \\
            w
        \end{cases}
        $
    \end{minipage}
    \begin{minipage}{0.6\linewidth}
        \centering
        $
        \begin{cases}
            x y z (1 - w) = 0 \\
            x (1 - y) = 0 \\
            (1 -  x) z = 0 \\
            w = 0
        \end{cases}
        $
        $\bigwedge$
        $
        \begin{cases}
            x^2 - x = 0 \\
            y^2 - y = 0 \\
            z^2 - z = 0 \\
            w^2 - w = 0
        \end{cases}
        $
    \end{minipage}

    \pause
    \vspace{0.2cm}
    \begin{minipage}{0.45\linewidth}
        \centering
        \begin{tikzpicture}
            \node[jester, minimum size = 1cm, good] at (0, 0) {There is a solution!};
            \onslide<4->{
                \node[rectangle, rounded corners = 3pt, draw = LEIblue!15, fill = LEIblue!5] at (0, -1.5)
                    {\textcolor{blue}{Restricted} poly-time Turing machine};
            }
        \end{tikzpicture}
    \end{minipage}
    \begin{minipage}{0.45\linewidth}
        \centering
        \begin{tikzpicture}
            \node[jester, mirrored, evil, minimum size = 1cm] at (0, 0) {There is \alert{no}
                solution!};
            \onslide<4->{
                \node[rectangle, rounded corners = 3pt, draw = LEIblue!15, fill = LEIblue!5] at (0, -1.5)
                    {Infinite power};
            }
        \end{tikzpicture}
    \end{minipage}

    \pause
\end{frame}


\begin{frame}{Proof systems}

    The \deftext{Resolution} ($\Res$) proof of $\varphi \coloneqq \bigvee\limits_{i} C_i$ is a sequence
    of conjuctions $(D_1, D_2, D_3, \dots, D_{\ell})$:
    \pause
    \begin{itemize}
        \item $D_i \in \{C_i\}$;
        \pause
        \item $A \lor x = D_j$ and $B \lor \bar{x} = D_k$ for some $j, k < i$:
            $$\frac{A \lor x ~~~~~ B \lor \bar{x}}{D_i \coloneqq A \lor B};$$
        \pause    
        \item $D_{\ell} = \emptyset$.
    \end{itemize}

    \pause
    \vspace{0.3cm}

    The \deftext{Polynomial Calculus} ($\PCR[\field]$) proof of $\Fs$ is a sequence
    $(p_1, p_2, p_3, \dots, p_{\ell})$:
    \pause
    \begin{itemize}
        \item $p_i \in \Fs \cup \bigcup\limits_{j = 1}^{n} \{R_j\}$;
        \pause
        \item $p_i = x_j p_k$ for some $j$ and $k < i$;
        \pause    
        \item $p_i = \alpha p_k + \beta p_s$ for some $k, s < i$ and $\alpha, \beta \in \field$;
        \pause
        \item $p_{\ell} = 1$.
    \end{itemize}

    \pause
    \vspace{-0.2cm}
    \begin{minipage}{0.2\linewidth}
        $
        \begin{cases}
            x + y + z - 2 = 0 \\
            xy = 0 \\
            xz = 0 \\
            yz = 0 \\
        \end{cases}$
    \end{minipage}
    \pause
    \begin{minipage}{0.78\linewidth}
        \begin{prooftree}
            \AxiomC{$x^2 - x$}
            \AxiomC{$x + y + z - 2$}
            \UnaryInfC{$x^2 + xy + y z - 2x$}
            \AxiomC{$xy$}
            \AxiomC{$yz$}
            \def\defaultHypSeparation{\hskip.10in}
            \BinaryInfC{$xy + yz$}
            \BinaryInfC{$x^2 - 2x$}
            \BinaryInfC{$x$}
            \AxiomC{$\vdots$}
            \UnaryInfC{$y + z$}
            \BinaryInfC{$x + y + z$}
            \AxiomC{$x + y + z - 2$}
            \BinaryInfC{$1$}
        \end{prooftree}
    \end{minipage}
\end{frame}


\begin{frame}{Proof size}
    \begin{itemize}
        \item Explicit representations of polynomials:
            $$
                \sum\limits_{i} \alpha_i \prod\limits_{j} x_{i, j} \cdot \prod\limits_{j} (1 - x_{i, j})
            $$
        \pause
        \item Size is a number of ``monomials''.
        \pause
        \item Do \alert{not} pay for coefficients:
            \begin{itemize}
                \item because we can;
                \pause
                \item we can forget about representation problems in large fields.
            \end{itemize}
        \pause
        \item $x_i^2 - x_i = 0 \Rightarrow$ we can ``linearize'' the proof.
    \end{itemize}
    
\end{frame}

\begin{frame}{Motivation}

    \begin{itemize}
        \item If there is no short proofs of unsatisfiability $\Rightarrow$ no poly-time algorithm for
            $\SAT$ and $\P \neq \NP$.
        \pause
        \item Resolution lower bounds $\Rightarrow$ explicit hard examples for some classes of algorithms
            for $\SAT$.
        \pause
        \item Weak proof systems (Resolution etc.) lower bounds $\Rightarrow$ explicit hard examples for
            \alert{monotone} computations.
    \end{itemize}

    \pause
    \vspace{1cm}
    \begin{block}{Remark}
        For some applications we should prove lower bounds on the specific formulas.        
    \end{block}
\end{frame}

\begin{frame}{Naive restiction}

    Let $\Fs$ be a system of polynomial equations.

    $\Fs \circ \oplus_2$:
    \begin{itemize}
        \item replace $x_i$ by $y_i \oplus z_i$, where $y_i, z_i$ are fresh variables;
        \item transform it into polynomials over $\mathbb{F}$.
    \end{itemize}

    \vspace{0.5cm}
    \pause
    $\pi \coloneqq (p_1, \dots, p_{\ell})$ is a proof of $\Fs \circ \oplus_2$.
    \pause
    \begin{enumerate}            
        \item Partial assignment $\rho$: for each $i \in [n]$ choose either $y_i$ or $z_i$ and assign it
            $\{0, 1\}$ uniformly at random.
        \pause
        \item $t$ is multilinear $\Rightarrow$ $\Pr[t\rest \rho \neq 0] \le
            \left(\frac{3}{4}\right)^{\deg(t)}$.
        \item $(\Fs \circ \oplus_2) \rest \rho \equiv \Fs$.
        \pause
        \item $|\pi| < (\frac{4}{3})^{\varepsilon n} \Rightarrow$ whp $\pi\rest \rho$ is a proof of $\Fs$
            of degree $\varepsilon n$.
        \pause
        \item It is enough to show the degree lower bound...
    \end{enumerate}    
\end{frame}


\begin{frame}{Hierarchy}

    
\tikzset{
    vert/.style = {
        draw,
        ellipse
    },
    tikzart-fire/.pic = {
        \draw[fill = red!60] (0, 0) .. controls (0.3, 0) and (0.6, 0.1) .. (0.7, 0.3)
            .. controls (0.8, 0.5) and (0.85, 0.6) .. (0.8, 0.9)
            .. controls (0.75, 1.1) and (0.7, 1.2) .. (0.6, 1.4)
            .. controls (0.65, 1.2) and (0.6, 1.05) .. (0.5, 0.9)
            .. controls (0.5, 1.2) and (0.2, 1.3) .. (0.1, 1.6)
            .. controls (0.05, 1.75) and (0.1, 2) .. (0.2, 2.1)
            .. controls (-0.1, 2) and (-0.2, 1.85) .. (-0.3, 1.7)
            .. controls (-0.4, 1.5) and (-0.45, 1.3) .. (-0.4, 1.1)
            .. controls (-0.5, 1.2) and (-0.51, 1.4) .. (-0.5, 1.5)
            .. controls (-0.75, 1.2) and (-0.8, 0.7) .. (-0.7, 0.5)
            .. controls (-0.6, 0.28) and (-0.4, 0) .. (0, 0);
            \fill[white] (0, 0) .. controls (0.3, 0) and (0.52, 0.34) .. (0.37, 0.61)
            .. controls (0.4, 0.54) and (0.32, 0.32) .. (0.25, 0.25)
            .. controls (0.3, 0.35) and (0.25, 0.5) .. (0.2, 0.6)
            .. controls (0.1, 0.8) and (-0.05, 1) .. (0, 1.2)
            .. controls (-0.32, 1) and (-0.3, 0.72) .. (-0.2, 0.47)
            .. controls (-0.3, 0.51) and (-0.31, 0.6) .. (-0.33, 0.7)
            .. controls (-0.4, 0.6) and (-0.4, 0.5) .. (-0.4, 0.4)
            .. controls (-0.35, 0.18) and (-0.2, 0) .. (0, 0);
    },
    semisim/.style = {
        ->,
        blue,
        dashed,
        decorate,
        decoration = {
            snake,
            amplitude = 0.5,
            segment length = 2
        }
    },        
}


    
\begin{tikzpicture}[xscale = 1.3, xshift = -1]
    \node[vert] (res) at (1, 0) {$\Res$};
    \node[vert] (ns) at (-3, 0) {$\NS$};
    \node[vert] (cp) at (3, 1) {$\CP$};
    \node[vert] (resk) at (1.2, 1.4) {$\Res(k)$};
    \node[vert] (acf) at (1.3, 2.4) {$\AC_0$-Frege};
    \node[vert] (resl) at (-1.3, 2.7) {$\ResL$};
    \node[vert] (acfp) at (0.5, 3.8) {$\AC_0[p]$-Frege};
    \node[vert] (fre) at (0.5, 5) {Frege};
    \node[vert] (ips) at (-2, 6) {$\PrSys{IPS}$};
    \node[vert] (pcr) at (-2.8, 1.9) {$\PCR[]$};
    \node[vert] (sos) at (-4, 2.5) {$\SOS$};
    
    \node[vert] (cps) at (-4, 6.5) {$\PrSys{CPS}$};

    

    \draw[->, semisim] (pcr) -- (sos);
    \draw[->] (res) -- (cp);
    \draw[->] (cp) to[out = 90, in = -20] (fre);
    \draw[->] (res) -- (resl);
    \draw[->] (res) -- (resk);
    \draw[->] (resk) -- (acf);
    \draw[->] (res) to[out = 138, in = -30] (pcr);
    \draw[->] (ns) -- (pcr);
    \draw[->] (resl) -- (acfp);
    \draw[->] (acf) -- (acfp);
    \draw[->] (acfp) -- (fre);
    \draw[->] (fre) -- (ips);
    \draw[->, semisim] (ips) -- (cps);

    \draw[->] (pcr) -- (ips);
    \draw[->] (sos) -- (cps);

    \node at (0, 6.9) {};
    

    \begin{scope}[on background layer]
        \draw[ultra thick, fill = black!5] (-4, -1) to[out = 110, in = 220] (-4.4, 3)
            to[out = 40, in = 180] (-1, 2) to[out = 0, in = 125] (2.3, 3) to[out = -55, in = 90]
            (1.5, -1);
    \end{scope}
    \node[blue] at (-1.5, 0.95) {Restriction};


    \begin{scope}[on background layer]
        \fill[orange!5] (-4, -1) to[out = 90, in = 180] (-2.3, 0.75) to[out = 0, in = 180] (0.5, 0.7)
            to[out = 0, in = 200] (3.5, 2) -- (3.5, -1) -- cycle;
        \draw[ultra thick, orange] (-4, -1) to[out = 90, in = 180] (-2.3, 0.75) to[out = 0, in = 180]
            (0.5, 0.7) to[out = 0, in = 200] (3.5, 2);
        \draw[ultra thick] (-4, -1) to[out = 110, in = 220] (-4.4, 3) to[out = 40, in = 180] (-1, 2)
            to[out = 0, in = 125] (2.3, 3) to[out = -55, in = 90] (1.5, -1);
    \end{scope}
    \node[blue] at (0, -0.7) {Mon. Interpolation};
\end{tikzpicture}
    
\end{frame}
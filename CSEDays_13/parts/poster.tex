\begin{frame}{}

    \begin{block}{Abstract} \justifying
        We consider heuristic proof
		systems and give non-trivial examples of proof
        systems of this kind. We give an example of a distributional problem $(Y,D)$
        that is in the complexity class $\rm HeurNP$ but $Y\notin \rm NP$ if $\rm
        NP \neq coNP$, and $(Y, D)\notin \rm HeurBPP$ if $(\rm NP, PSamp) \not
        \subseteq HeurBPP$. 

		For a language $L$ and a polynomial $q$ we define the language $L_q$ composed
        of pairs $(x, r)$ where $x$ is an element of $L$ and $r$ is an arbitrary
        binary string of length $q(|x|)$.  If $D = \{D_n\}_{n=1}^\infty$ is an
        ensemble of distributions on strings, let $D \times U^q$ be a distribution on
        pairs $(x, r)$, where $x$ is distributed according to $D_n$ and $r$ is
        uniformly distributed on strings of length $q(n)$.  We show that for every
        language $L$ in $\rm AM$ there is a polynomial $q$ such that for every
        distribution $D$ concentrated on the complement of $L$ the distributional
        problem $(L_q, D \times U^q)$ has a polynomially bounded heuristic proof
        system. Since graph non-isomorphism ($GNI$) is in $\rm AM$, the above result
        is applicable to $GNI$.
    \end{block}

    
\begin{columns}[t]
	\begin{column}{0.32\linewidth}

            
		\begin{block}{Preliminaries}
                {\bf \em An ensemble of distributions} $\{D_n\}_{n=1}^{\infty}$ is a
                sequence of distributions $D_i : \{0,1\}^* \to [0, 1]$ with a finite
                support. Here $n$ is a complexity parameter, it is called a security
                parameter in cryptography. A support of a distribution $D_n$ is a set
                of strings, such that $D_n(x) > 0$, we denote it as $\supp D_n$.

                The ensemble $U$ denotes the uniform ensemble, that is $U_n(x) =
                2^{-n}$ for all $x\in \{0,1\}^n$.

				For two ensembles of distribution $D$ and $F$ we define an ensemble
                $\alpha D + (1 - \alpha) F$ for $0 \le \alpha \le 1$ such that
                $(\alpha D + (1 - \alpha) F)_n(x) = \alpha D_n(x) + (1 - \alpha)
                F_n(x)$. We also define an ensemble of distributions $D \times F$
                concentrated on pairs of strings as follows: $(D \times F)_n(x, y) =
                D_n(x) F_n(y)$

                {\bf \em A distributional (decision) problem} is a pair $(L, D)$
                where $L$ is a language and $D$ is an ensemble of distributions.
		\end{block}

        \begin{block}{Heuristic classes}
	        $\rm HeurP (HeurNP, HeurBPP)$ is a class of distributional decision
            problems $(L,D)$ that have deterministic (non-deterministic and
            randomized, respectively) algorithm $A(x, \delta; n)$ and polynomial
            $p$ such that:
			\begin{enumerate}
				\item For all $x \in \supp D_n$ the running time of $A(x, \delta; n)$
		            is bounded by  $p(\frac{n}{\delta})$; 
				\item $\Pr_{x \gets D_n} [A(x, \delta; n) \neq L(x)] < \delta$, in
		            the case of a randomized algorithm the probability is taken also
                    on random bits of the algorithm $A$.
			\end{enumerate}            
        \end{block}

        \begin{block}{Heuristic proof systems}
            {\bf \em A distributional proving problem} is a pair $(L,D)$ where $L$ is a
			language, $D$ is an ensemble of distributions concentrated in the
            complement of the language $L$.

			The following definition is a deterministic version of a randomized
            heuristic proof system from \cite{HIMS12}.

            {\bf A heuristic proof system} for a distributional proving problem
            $(L, D)$ is an algorithm $\Pi(x, w, \delta; n)$ that satisfies the
            following properties: 
			\begin{enumerate}
				\item For every $x \in L$ there is $w \in \{0, 1\}^*$, such that
		            $\Pi(x, w, \delta; n)=1$;
				\item $\Pr_{x\gets D_n}[\exists w \Pi(x, w, \delta; n] = 1) <
		            \delta$;
				\item The running time of $\Pi(x, w, \delta; n)$ is bounded by a
		            polynomial in $\frac{n + |w|}{\delta}$. 
			\end{enumerate}
			A heuristic proof system is called polynomially bounded if there exists
            such polynomial $p$ that for every $x\in L$ there is $w$ that 
			$\Pi(x, w, \delta; n) = 1$ and the size of $w$ is bounded by
            $p(\frac{n}{\delta})$.
        \end{block}
        
    \end{column}%1

    %-- Column 2 ---------------------------------------------------
	\begin{column}{0.32\linewidth}
        
		\begin{block}{Motivation}
			\begin{itemize}
			    \item Structural properties (optimal randomized acceptor, hierarchy
		            theorem, complete language...). 
		    	\item Practice.
		        \item New methods and bounds for classical case.
		    \end{itemize}

            Is there a gap between classical and heuristic case on {\em natural}
            language?
        \end{block}

        \begin{block}{Results}
            $L$ is a language.
		    $L_{pad} = \{(x, r) \mid x \in L, |r| \in \{0, 1 \}^{p(|x|)}\}$.
    
		    \begin{enumerate}
				\item $\exists L \in PP$-complete, $\forall D~
		    		(L_{pad}, D \times U) \in HeurP$.
		    	\item $\exists L \in (PP \cdot NP)$-complete, $\forall D~
				    (L_{pad}, D \times U) \in HeurNP$.
		    	\item $\forall L \in AM$ there exists a polynomially bounded
    				heuristic proof system for $L_{pad}$.
            \end{enumerate}
        \end{block}

        \begin{alertblock}{Theorem}
            Let $f:\{0,1\}^* \to \{0,1\}$ be a function. Denote the language
            $L_f = \cup_{n \in \mathbb{N}} \{r \in \{0, 1\}^n \mid 0.r <
            \overline{f_n} \}$, where $\overline{f_n} = \frac{1}{2^n}
            \sum_{x \in \{0,1\}^n}f(x)$.

            If $f$ is computed in (nondeterministic) polynomial time then $(L_f,U)
            \in \rm Heur(N)P$. 
        \end{alertblock}

        \begin{block}{Algorithm for $L_{pad}$}
            A Boolean sampler is a randomized algorithm $S$, which takes on the input
			a integer number $n$ and rational numbers $\epsilon, \delta$.  $S$ has an
            oracle access to a function $f:\{0,1\}^n \to \{0,1\}$; $S$ makes several
            non-adaptive requests to the function $f$ and outputs a number in the
            range $[0,1]$. Let's denote $\overline{f} =
            \frac{1}{2^n}\sum_{x \in \{0,1\}^n}f(x)$. Every function
            $f:\{0,1\}^n \to \{0,1\}$ must satisfy the following inequality: 
			$\Pr[|S^f(n,\epsilon,\delta)-\overline{f}|\ge \epsilon]<\delta$.  A
            Boolean sampler is called averaging if it outputs the arithmetical mean
            of requested values.

            We describe the algorithm $A(x,\delta)$  (we can omit $n$ as it equals to $|x|$).
			Let $S$ be an averaging Boolean sampler \cite{Gol11}.
            \begin{itemize}
				\item If $\delta \le 2^{-n+2}$ then compute $\overline{f}$, and
		            output $1$ if $0.x < \overline{f_n}$, and $0$ otherwise. 
				\item Run the sampler $S^f(n, \delta / 4, \delta / 4)$ that uses the
		            string $x$ instead of random bits.
					Let $\nu_n$ be it`s result. If $\nu_n> 0.x$ then return $1$ else
                    return $0$. 
			\end{itemize}
        \end{block}
        
    \end{column}%2

    %-- Column 3 ---------------------------------------------------
    \begin{column}{0.32\linewidth}
        \begin{block}{Languages}
            Consider the language $X = \cup_{n \in \mathbb{N}}\{(C, r) \mid \mbox{$C$
            is a circuit with $n$ inputs}, r \in \{0, 1\}^n, \sharp C > r\}$, where
        	$\sharp C$ is a number of satisfying assignments of the circuit $C$. 

            Consider the language $Y = \cup_{n \in \mathbb{N}}\{(C,r) \mid \mbox{$C$
            is a non-deterministic circuit with $n$ inputs}, r \in \{0, 1\}^n, \sharp
          	C > r\}$, where $\sharp C$ is a number of satisfying assignments of the
            circuit $C$, in other words, if the circuit $C$ uses $m$ bits of the
            witness, then $\sharp C = |\{x \in \{0,1\}^n \mid \exists y \in
            \{0,1\}^m: C(x,y) = 1\}|$.
        \end{block}

        \begin{block}{$PP$, $PP \cdot NP$}
            The complexity class $\rm PP$ consists of languages $L$ that 
            have a randomized polynomial time Turing machine $M$ such that for every
            $x$, $x \in L \iff \Pr_M[M(x) = L(x)] \ge 1 / 2$.

            The class $\rm PP \cdot NP$ consists of languages $L$ that have a
			randomized polynomial time Turing machine $M$ such that for every $x$,
            $x \in L$ if and only if $\Pr[M(x)\in SAT]\ge \frac12$, where $SAT$ is
            the language of satisfiable propositional formulas.
        \end{block}

        \begin{alertblock}{Theorem}
        	\begin{enumerate}
				\item The language $X$ is $PP$-complete.
		    	\item Let $D$ be a polynomially bounded ensemble and let a
				    support of $D_n$ consist of circuits with $n$ inputs, then 
		            $(X, D \times U) \in HeurP$.
                \item The language $Y$ is $PP \cdot NP$-complete.
		    	\item Let $D$ be a polynomially bounded ensemble and let a
				    support of $D_n$ consist of circuits with $n$ inputs, then
		            $(X, D \times U) \in HeurNP$.
		    \end{enumerate}
        \end{alertblock}

        \begin{block}{Short proofs}
            Let $q$ be a polynomial, for every language $L$ we denote $pad_q(L) = 
			\{(x, r) \mid x \in L, r \in \{0, 1\}^* , |r| \ge q(|x|)\}$.
        \end{block}

        \begin{alertblock}{Theorem}
            Let a language $L \in AM$ and let an ensemble $D$ be polynomially 
		    bounded, then:
		    \begin{enumerate}
    			\item There is a polynomial $q$ such that for $(pad_q(L),
				    D \times U(q)) \in HeurNP$.
        		\item Then there is a polynomial $q$ such that for the distributional
		            proving problem $(pad_q(L), D \times U(q))$ 
        		    there is a polynomially bounded proof system.
		    \end{enumerate}
        \end{alertblock}


        \begin{block}{}
        	\bibliography{main}    
        \end{block}
    \end{column}%3

\end{columns}

\end{frame}

%%% Local Variables: 
%%% mode: latex
%%% TeX-master: t
%%% End: 

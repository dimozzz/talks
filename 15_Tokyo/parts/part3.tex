\begin{frame}
    \frametitle{$\ResL$}

    \begin{itemize}
		\item Literal: $L = x_{1} + x_{2} + \dots + x_{k} = \alpha$,
    		$\lnot L = x_{1} + x_{2} + \dots + x_{k} = \alpha + 1$.
		\item Linear clause: $\bigvee\limits_i x_{i, 1} + x_{i, 2} + \dots + x_{i, k_i} =
		    \alpha_i$ or $\lnot \bigwedge\limits_i x_{i, 1} + x_{i, 2} + \dots + x_{i, k_i} =
            1 + \alpha_i$.
        \pause
		\item $\ResL$:
			\begin{itemize}
				\item Resolution rule: $\frac{(f = 0) \lor D \qquad (f = 1) \lor D'}
            		{D \lor D'}$ 
				\item Weakening rule: $\frac{C}{C'}$ if $C$ semantically implies $C'$
			\end{itemize}
        \pause
		\item \mycolor{blue}{Tree-like} $\ResL$ is equivalent to LST.
	\end{itemize}
	\scalebox{0.8}{\tikzstyle{end} = [circle, minimum size = 0.1cm, draw, inner sep = 0.1pt]
\tikzstyle{leaf} = [circle, minimum size = 0.6cm, draw, inner sep = 0.1pt, blue]


\onslide<1->{
	\tikzstyle{level 1}=[level distance = 1.5cm, sibling distance = 6cm, opacity = 0]
	\tikzstyle{level 2}=[sibling distance = 3cm, opacity = 0]
}
\only<3->{
	\tikzstyle{level 1}=[level distance = 1.5cm, sibling distance = 6cm, opacity = 1]
	\tikzstyle{level 2}=[sibling distance = 3cm, opacity = 1]
}






    
\begin{tikzpicture}[label distance = 8mm, all/.style = {opacity = 0}]
	\node [end] {}
        child {
            node {\mycolor{blue}{$\neg (f = 0)$}}
           	child {
                node {\mycolor{blue}{$\neg (f = 0 \land g = 0)$}}
                edge from parent
	            node[left] {$g = 0$}
            }
            child {
               	node {\mycolor{blue}{$\neg (f = 0 \land g = 1)$}}
                edge from parent
	            node[right] {$g = 1$}
            }
           	edge from parent
            node[left] {$f = 0$}
        }
        child {
            node {\mycolor{blue}{$\neg (f = 1)$}}
           	child {
                node {\mycolor{blue}{$\neg (f = 1 \land h = 0)$}}
                edge from parent
	            node[left] {$h = 0$}
            }
            child {
               	node {\mycolor{blue}{$\neg (f = 1 \land h = 1)$}}
                edge from parent
	            node[right] {$h = 1$}
            }
           	edge from parent
            node[right] {$f = 1$}
        };
\end{tikzpicture}

%%% Local Variables: 
%%% mode: latex
%%% TeX-master: t
%%% End: 
}
\end{frame}



\begin{frame}
    \frametitle{$\ResL$ vs $\SemL$}

    \begin{itemize}
		\item $\SemL$:
			\begin{itemize}
				\item Semantic rule: $\frac{C_1 \qquad C_2}{C'}$ if $C_1 \land C_2$
		            semantically implies $C'$.
			\end{itemize}
	\end{itemize}

	\pause
    \begin{theorem}
        \begin{enumerate}
            \item $\ResL$ p-simulates $\SemL$.
        	\pause
        	\item $\ResL$ is implication complete.
        \end{enumerate}
    \end{theorem}

    \pause

    The weakening rule may be simulated by a polynomial number of pure syntactic rules.
\end{frame}



\begin{frame}
    \frametitle{$\mathrm{R(lin)}$}

    $\mathrm{R(lin)}$ [Raz, Tsameret, 2007] operates with disjunction of linear equalities with
    \mycolor{blue}{integer} coefficients.
    
	\begin{itemize}
		\item Resolution rule: $\frac{A \lor (F_1 = a_1) \qquad B \lor (F_2 = a_2)}
    		{A \lor B \lor (F_1 \pm F_2 = a_1 \pm a_2)}$
		\item Weakening rule: $\frac{A}{A \lor (F = a)}$
		\item Simplification rule: $\frac{B \lor (0 = c)}{B}$, where $c \neq 0$. 
	\end{itemize}

	\pause
    \begin{theorem}
    	$\mathrm{R(lin)}$ p-simulates $\ResL$.    
    \end{theorem}
    
	\medskip

	\pause
	\begin{columns}
		\begin{column}{5cm}
			$x_1 + x_2 + \dots + x_n \equiv 0 \bmod 2$:
			$\left[\begin{aligned}
            	x_1 + &\dots + x_n = 0\\
                x_1 + &\dots + x_n = 2\\
                &\dots \\
                x_1 + &\dots + x_n = 2 \lceil n / 2 \rceil
            \end{aligned}\right.$ 
		\end{column}
		\begin{column}{5cm}
			$x_1 + x_2 + \dots + x_n \equiv 1 \bmod 2$:
			$\left[\begin{aligned}
                x_1 + &\dots + x_n = 1\\
                x_1 + &\dots + x_n = 3\\
                &\dots \\
                x_1 + &\dots + x_n = 2 \lceil (n - 1) / 2 \rceil + 1
            \end{aligned}\right.$
		\end{column}
	\end{columns}
\end{frame}


\begin{frame}
    \frametitle{2-fold Tseitin formulas}

    \begin{itemize}
        \item $\phi(x_1, \dots, x_n)$ is a Tseitin formula;
        \item $x_i \to z_{i1} \land z_{i2}$;
    	\item rewrite in CNF.
	\end{itemize}

	\pause
    \begin{theorem}
        In time polynomial in $n$ one may construct a graph $G(V, E)$ on $n$ vertices with maximal degree bounded by a
        constant and a function $c: V \to \mathbb{F}_2$ such that the size of any \mycolor{blue}{tree-like} proof in
        $\ResL$ of $TS^2_{(G,c)}$ is at least $\Omega\left(2^{n^{1 / 3} / \log^3(n)} \right)$.
    \end{theorem}
\end{frame}

\begin{frame}
    \frametitle{2-fold Tseitin formulas}
    
    \begin{itemize}
        \item Alice and Bob know a different part of assignment.
        \pause
		\item For unsatisfiable $\phi$ a search problem $Search_\phi$: find a falsified clause.
        \pause
        \item{} [Beame et al., 2007] There exists a graph $G$, $|G| = n$ such that $DISJ_{m} \le Search_{TS_{G}}$
			where $m = \frac{n^{1 / 3}}{\log^2(n)}$.
		\item We prove that it is possible to transform tree-like proof $T$ into a randomized communication protocol for
			the problem $Search_\phi$ of depth $O(\log |T| \log\log |T|)$.
        \pause
	\end{itemize}
\end{frame}




\begin{frame}
    \frametitle{Space vs. Size Tradeoff}

    \pause
    \begin{itemize}
    	\item Write a basic clause to memory.
    	\item Remove a clause from memory.
    	\item Deduce a clause from clauses in memory and add it in memory.
    \end{itemize}

    \pause
    $CS(\pi)$ is a maximal number of clauses in memory in proof $\pi$.

    \pause
    \begin{theorem}
        Let $\pi$ is a $\ResL$ proof of formula $\phi$ then $R_{1 / 3}(Search_{\phi}) = O(\mu(\pi) \log(\mu(\pi)))$,
        where $\mu(\pi) = CS(\pi) \log(size(\pi))$.
    \end{theorem}

    $R_{1 / 3}$ is a complexity of a randomized communication protocol.
    
    \pause

    $R_{1 / 3}(Search_{TS_{G,c}^2}) = \Omega(n^{1 / 3} / \log^2(n))$
\end{frame}


\begin{frame}
    \frametitle{$\ResL$ vs. Interpolation}

    \begin{itemize}
        \item Alice and Bob know a different part of assignment.
    	\pause
        \item They want to evaluate a clause by \mycolor{blue}{deterministic} protocol.
    	\pause
        \item $DISJ_{n}$ is not harder than this problem.
    	\pause
        \item{} [Cook, Pitassi et al. 2013] Some barriers on randomized Karchmer-Wigderson game.
    	\pause
    	\item{} \mycolor{blue}{Open question}. Maybe in our case we have an analogue of Karchmer-Wigderson game.
	\end{itemize}
\end{frame}


\begin{frame}
    \frametitle{Open question}

    \begin{itemize}
		\item Prove lower bound for dag-like $\ResL$.
		\item What is tree-like $\ResL$ complexity of the Perfect Matching Principle for $K_{n - 2, n}$?
		\item Does PCR p-simulate $\ResL$ (maybe in the other direction)?
		\item Prove lower bound for splitting by linear combinations on \mycolor{blue}{satisfiable} formulas.
	\end{itemize}
\end{frame}


%%% Local Variables: 
%%% mode: latex
%%% TeX-master: t
%%% End: 

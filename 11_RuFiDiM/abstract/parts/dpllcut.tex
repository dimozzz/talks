We introduce a lower bounds on the time complexity of DPLL
algorithms that solve the satisfiability problem using a splitting
strategy. We prove the theorem about effectiveness vs. correctness
trade-off for deterministic myopic DPLL algorithms with cut heuristic.

DPLL (are named by the authors: Davis, Putnam, Logemann and Loveland)
algorithms are one of the most popular approach to the problem of
satisfiability of Boolean formulas (SAT). DPLL algorithm is a
recursive algorithm that takes the input formula $\phi$, uses a
procedure $\mathbf{A}$ to choose a variable $x$, uses a procedure
$\mathbf{B}$ that chooses the value $a \in \{0, 1\}$ for the variable
$x$ that would be investigated first, and makes two recursive calls on
inputs $\phi[x := a]$ the $\phi[x := 1 - a]$.  Note that the second
call is not necessary if the first one returns the result, that the
formula is satisfiable.

There is a number of works concerning lower bounds for DPLL
algorithms: for unsatisfiable formulas exponential lower bounds follow
from lower bounds on the complexity of resolution proofs \cite{Urq87},
\cite{Tse68}. In case of satisfiable formulas we have no hope to prove
superpolynomial lower bound since if $\P = \NP$, then procedure
$\mathbf{B}$ may always choose the correct value of the variable
according to some satisfying assignment. The paper \cite{AHI05} gives
an exponential lower bounds for two wide enough classes of DPLL
algorithms: myopic and drunken algorithms.  In the myopic case
procedures $\mathbf{A}$ and $\mathbf{B}$ can see formula with erased
signs of negation, they can request the number of positive and
negative occurrences for every variable and also may read $K = n^{1 -
\epsilon}$ clauses precisely.  The paper \cite{CEMT09} shows that
myopic algorithms invert Goldreich's function (\cite{Gol00}) based on
a random graph in at least exponential time, \cite{I10} extends this
result for drunken algorithms.  The paper \cite{IS11} presents the
explicit Goldreich's function based on any expander that is hard to
invert for drunken and myopic algorithms.

All lower bounds for satisfiable instances are based on the fact that
during several first steps algorithm falls into a hard unsatisfiale
formula, and algorithm should investigate the whole it' splitting
tree.  In this work we extend the class of DPLL algorithms by adding
the procedure $\mathbf{C}$ that may decide that some branch of the
splitting tree will not be investigated since it is not too
``perspective''.  More precisely, before each recursive call an
algorithm calls the procedure $\mathbf{C}$ that decides whether to
make this recursive call or not.  DPLL algorithms with cut heuristic
are always give a correct answer on unsatisfiable formulas; however
they may err on satisfiable formulas. On the other hand if the
presence of a cut heuristic gives the substantial improvement on the
time complexity while the bad instances (i.e. instances on which the
algorithm errs) are not easy to find, then such algorithms become
reasonable.

In this work we show that it is possible to construct the family of
unsatisfiable formulas $\Phi^{(n)}$ in polynomial time such that for
every myopic deterministic heuristics $\mathbf{A}$ and $\mathbf{C}$
there exists a polynomial time samplable ensemble of distributions
$R_n$ such that the DPLL algorithm based on procedures $\mathbf{A, B}$
and $\mathbf{C}$ for some $\mathbf{B}$ either errs on $99\%$ of random
inputs according $R_n$ or runs exponential time on formulas
$\Phi^{(n)}$.  In case $\mathbf{A}$ and $\mathbf{C}$ are not
restricted we show that a statement similar to above is equivalent to
$\P \neq \NP$.  The case of randomized myopic procedures $\mathbf{A}$
and $\mathbf{C}$ is left open.


\paragraph{Heuristic acceptors.}  The study of DPLL algorithms with
cut heuristic was also motivated by the study of heuristic acceptors.
\cite{HI10}, \cite{HIMS10}
The distributional proving problem is a pair $(L, D)$ of a language
$L$ and a polynomial time samplable distribution $D$ concentrated on
the complement of $D$. An algorithm $A$ is called a heuristic acceptor
if it has additional input $d$ that represents the parameter of the
error and for every $x \in L$ and $d \in \mathbb{N}$, $A(x, d)$
returns $1$ and $\Pr_{x\gets D_n}[A(x) = 1] < 1/d$ for every integer
$n$.  We call an acceptor polynomially bounded if for every $x \in L$
running time of $A(x, d)$ is bounded by polynomial in $|x| \cdot d$.
The paper \cite{HIMS10} shows that the existence of distributed proving
problems that have no polynomially bounded acceptors is equivalent to
the existence of infinitely often one-way functions.

Let $D$ be some distribution concentrated on satisfiable formulas. We
consider DPLL algorithm with a cut heuristic supplied with an
additional parameter $d$ that is available for procedures $\mathbf{A,
B, C}$. We call such an algorithm a heuristic DPLL acceptor if it
satisfies the definition of a heuristic acceptor. Our result implies
that there are no deterministic polynomially bounded myopic DPLL
acceptors for the proving problem $(UNSAT, Q)$.
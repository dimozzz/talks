\newcommand{\ba}{{\color{blue} $\boldsymbol{\forall}$}}
\newcommand{\be}{{\color{blue} $\boldsymbol{\exists}$}}

\begin{frame}{Следствия}
    $m = n^{O(1)}$, $w(\varphi)$~--- минимальная ширина резолюционного доказательства $\varphi$.

    \pause
    \begin{enumerate}
        \item \ba $\varphi$ число строк в $\CP$-доказательстве $\varphi \circ \Ind_m$ не менее
            $n^{w(\varphi)}$.
        \pause
        \item \textbf{($\CP$ vs. $\NS$)} \be $\varphi$, что:
            \begin{itemize}
                \item над любым полем $\mathbb{F}$ формула $\varphi$ имеет
                    $\NS_{\mathbb{F}}$-доказательство степени $O(\log(n))$;
                \item любое $\CP$-доказательство $\varphi$ имеет размер не менее $2^{n^{\varepsilon}}$,
                    где $\varepsilon > 0$.
            \end{itemize}
        \pause
        \vspace{0.2cm}
        \item \be $f$, которая может быть посчитана при помощи $mSP$ размера $\poly(n)$ над любым полем,
            но любая монотонная вещественная схема для $f$ имеет размер не менее $2^{n^{\varepsilon}}$,
            где $\varepsilon > 0$.
        \pause
        \item \be $f \in \NC^2$, что любая монотонная вещественная схема для $f$ имеет размер не менее
            $2^{n^{\varepsilon}}$, где $\varepsilon > 0$.
    \end{enumerate}
\end{frame}

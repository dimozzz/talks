\begin{frame}
    \frametitle{План}
\end{frame}

\begin{frame}
	\frametitle{Системы доказательств}

    \begin{definition}
        Полиномиально ограниченый алгоритм $\Pi$~--- система доказательств для языка
        $L$, если:
        \begin{itemize}
            \item $\forall x \in L~ \exists \omega \in \{0, 1\}^{*}, \Pi(x, \omega) = 1$;
        	\item $\forall x \notin L~ \forall \omega \in \{0, 1\}^{*}, 
        		\Pi(x, \omega) = 0$.
        \end{itemize}
    \end{definition}

    Система доказательств полиномиально ограничена, если существует такой полином
    $p$, что:
    $\forall x \in L~ \exists \omega \in \{0, 1\}^{p(|x|)}, \Pi(x, \omega) = 1$.
\end{frame}

\begin{frame}
    \frametitle{Эвристические системы доказательств}

    $L$~--- язык, $D$~--- распределение на дополнении $L$.

    Разрешим системе доказательств ошибаться на небольшой доле долнения языка $L$.

    
    
    %НЕФОРМАЛЬНО%

    %ФОРМАЛЬНО%

    %Нет примеров%
\end{frame}

\begin{frame}
    \frametitle{Вспомогательная задача. Подсчет среднего}

    $f:\{0, 1\}^{n} \rightarrow \{0, 1\}$
    $\overline{f} = \frac{\sum\limits_{i = 0}^{2^n - 1}f(i)}{2^n}$

    $f$ может быть посчитана детерминированным алгоритмом за время $t$.

    \begin{proposition}%[]
        Существует такой алгоритм $A$, что:
        \begin{itemize}
	        \item $A$ полиноален относительно $\frac{n}{\epsilon\delta}$
        	\item $A$ использует $n$ случайных битов.
        	\item $\Pr[|A^{f}(n, \epsilon, \delta) - \overline{f}| \ge \epsilon] <
		        \delta$
        \end{itemize}
    \end{proposition}

    Определим язык $L_f$.

    $L_f = \bigcup\limits_{n \in \mathbb{N}}{r \in {0, 1}^n \mid 0.r < \overline{f}}$

    \begin{lemma}
        $L_f \in HeurP$
    \end{lemma}
\end{frame}

\begin{frame}
    \frametitle{Вспомогательная задача. Подсчет среднего}

    Данный результат можно обобщить на случай недетерминированной функции.

    \begin{proposition}%[]
        Существует недетерминированный такой алгоритм $A$, что:
        \begin{itemize}
	        \item $A$ полиномиален относительно $\frac{n}{\epsilon\delta}$
        	\item $A$ использует $n$ случайных битов.
        	\item $\Pr[|A^{f}(n, \epsilon, \delta) - \overline{f}| \ge \epsilon] <
		        \delta$
        \end{itemize}
    \end{proposition}

    $L_f = \bigcup\limits_{n \in \mathbb{N}}\{r \in {0, 1}^n \mid 0.r < \overline{f}\}$
	$Y = \{(C, x) \mid C$~--- схема $x \in \{0, 1\}^n, x \in L_{C}\}$
    
    \begin{lemma}
        $Y \in HeurNP$
    \end{lemma}

    \begin{proposition}
        $Y \in SpanP-complete$
    \end{proposition}
    $\#P \in SpanP$, $Perm \in \#P$
\end{frame}

\begin{frame}
    \frametitle{Неизоморфизм}

    $GNI \in coNP$
    
    $GNI \in NP$~--- открытый вопрос. Следовательно нет примеров полиномиально
    ограниченных систем доказательств.

    Определим новый язык.

    \begin{definition}
        $GNI_{pad} = \{(G_1, G_2, r) \mid (G_1, G_2) \in GNI,
        r \in \{0, 1\}^{q(n)}\}$, $q$~--- полином.
    \end{definition}

    \begin{lemma}
        Если существует полиномально ограниченная системадоказательств для
        $GNI_{pad}$, то $GNI \in NP$.
    \end{lemma}
\end{frame}

\begin{frame}
    \frametitle{Результаты про неизоморфизм}

    \begin{theorem}
        Существует полиномиально ограниченная эвристическая система доказальств для
        $GNI_{pad}$.
    \end{theorem}

    \begin{proof}[sketch]
        $GNI \in AM$
    \end{proof}
    $GNI \in AM$
\end{frame}
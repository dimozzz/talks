\begin{frame}
    \frametitle{План}
\end{frame}

\begin{frame}
	\frametitle{Proof systems}

    \begin{definition}
        Proof system for language $L$:
        \begin{itemize}
            \item Polynomially bounded algoithm $\Pi$;
            \item $\forall x \in L~ \exists \omega \in \{0, 1\}^{*}, \Pi(x, \omega) = 1$;
        	\item $\forall x \notin L~ \forall \omega \in \{0, 1\}^{*}, 
        		\Pi(x, \omega) = 0$.
        \end{itemize}
    \end{definition}

    Proof system is polynomially bounded is there exists such polynomial $p$ that:
    $\forall x \in L~ \exists \omega \in \{0, 1\}^{p(|x|)}, \Pi(x, \omega) = 1$.
\end{frame}

\begin{frame}
    \frametitle{Heuristic proof system}

    $L$ is a language, $D$ is a distributional on comlement $L$.

    Allow proof system to have a mistakes on complement $L$.

    
    
    %НЕФОРМАЛЬНО%

    %ФОРМАЛЬНО%

    %Нет примеров%
\end{frame}

\begin{frame}
    \frametitle{Вспомогательная задача. Подсчет среднего}

    $f:\{0, 1\}^{n} \rightarrow \{0, 1\}$
    $\overline{f} = \frac{\sum\limits_{i = 0}^{2^n - 1}f(i)}{2^n}$

    $f$ can be computed by deterministic algorithm in time $t$.

    \begin{proposition}%[]
        There exists algorithm $A$ that:
        \begin{itemize}
	        \item $A$ is polynomially bounded in $t\frac{n}{\epsilon\delta}$
        	\item $A$ uses $n$ random bits.
        	\item $\Pr[|A^{f}(n, \epsilon, \delta) - \overline{f}| \ge \epsilon] <
		        \delta$
        \end{itemize}
    \end{proposition}

    Define language $L_f$.

    $L_f = \bigcup\limits_{n \in \mathbb{N}}{r \in {0, 1}^n \mid 0.r < \overline{f}}$

    \begin{lemma}
        $L_f \in HeurP$
    \end{lemma}
\end{frame}

\begin{frame}
    \frametitle{Nondeterministic case}

    \begin{proposition}%[]
        There exists nondeterministic algorithm $A$ that:
        \begin{itemize}
	        \item $A$ is polynomially bounded in $t\frac{n}{\epsilon\delta}$
        	\item $A$ uses $n$ random bits.
        	\item $\Pr[|A^{f}(n, \epsilon, \delta) - \overline{f}| \ge \epsilon] <
		        \delta$
        \end{itemize}
    \end{proposition}

    $L_f = \bigcup\limits_{n \in \mathbb{N}}\{r \in {0, 1}^n \mid 0.r < \overline{f}\}$
	$Y = \{(C, x) \mid C$ is a circuit $x \in \{0, 1\}^n, x \in L_{C}\}$
    
    \begin{lemma}
        $Y \in HeurNP$
    \end{lemma}

    \begin{proposition}
        $Y \in SpanP-complete$
    \end{proposition}
    
    $\#P \in SpanP$, $Perm \in \#P$
\end{frame}

\begin{frame}
    \frametitle{Graph nonisomorphism}

    $GNI \in coNP$
    
    $GNI \in NP$ is an open question. We have no polynomially bounded proof systems
    for $GNI$ in classical case.

    Define new language.

    \begin{definition}
        $GNI_{pad} = \{(G_1, G_2, r) \mid (G_1, G_2) \in GNI,
        r \in \{0, 1\}^{q(n)}\}$, $q$ is a polynomial.
    \end{definition}

    \begin{lemma}
        If there exists polynomially bounded proof system for
        $GNI_{pad}$, then $GNI \in NP$.
    \end{lemma}
\end{frame}

\begin{frame}
    \frametitle{Результаты про неизоморфизм}

    \begin{theorem}
        Существует полиномиально ограниченная эвристическая система доказальств для
        $GNI_{pad}$.
    \end{theorem}

    \begin{proof}[sketch]
        $GNI \in AM$
    \end{proof}
    $GNI \in AM$
\end{frame}
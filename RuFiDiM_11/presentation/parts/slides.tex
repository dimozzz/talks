\begin{frame}
	\frametitle{Heuristic DPLL algorithms}

   	\input{pics/tree.tex}
    
	\pause
    \pause
    \pause
    \pause
    \pause
    \begin{itemize}
        \item Heuristic $\mathbf{A}$ chooses a variable for splitting.
    	\pause
	    \item Heuristic $\mathbf{B}$ chooses first value.
    \end{itemize}

    \pause
    It's allowed algogithm to have an error.

    \pause
    \pause
    \begin{itemize}
	    \item Heuristic $\mathbf{C}$ cuts branches.
    \end{itemize}
\end{frame}

\begin{frame}
    \frametitle{Lower bounds for DPLL algorithms without cut heuristic}

    \pause
	\begin{itemize}
		\item Unsatisfiable formulas
		\begin{itemize}
            \item{} Exponential lower bounds for resolution refutations
				of unsatisfiable formulas translate to backtracking
                algorithms.
			\item{} [Tseitin, 1968] ... [Pudlak, Implagliazzo, 2000].
		\end{itemize}
        \pause
		\item Satisfiable formulas
		\begin{itemize}
			\item If $\bf P = NP$ then there are no superpolynomial
		        lower bounds for backtracking algorithms since
                heuristic $\mathbf{B}$ may choose correct value.
            \pause
			\item Inverting of functions corresponds to satisfiable
		        formulas.
            \pause
            \item{} [Nikolenko~2002], [Achilioptas, Beame, Molloy~2003-2004]
				exponential lower bound for specific backtracking
                algorithms.
            \item{} [Alekhnovich, Hirsch, Itsykson~2005] Exponential lower bound 
				for myopic and drunken algorithms.
            \pause
            \item{}  Exponential lower bound 
		for inversion of Goldreich's function by myopic [J. Cook et al.~2009] 
		and drunken [Itsykson~2010] algorithms.
%            \item{}  Exponential lower bound 
%				for inversion of Goldreich's function by drunken algorithms.
		\end{itemize}
	\end{itemize}
\end{frame}


\begin{frame}
	\frametitle{Myopic heuristics}
    \pause

    \begin{definition}
        Myopic heuristic:
        \pause
        \begin{itemize}
	        \item sees structure of the formula;
        	\pause
        	\item doesn't see negation signs;
        	\item<6-> requests negations in $K = n^{1 - \epsilon}$ clauses.
        \end{itemize}
    \end{definition}

    \pause
    $\begin{array}{l}
        (x_1 \vee x_3 \vee x_5) \\
        \alert<7->{(x_2 \vee x_3)} \\
        (x_2 \vee x_4 \vee x_5) \\
        \alert<7->{(x_1 \vee x_4 \vee x_6)} \\
    \end{array}
    \pause
    \pause
    \pause
    \Rightarrow
    \begin{array}{l}
        (x_1 \vee x_3 \vee x_5) \\
        (x_2 \vee \alert{\neg} x_3) \\
        (x_2 \vee x_4 \vee x_5) \\
        (x_1 \vee \alert{\neg} x_4 \vee x_6) \\
    \end{array}$
    
\end{frame}

\begin{frame}
    \frametitle{Myopic algorithms}

    \pause

    \begin{definition}
		Myopic algorithm:
        \begin{itemize}
            \pause
	        \item $\mathbf{A}, \mathbf{C}$ are myopic, polynomial, determinate heuristic.
            \pause
        	\item $\mathbf{B}$ can be any.
        \end{itemize}
	\end{definition}

    \pause
    If $\mathbf{C} = 1$, then there exist some bounds: 

    \pause
    \begin{columns}
        \begin{column}{5cm}
            
            polynomial time (upper bound):
        \end{column}
        \begin{column}{5cm}
            
            exponential time (lower bound):
        \end{column}
    \end{columns}

    \pause
    \begin{columns}
        \begin{column}{5cm}
            \begin{itemize}
		 		\item $2$-CNF formulas;
			    \pause
    			\item horn formulas.
            \end{itemize}
        \end{column}
        \begin{column}{5cm}
            \begin{itemize}
	            \pause
    	        \item some classes of unsatifiable formulas;
        		\pause
            	\item formulas, which encode system of linear
		            equatitions.
            \end{itemize}
        \end{column}
    \end{columns}
\end{frame}

\begin{frame}
	\frametitle{Main theorem}
	\pause
	\begin{theorem}
        Для любой детерминированной полиномиальной эвристики $\mathbf{A}$
        существует семейство $(\Phi_n, \mathcal{D}_n)$ и $\delta \ne
        \delta(n)$ такие, что:
        \pause
		\begin{itemize}
            \item $\Phi_n$~--- невыполнимая формула, которую можно
		        построить полиномиальным алгоритмом.
            \pause
            \item $\mathcal{D}_n$~--- распределение на выполнимых
		        формулах, которое моделируется полиномиальным алгоритмом.
            \pause
			\item $\forall \mathbf{B}, \mathbf{C}$,
				$\Pr\limits_{x \gets \mathcal{D}_n}[\mathcal{A}(x)
                \ne 0] > 1 - \epsilon \Rightarrow
                t(\mathcal{A}(\Phi_n)) \ge 2^{\Omega(min(n^\delta,
                \frac{n}{K}))}$, где $\mathcal{A}$~--- близорукий
			    алгоритм $\mathcal{A}(\mathbf{A}, \mathbf{B},\mathbf{C})$.
		\end{itemize}
	\end{theorem}
\end{frame}

\begin{frame}
    \frametitle{Distribution for all algorithms}

    \begin{theorem}
        Существует семейство $(\Phi_n, \mathcal{D}_n)$ и $\delta \ne
        \delta(n)$ такие, что:
        \pause
        \begin{itemize}
	        \item $\Phi_n$~--- невыполнимая формула, которую можно
		        построить полиномиальным алгоритмом.
            \pause
            \item $\mathcal{D}_n$~--- распределение на выполнимых
		        формулах, которое моделируется полиномиальным алгоритмом.
            \pause
            \item $\forall \mathbf{A}, \mathbf{B}, \mathbf{C}$ 
		        если $\exists n_0, \epsilon$ такие, что:
				$\forall n > n_0, ~~ \Pr\limits_{x \gets \mathcal{D}_n}[\mathcal{A}(x)
                \ne 0] > 1 - \epsilon$, то
                $\forall n > n_0, ~~ t(\mathcal{A}(\Phi_n)) \ge 2^{\Omega(min(n^\delta,
                \frac{n}{K}))}$, где $\mathcal{A}$~--- близорукий
			    алгоритм $\mathcal{A}(\mathbf{A}, \mathbf{B},\mathbf{C})$.
        \end{itemize}
    \end{theorem}
\end{frame}

\begin{frame}
	\frametitle{Goldreich's function}
	$f:\{0, 1\}^n \rightarrow \{0, 1\}^n$

    \pause

    \begin{columns}
    	\begin{column}{5.5cm}
            \input{pics/function_graph.tex}
        \end{column}

        \pause
        \pause
        \begin{column}{5.5cm}
            \begin{itemize}
	            \item $f$ is a linear;
            	\pause
                \item $\forall y \in Y ~~ deg(y) \le d$
            	\pause
            	\item $d$ is a constant.
            \end{itemize}
        \end{column}
	\end{columns}
    
	\pause
    \begin{itemize}
	    \item $f_{G, P}(x) = b \Leftrightarrow \Phi(x)$;
    	\pause
	    \item $\Phi(x)$ contains no more than $2^dn$ clauses;
    	\pause
    	\item It's easy to create $\Phi(x)$.
    \end{itemize}
\end{frame}

\begin{frame}
    \frametitle{Dependencies graph}

    Bipartite grapth $(X, Y, E)$
    \pause
    \begin{itemize}
	    \item Full rank of adjacency matrix (over $\mathbb{F}_2$).
    	\pause
        \item $\forall z \in X \cup Y, ~~ deg(z) \le d$, ~~ $d$ is a constant.
    	\pause
        \item Expansion property.
    \end{itemize}

    \pause

    \begin{columns}
        \begin{column}{5cm}
            \input{pics/expander.tex}
        \end{column}

        \pause
        \pause
        \pause
        \begin{column}{5cm}
            \begin{itemize}
                \item $\forall y \in Y ~~ deg(y) \le d$
            	\pause
	            \item $\forall J \subset Y, ~
            		|J| < r \Rightarrow \Gamma(J) \ge \frac{3}{4}d|J|$
            \end{itemize}
		\end{column}
    \end{columns}

\end{frame}

\begin{frame}
	\frametitle{Idea of the proof}

    \pause
    Носитель распределения на выполнимых формулах состоит из
    модификаций невыполнимой формулы.
	\pause
    \begin{itemize}
        \item $\Phi_n$ кодирует задачу обращения функции Голдрейха.
    	\pause
    	\item $\Phi_n \mapsto \mathcal{D}_n, ~ |supp(\mathcal{D}_n)| =
		    2^{min(n^\delta, \Omega(\frac{n}{K}))}$ так, чтобы корректная
		    работа на конкретной выполнимой формуле означала
            ``увеличение времени работы на 1''.
	\end{itemize}

    \pause
    Построение единого распределения для всех близоруких DPLL
    алгоритмов:
    
    \pause
    \begin{itemize}
        \item комбинирование распределений из основной теоремы.
    \end{itemize}
\end{frame}

\begin{frame}
    \frametitle{Результаты}

    \pause
    \begin{enumerate}
	    \item Теорема о ``компромиссе''. Если близорукий алгоритм работает
		    корректно на построенном распределении, то он работает долго, на
		    построенной формуле.
        \pause
        \item Построение единого распределения для всех близоруких алгоритмов.
        \pause
        \item Конструкция экспандеров с полным рангом и ограниченной
		    степенью всех вершин.
    \end{enumerate}
\end{frame}
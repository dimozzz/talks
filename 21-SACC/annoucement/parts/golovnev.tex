\begin{center}
    \textsc{Data Structures Meet Circuits and Cryptography}

    \textsc{Speaker: Alexander Golovnev}
\end{center}



In this talk we will discuss the state of the art in the field of data structure lower bounds and
surprising connections between such lower bounds, circuit complexity, and cryptography. Proving such
limitations on data structures and circuits has been a fundamental research endeavor for several decades,
with connections to efficient parallel computation and the $\P$-vs-$\NP$ question. Despite much effort,
our best lower bounds in both fields have remained unchanged since the 1980s.

We show a surprising connection between both fields, offering an explanation for this lack of
progress. Specifically, we show that any improvement on the best known lower bound for a (linear) data
structure problem would imply new circuit lower bounds, and vice versa.

We continue this line of research by showing that data structure lower bounds for a specific class of
problems are equivalent to a certain kind of cryptography. We use this connection to construct surprising
(crypto-inspired) data structures for the $3\text{-}\langcplx{SUM}$ problem, refuting a data structure
variant of the $3\langcplx{SUM}$ conjecture due to Goldstein et al.
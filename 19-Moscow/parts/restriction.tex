\begin{frame}{Size measure. All operations modulo $\langle R_i \rangle$}
    \pause
    $\SOS: \sum\limits_{u = 1}^{a} p_u f_u + \sum\limits_{v = 1}^{b} q_v^2 h_v = -1$

    $$
    \text{Size} \coloneqq \sum_{u = 1}^{a} \left( \msize(p_u) + \msize(f_u) \right) +
    \sum_{v = 1}^{b} \msize(q_v) + \sum_{h \in \Hs} \msize(h)
    $$
        
    \vspace{1cm}
    $\PCR[\field]: (p_1, \dots, p_{\ell})$
    
    $$
    \text{Size} \coloneqq \sum_{u = 1}^{\ell} \left( \msize(p_u) \right)
    $$
\end{frame}


\begin{frame}{Strategy for the $\{0, 1\}$ basis ($\PCR[\field]$)}
    
    $\pi \coloneqq (p_1, \dots, p_{\ell})$ is a proof of $\Fs$. $H \coloneqq \{t \mid t \in p_i, \deg(t)
    \text{ is big}\}$.

    \begin{enumerate}
        \item $\pi$ is small $\Rightarrow$ size of $H$ is small.
        \pause
        \item Pick the most frequent literal $x$ in $H$.
        \pause
        \only<-9>{\item Set $x$ to $0$ in $\pi$. This operation kills all terms that contain $x$.}
        \only<10->{\item \textcolor{red}{Set $x$ to $0$ in $\pi$. This operation kills all terms that
            contain $x$.}}
        \pause
        \item $\pi \rest (x = 0)$ is still a proof of $\Fs \rest (x = 0)$.
        \pause
        \only<-5>{\item Keep $\Fs \rest (x = 0)$ hard in terms of degree. \begin{tikzpicture}
    \node[inner sep = 0pt, xscale = -1] at (0, 0) {\includegraphics[width = 0.08\textwidth]{pics/wand.png}};    
\end{tikzpicture}
}
        \only<6->{\item Try to avoid local contradictions in $\Fs \rest (x = 0)$. \begin{tikzpicture}
    \node[inner sep = 0pt, xscale = -1] at (0, 0) {\includegraphics[width = 0.08\textwidth]{pics/wand.png}};    
\end{tikzpicture}
}
        \pause
        \pause    
        \item Repeat until we have terms of big degree.
    \end{enumerate}

    \vspace{0.3cm}
    \pause
    We kill all terms of big degree but remaining system is still hard in terms of degree.

    \vspace{0.3cm}
    \pause
    \begin{center}
        \Huge Degree is the source of hardness.
    \end{center}
\end{frame}


\begin{frame}{Degree and the $\{\pm 1\}$ basis}
    \begin{center}
        \textcolor{red}{Set $x$ to $0$ in $\pi$. This operation kills all terms that contain $x$}
    \end{center}

    \pause
    \vspace{0.5cm}

    Attempts.
    \begin{enumerate}
        \item Set $x$ to $0$. \pause \hspace{1cm} $x^2 - 1 \rest (x = 0) \to -1$.
        \pause
        \item $\tau(p) \coloneqq \frac{p \rest (x = -1) + p \rest (x = 1)}{2}$. Consider $\tau(\pi)$.
    \end{enumerate}

    \pause

    \begin{minipage}{0.3\linewidth}
        \begin{prooftree}
            \AxiomC{$p$}
            \UnaryInfC{$xp$}
            \UnaryInfC{$p$}
        \end{prooftree}
    \end{minipage}
    \pause
    \begin{minipage}{0.3\linewidth}
        \begin{prooftree}
            \AxiomC{$\tau(p)$}
            \UnaryInfC{$\tau(xp)$}
            \UnaryInfC{$\tau(p)$}
        \end{prooftree}
    \end{minipage}
    \pause
    \begin{minipage}{0.3\linewidth}
        \begin{prooftree}
            \AxiomC{$p$}
            \UnaryInfC{$0$}
            \UnaryInfC{$p$}
        \end{prooftree}
    \end{minipage}

    \pause

    Multiplication is invertible.

    \pause


    \begin{block}{Grigoriev 98; Buss, Grigoriev, Impagliazzo, Pitassi 01; Grigoriev 01}
        \begin{enumerate}
            \item Tseitin formulas has small $\PCRf[\field]$ and $\SOSf$-proofs.
            \item There are Tseitin formulas that has $\PCR[\field]$ or $\SOS$-degree $\Omega(n)$.
        \end{enumerate}
    \end{block}
\end{frame}


\begin{frame}{Degree and the $\{\pm 1\}$ basis. Part 2}

    $\pi \coloneqq (p_1, \dots, p_{\ell})$.

    Can we reduce the degree of $p_i$?
    \pause

    \begin{enumerate}
        \item $p_i \coloneqq \prod\limits_{i = 1}^{n} x_i$ \pause \hspace{0.5cm} \textcolor{green}{YES}
            \pause
            
            \begin{prooftree}
                \AxiomC{$p$}
                \UnaryInfC{$x_1 p$}
                \UnaryInfC{$\vdots$}
                \UnaryInfC{$1$}
            \end{prooftree}
            \pause
        \item $p_i \coloneqq \prod\limits_{i = 1}^{n} x_i - 1$ \pause \hspace{0.5cm} \textcolor{red}{NOT REALLY}
    \end{enumerate}

    \vspace{0.5cm}
    \pause
    $p_i \coloneqq \sum\limits_j t_j$. Degree of the symmetric differences between $t_j$'s is the new
    source of hardness.    
\end{frame}


\begin{frame}{Quadratic representation and $\Split{x}$}

    $\pi \coloneqq (p_1, \dots, p_{\ell})$.  $p_i \coloneqq \sum\limits_j t_{i, j}$.

    \pause

    $p_i^2 \coloneqq \sum\limits_{j, j'} t_{i, j} t_{i, j'}$.
    
    We want to see all possible pairs, hence we prohibit cancellations.


    \begin{block}{Quadratic representation (QR)}
        The \deftext{QR} of $\pi$ is the sequence $(p_1^2, \dots, p_{\ell}^2)$ where squares are computed
        without cancellations.
    \end{block}

    \pause
    Reminder:
    $\tau(p) \coloneqq \frac{p \rest (x = -1) + p \rest (x = 1)}{2}$.

    \pause

    We want operation that apply $\tau$ to the QR of $\pi$.

    \pause
    \begin{block}{$\Split{x}$}
        $p_i \coloneqq r_i + x q_i$.

        $\Split{x}(\pi) \coloneqq (r_1, q_1, r_2, q_2, r_3, q_3, \dots, r_{\ell}, q_{\ell})$.
    \end{block}

    \pause
    $\Split{x}(\pi)$ is a proof of \deftext{damaged} version of $\Fs$.
    
\end{frame}


\begin{frame}{Strategy for the $\{\pm 1\}$ basis ($\PCR[\field]$)}
    
    $\pi \coloneqq (p_1, \dots, p_{\ell})$ is a proof of $\Fs$. $H \coloneqq \{t \mid t \in \text{ QR of }
    \pi, \deg(t) \text{ is big}\}$.

    \begin{enumerate}
        \item $\pi$ is small $\Rightarrow$ size of $H$ is small.
        \pause
        \item Pick the most frequent literal $x$ in $H$.
        \pause
        \item Apply $\Split{x}$ to $\pi$. This operation kills all terms that contain $x$ in the QR of $\pi$.
        \pause
        \item $\Split{x}(\pi)$ is still a proof of \deftext{damaged} $\Fs$.
        \pause
        \item Try to avoid local contradictions in $\Split{x}(\Fs)$. \begin{tikzpicture}
    \node[inner sep = 0pt, xscale = -1] at (0, 0) {\includegraphics[width = 0.08\textwidth]{pics/wand.png}};    
\end{tikzpicture}

        \pause
        \item Repeat until we have terms of big degree in the QR.
        \vspace{0.3cm}
        \pause
        \item Try to satisfy all \deftext{broken} constraints. \pause \textcolor{red}{Impossible for
            Tseitin formulas.}
    \end{enumerate}

    \pause
    \begin{lemma}
        Let $\pi$ be a $\PCRf[\field]$-proof of $\Fs$ and QR of $\pi$ has degree $d$. Then there is a
        $\PCRf[\field]$-proof $\pi'$ of $\Fs$ of degree $2d$.
    \end{lemma}
    \pause
    \textcolor{red}{This is wrong Lemma, we need to change definition of QR to fix it.}
\end{frame}

\begin{frame}{Lazy computations}
    
    $\pi \coloneqq (p_1, \dots, p_{\ell})$ is a proof of $\Fs$.

    \pause
    \deftext{Lazy representation} of $p_i$ ($\lazy{p_i})$ in the proof $\pi$:
    \begin{itemize}
        \item $\lazy{p}_i \coloneqq p_i$, if $p_i \in \Fs$ or $p_i \coloneqq p_j$ for some $j < i$;
        \item $\lazy{p}_i \coloneqq \alpha p_j + \beta p_k$ \textcolor{blue}{without cancellations}, if
            $p_i \coloneqq \alpha p_j + \beta p_k$.
    \end{itemize}

    \pause
    The fixed \deftext{QR} of $\pi$ is the sequence $(\lazy{p}_1^2, \dots, \lazy{p}_{\ell}^2)$ where
    squares are computed without cancellations.

    \pause
    \begin{lemma}
        Let $\pi$ be a $\PCRf[\field]$-proof of $\Fs$ and QR of $\pi$ has degree $d$. Then there is a
        $\PCRf[\field]$-proof $\pi'$ of $\Fs$ of degree $2d$.
    \end{lemma}

    \pause
    $p_i \coloneqq \sum\limits_{j} t_{i, j}$ and $s_i \coloneqq \sum\limits_{j} t_{i, 1} t_{i, j}$
    \hspace{0.4cm} $\Rightarrow$ \hspace{0.4cm} $p_i = t_{i, 1} s_i$ and $s_i = t_{i, 1} p_i$.

    \pause
    $\pi'' \coloneqq (s_1, \dots, s_{\ell})$
    \pause
    
    \begin{enumerate}
        \item $\mathbf{p_i \in \boldsymbol{\Fs}}$: \hspace{0.3cm} $s_i = t_{i, 1} p_i$.
        \item $\mathbf{p_i \coloneqq x p_j}$: \hspace{0.3cm} $s_i = s_j$.
        \item $\mathbf{p_i \coloneqq \boldsymbol{\alpha} p_a + \boldsymbol{\beta} p_b}$: \hspace{0.3cm}
            \pause $q \coloneqq \alpha \sum\limits_{j} t_{a, 1} t_{a, j} + \beta \sum\limits_{j} t_{a, 1}
            t_{b, j}$.

            \pause
            $q = \alpha s_{a} +  \beta \sum\limits_{j} t_{a, 1} t_{b, j} = \alpha s_a + \beta t_{a, 1}
            t_{b, 1} \sum\limits_{j} \beta t_{b, 1} t_{b, j} = \alpha s_a + \beta t_{a, 1} t_{b, 1} s_b$.

            \pause
            $s_i = \sum\limits_{j} t_{i, 1} t_{i, j}$. Wlog $t_{i, 1} \coloneqq t_{a, k}$ hence $s_i = t_{a, k} t_{a, 1} q$.
    \end{enumerate}


\end{frame}



\begin{frame}{Open problems}

    \begin{enumerate}
        \item Lower (or upper!) bound on $\PCRf[]$-proofs of Functional Pigeonhole Principle.
        \item Lower bound on $\PCRb[]$-proofs of Weak Pigeonhole Principle.

            \begin{tikzpicture}
    \node[inner sep = 0pt] at (0, 0) {\includegraphics[width = 0.2\textwidth]{pics/pigeon.png}};    
\end{tikzpicture}

        \pause
        \item Can we simulate Resolution in $\PCRf[\field]$? \pause Conjecture: NO.
    \end{enumerate}
\end{frame}
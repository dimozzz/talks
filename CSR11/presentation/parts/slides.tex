\begin{frame}
	\frametitle{Goldreich's function}
	$f:\{0, 1\}^n \rightarrow \{0, 1\}^n$

    \pause

    \begin{columns}
    	\begin{column}{5.5cm}
            \input{pics/function_graph.tex}
        \end{column}

        \pause
        \pause
        \begin{column}{5.5cm}
            \begin{itemize}
	            \item $G(X, Y, E)$~--- bipartite graph;
            	\pause
                \item $\forall y \in Y ~~ deg(y) \le d$
            	\pause
            	\item $d$ is a constant.
            \end{itemize}
        \end{column}
	\end{columns}
    
	\pause

    Goldreich's conjecture:
    \begin{itemize}
	    \item $P$ is a random predicate;
    	\item $G$ is an expander;
    \end{itemize}
    then function $f$ is a one-way.

    \pause
    \begin{itemize}
	    \item $f$ вычисляется схемами константной глубины;
    	\pause
	    \item{} [Crypto] Если есть односторонние функции, то есть и те,
		    которые вычисляются схемами константной глубины.
    \end{itemize}
\end{frame}

\begin{frame}
	\frametitle{DPLL algorithms}

   	\input{pics/tree.tex}
    
	\pause
    \pause
    \pause
    \pause
    \pause
    \begin{itemize}
        \item Heuristic $\mathbf{A}$ chooses variable.
    	\pause
	    \item Heuristic $\mathbf{B}$ chooses first value.
    	\pause
    	\item Simplification rules:
	    \begin{itemize}
            \item unit clause elimination;
        	\item pure literal rule.
    	\end{itemize}
    \end{itemize}

\end{frame}

\begin{frame}
    \frametitle{Lower bounds}

	\begin{itemize}
		\item Unsatisfiable formulas
		\begin{itemize}
			\item Lower bounds for resolution proof system.
			\item{} [Tseitin, 1968] ... [Pudlak, Implagliazzo, 2000].
			\item{} Exponential lower bounds for resolution refutations
				of unsatisfiable formulas translate to backtracking
                algorithms.
		\end{itemize}
        \pause
		\item Satisfiable formulas
		\begin{itemize}
			\item If $\bf P = NP$ then no superpolynomial lower bounds
		        for backtracking algorithms since heuristic $B$ may
                choose corect value.
            \pause
			\item Inverting of functions corresponds to satisfiable
		        formulas.
            \pause
            \item{} [Nikolenko, 2002], [Achilioptas,Beame, Molloy, 2003-2004]
				exponential lower bound for specific backtracking
                algoritms.
            \item{} [Alekhnovich, Hirsch, Itsykson, 2005] Exponential lower bound 
				for myopic and drunken algorithms.
		\end{itemize}
	\end{itemize}
\end{frame}

\begin{frame}
	\frametitle{Drunken algorithms}

    \begin{definition}
        Drunken DPLL algorithm:
        \begin{itemize}
	        \item $\mathbf{A}$~--- any.
        	\item $\mathbf{B}$~--- uniformly random.
        \end{itemize}
    \end{definition}

    \pause
    \begin{itemize}
    	\item{} [Alekhnovich, Hirsch, Itsykson~2005]
            \begin{itemize}
	            \item $P$~--- linear predicate.
            	\item $G$~--- random graph.
            \end{itemize}
        \pause
        \item{} [Itsykson~2010]
            \begin{itemize}
	            \item $P = x_1 + x_2 + \dots + x_{d - k} +
            		Q(x_{d - k + 1}, \dots, x_d)$.
            	\item $G$~--- random graph.
            \end{itemize}
    \end{itemize}
\end{frame}

\begin{frame}
	\frametitle{Myopic heuristic}
    \pause
    
    \begin{definition}
        Myopic heuristic:
        \pause
        \begin{itemize}
	        \item видит структуру формулы;
        	\pause
        	\item не видит знаков отрицаний;
        	\item<6-> на каждом шаге может запросить знаки отрицания в
        		$K = n^{o(n)}$ дизъюнктах.
        \end{itemize}
    \end{definition}

    \pause
    $\begin{array}{l}
        (x_1 \vee x_3 \vee x_5) \\
        \alert<7->{(x_2 \vee x_3)} \\
        (x_2 \vee x_4 \vee x_5) \\
        \alert<7->{(x_1 \vee x_4 \vee x_6)} \\
    \end{array}
    \pause
    \pause
    \pause
    \Rightarrow
    \begin{array}{l}
        (x_1 \vee x_3 \vee x_5) \\
        (x_2 \vee \alert{\neg} x_3) \\
        (x_2 \vee x_4 \vee x_5) \\
        (x_1 \vee \alert{\neg} x_4 \vee x_6) \\
    \end{array}$
    
\end{frame}

\begin{frame}
    \frametitle{Myopic algorithms}

    \pause

    \begin{definition}
		Myopic algorithm:
        \begin{itemize}
	        \item $\mathbf{A}, \mathbf{B}$~--- myopic heurictic.
        \end{itemize}
	\end{definition}

    \pause

    \begin{itemize}
    	\item{} [Alekhnovich, Hirsch, Itsykson~2005]
            \begin{itemize}
	            \item $P$~--- linear predicate.
            	\item $G$~--- random graph.
            \end{itemize}
        \pause
        \item{} [J. Cook, Etesami, Miller, Trevisan~2009] 
            \begin{itemize}
                \pause
   	            \item $P = x_1 + x_2 + \dots + x_{d - 2} + x_{d - 1}x_{d}$.
	            \item In fact: $P = x_1 + x_2 + \dots + x_{d - k} +
            		Q(x_{d - k + 1}, \dots, x_d)$.
                \pause
            	\item $G$~--- random graph.
            	\pause
            	\item $K$ is a constant.
            	\pause
            	\item Too complicated proof.
            \end{itemize}
    \end{itemize}
\end{frame}


\begin{frame}
    \frametitle{Function properties}

    \pause
    Goldreich's properties:
    \pause
    \begin{itemize}
	    \item ``almost'' bijection;
    	\pause
        \item graph $G$ is an expander.
    \end{itemize}

    \pause

    \begin{columns}
        \begin{column}{5cm}
            \input{pics/expander.tex}
        \end{column}

        \pause
        \pause
        \pause
        \begin{column}{5cm}
            \begin{itemize}
                \item $\forall y \in Y ~~ deg(y) = d$
            	\pause
	            \item $\forall J \subset Y, ~
            		|J| < r \Rightarrow \Gamma(J) \ge \frac{2}{3}d|J|$
            \end{itemize}
		\end{column}
    \end{columns}

    \pause
    In fact we are unable to check this properties in polynomial time.

\end{frame}

\begin{frame}
	\frametitle{Main theorem}

	$P(x_1, \ldots, x_d) = x_1 \oplus x_2 \oplus \ldots \oplus x_{d - k} \oplus 
	Q(x_{d - k + 1}, \ldots, x_d)$, $Q$ --- arbitrary.

	\pause
	\begin{theorem}
		Существует явная конструкция графа $G$ такая, что:
		\begin{itemize}
			\item $\forall$ ``близорукого'' алгоритма
        	$A,~ Pr_{x, r}[t(A_m(G, P, b)) \ge 2^{\Omega(n)}] \ge 1 - 2^{-\Omega(n)}$
		\end{itemize}
	\end{theorem}

\end{frame}

\begin{frame}
	\frametitle{План доказательства}

	\begin{itemize}
		\pause
		\item Время работы DPLL алгоритмов на невыполнимой формуле не менее $2^{\Omega(n)}$
		\pause
		\item With probability $2^{-\Omega(n)}$ after several subtitutions 
    	После нескольких шагов алгоритма вероятность того, что частичная подстановка согласована
			с конкретным прообразом не более $2^{-\Omega(n)}$
		\pause
		\item $G$ такой, что число прообразов не более $2^{o(n)}$
	\end{itemize}
   
\end{frame}
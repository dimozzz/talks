\begin{frame}
    \frametitle{Задача поиска выполняющего набора}

    \pause
    \begin{task}
        По булевой формуле $\varphi(x_1, x_2 \dots, x_n)$ найти
        набор $c_1, c_2, \dots, c_n$ такой, что $\varphi(c_1, c_2
        \dots, c_n) = 1$.
    \end{task}

    \pause
    \begin{itemize}
        \item Задача поиска выполняющего набора $NP$-полная.
		\pause
    	\item Множество промышленных задач сводится к задачи поиска
		    выполняющего набора.
        \pause
        \item SAT competition.
    \end{itemize}
\end{frame}

\begin{frame}
    \frametitle{Классы формул}

    \pause
    Для многие классов булевых формул известны эффективные алгоритмы
    для нахождения выполняющего набора:
    \begin{itemize}
        \pause
 		\item $2$-КНФ формулы;
    	\pause
    	\item хорновские формулы;
	    \pause
    	\item цейтинские формулы;
    	\item ...
    \end{itemize}

    \pause
   	Будем рассматривать $d$-КНФ формулы~--- в каждом дизъюнкте не
    более $d$ переменных.
    
    \pause
    $(x_1 \vee x_3 \vee \neg x_5) \wedge (x_2 \vee \neg x_3) \wedge
    (\neg x_2 \vee x_4 \vee x_5) \wedge (x_1 \vee \neg x_4 \vee x_6)$
\end{frame}

\begin{frame}
	\frametitle{Эвристические DPLL алгоритмы}

   	\input{pics/tree.tex}
    
	\pause
    \pause
    \pause
    \pause
    \pause
    \begin{itemize}
        \item Эвристика $\mathbf{A}$ выбирает переменную.
    	\pause
	    \item Эвристика $\mathbf{B}$ выбирает первое значение.
    \end{itemize}

    \pause
    Разрешим алгоритму ошибаться.

    \pause
    \pause
    \begin{itemize}
	    \item Эвристика $\mathbf{C}$ обрезает ветви.
    \end{itemize}
\end{frame}

\begin{frame}
	\frametitle{Близорукие эвристики}
    \pause
    
    \begin{definition}
        Близорукая эвристика:
        \pause
        \begin{itemize}
	        \item видит структуру формулы;
        	\pause
        	\item не видит знаков отрицаний;
        	\item<6-> на каждом шаге может запросить знаки отрицания в
        		$K = n^{o(n)}$ дизъюнктах.
        \end{itemize}
    \end{definition}

    \pause
    $\begin{array}{l}
        (x_1 \vee x_3 \vee x_5) \\
        \alert<7->{(x_2 \vee x_3)} \\
        (x_2 \vee x_4 \vee x_5) \\
        \alert<7->{(x_1 \vee x_4 \vee x_6)} \\
    \end{array}
    \pause
    \pause
    \pause
    \Rightarrow
    \begin{array}{l}
        (x_1 \vee x_3 \vee x_5) \\
        (x_2 \vee \alert{\neg} x_3) \\
        (x_2 \vee x_4 \vee x_5) \\
        (x_1 \vee \alert{\neg} x_4 \vee x_6) \\
    \end{array}$
    
\end{frame}

\begin{frame}
    \frametitle{Близорукие алгоритмы}

    \pause

    \begin{definition}
		Близорукие алгоритмы:
        \begin{itemize}
            \pause
	        \item $\mathbf{A}$~--- детерминированная, полиномиальная.
            \pause
        	\item $\mathbf{B}$~--- произвольная.
        	\pause
        	\item $\mathbf{C}$~--- близорукая.
        \end{itemize}
	\end{definition}

    \pause
    Если $\mathbf{C} = 1$, то известны некоторые оценки: 

    \pause
    \begin{columns}
        \begin{column}{5cm}
            
            полиномиальное время (верхняя оценка):
        \end{column}
        \begin{column}{5cm}
            
            экспоненциальное время (нижняя оценка):
        \end{column}
    \end{columns}

    \pause
    \begin{columns}
        \begin{column}{5cm}
            \begin{itemize}
		 		\item $2$-КНФ формулы;
			    \pause
    			\item цейтинские формулы (у которых есть выполняющий
			        набор).
            \end{itemize}
        \end{column}
        \begin{column}{5cm}
            \begin{itemize}
	            \pause
    	        \item различные классы невыполнимых формул;
        		\pause
            	\item формулы, кодирующие линейные системы уравнений.
            \end{itemize}
        \end{column}
    \end{columns}
\end{frame}


\begin{frame}
	\frametitle{Функция Голдрейха}
	$f:\{0, 1\}^n \rightarrow \{0, 1\}^n$

    \pause

    \begin{columns}
    	\begin{column}{5.5cm}
            \input{pics/function_graph.tex}
        \end{column}

        \pause
        \pause
        \begin{column}{5.5cm}
            \begin{itemize}
	            \item $f$~--- линейная;
            	\pause
                \item $\forall y \in Y ~~ deg(y) \le d$
            	\pause
            	\item $d$~--- константа.
            \end{itemize}
        \end{column}
	\end{columns}
    
	\pause
	\begin{remark}
	    По уравнению $f_{G, P}(x) = b$ можно построить булеву формулу
		в КНФ, эквивалентную данной системе и формула будет содержать
        не более $2^dn$ дизъюнктов.
	\end{remark}
\end{frame}

\begin{frame}
    \frametitle{Граф зависимостей}

    Двудольный граф $(X, Y, E)$
    \pause
    \begin{itemize}
	    \item Полный ранг матрицы смежности.
    	\pause
        \item Ограниченная степень всех вершин.
    	\pause
        \item Свойства экспандера.
    \end{itemize}

    \pause

    \begin{columns}
        \begin{column}{5cm}
            \input{pics/expander.tex}
        \end{column}

        \pause
        \pause
        \pause
        \begin{column}{5cm}
            \begin{itemize}
                \item $\forall y \in Y ~~ deg(y) \le d$
            	\pause
	            \item $\forall J \subset Y, ~
            		|J| < r \Rightarrow \Gamma(J) \ge \frac{3}{4}d|J|$
            \end{itemize}
		\end{column}
    \end{columns}

\end{frame}


\begin{frame}
    \frametitle{Plan}

    \begin{enumerate}
	    \item Proof systems. Classical and heuristic case.
    	\item Auxiliary problem. Calculation the average value of function.
    	\item Graph nonisomorphism (GNI) problem and its modification.
    	\item Application methods to GNI.
    \end{enumerate}
\end{frame}

\begin{frame}
	\frametitle{Proof systems}

    \begin{definition}
        Proof system for language $L$:
        \begin{itemize}
            \item Polynomial time algorithm $\Pi$;
            \item $\forall x \in L~ \exists \omega \in \{0, 1\}^{*}, \Pi(x, \omega) = 1$;
        	\item $\forall x \notin L~ \forall \omega \in \{0, 1\}^{*}, 
        		\Pi(x, \omega) = 0$.
        \end{itemize}
    \end{definition}

    \pause
    Proof system is polynomially bounded is there exists such polynomial $p$ that:
    $\forall x \in L~ \exists \omega \in \{0, 1\}^{p(|x|)}, \Pi(x, \omega) = 1$.


    \pause
    If there exists polynomially bounded:
    \begin{itemize}
        \item algorithm for $L$ then $L \in P$
	    \item proof system for $L$ then $L \in NP$
    \end{itemize}

\end{frame}

\begin{frame}
    \frametitle{Heuristic classes}

    $L$ is a language, $D = {D_n}$ is a family of distributions.

    Allow algorithm to have a mistakes.

    \pause
    \begin{definition}
        $L \in HeurP$:
        \begin{itemize}
            \item Algorithm $\Pi(x, 1^n, \delta)$ is polynomially bounded in $\frac{n}{\delta}$;
            \item $\Pr\limits_{x \gets D_n}[\Pi(x, 1^n, \delta)] \neq
        		L(x)] < \delta$.
        \end{itemize}
    \end{definition}

    Define $HeurNP$ by analogy.

    \pause

    \vspace{0.5cm}
    $(P, U) \subset HeurP \subseteq (EXP, U)$
    
    $(NP, U) \subset HeurNP \subseteq (NEXP, U)$
\end{frame}

\begin{frame}
    \frametitle{Heuristic proof system}

    $L$ is a language, $D = {D_n}$ is a family of distributions on
    complement $L$.

    \pause
    \begin{definition}
        $\Pi$ is a heuristic proof system for $L$:
        \begin{itemize}
            \item $\Pi$ is polynomially bounded by $\frac{n}{\delta}$;
            \item $x \in L \Rightarrow \exists \omega~\Pi(x, \omega,
        		1^n, \delta) = 1$;
            \item $x \notin L \Rightarrow
        		\Pr\limits_{x \gets D_n}[\exists \omega~
                \Pi(x, \omega, 1^n, \delta) = 1] < \delta$.
        \end{itemize}
    \end{definition}
    
\end{frame}

\begin{frame}
    \frametitle{Motivation}

    \pause
    \begin{itemize}
	    \item Structural properties (optimal randomized acceptor,
		    hierarchy theorem, complete language...).
        \pause
    	\item Practice.
    	\pause
        \item New methods and bounds for classical case.
    \end{itemize}

    \begin{statement}
        There is no {\it natural} example with gap between classical and heuristic proof system.
    \end{statement}
\end{frame}

\begin{frame}
    \frametitle{Results}

    $L$ is a language.
    $L_{pad} = \{(x, r) \mid x \in L, |r| \in \{0, 1 \}^{p(|x|)}\}$.
    
    \pause
    \begin{enumerate}
		\item $\exists L \in PP$-complete, $\forall D~
    		(L_{pad}, D \times U) \in HeurP$.

            \pause
            If $P \neq NP$ then $L, L_{pad} \notin P$.
    	\pause
    	\item $\exists L \in (PP \cdot NP)$-complete, $\forall D~
		    (L_{pad}, D \times U) \in HeurNP$.

    	\pause
    	\item $\forall L \in AM \Rightarrow \forall D,
    		(L_{pad}, D \times U) \in HeurNP$.

            \pause
            
            \vspace{0.5cm}
            Generalization:
		    $\forall L \in BP \cdot \mathfrak{C}P \Rightarrow L_{pad}
            \in Heur\mathfrak{C}P$ ($\mathfrak{C}$ is a computation
            model).    
    \end{enumerate}

    \pause
    \begin{example}
        $GNI \in AM \Rightarrow (GNI_{pad}, U) \in HeurNP$
    \end{example}
\end{frame}

\begin{frame}
    \frametitle{AM}
    
    \begin{definition}
    	$L \in AM$ iff there exists polynomial randomized algorithm
        $M$:
        \begin{itemize}
			\item $x \in L \Leftrightarrow
        		\Pr\limits_{r}[\exists z~ M(x, z) = 1] 
        		> \frac{2}{3}$;
            \item $x \notin L \Leftrightarrow
        		\Pr\limits_{r}[\exists z~ M(x, z) = 1] 
        		< \frac{1}{3}$;
        \end{itemize}
    \end{definition}

    \pause
    \begin{theorem}
        $L \in AM$ iff there exists polynomial $p$, constant $c$ and
        polynomial randomized algorithm $M$ with $c p(|x|)$ random
        bits:
        \begin{itemize}
			\item $x \in L \Leftrightarrow
        		\Pr\limits_{r}[\exists z~ M(x, z) = 1] = 1$;
            \item $x \notin L \Leftrightarrow
        		\Pr\limits_{r}[\exists z~ M(x, z) = 1] 
        		< \frac{1}{2^{p(|x|)}}$;
        \end{itemize}
    \end{theorem}
\end{frame}

\begin{frame}
    \frametitle{Graph nonisomorphism}

    $GNI \in coNP$
    
    $GNI \in NP$ is an open question. We have no polynomially bounded proof systems
    for $GNI$ in classical case.

	\pause
    \begin{definition}
        $GNI_{pad} = \{(G_1, G_2, r) \mid (G_1, G_2) \in GNI,
        r \in \{0, 1\}^{q(n)}\}$, $q$ is a polynomial.
    \end{definition}

    \pause
    \begin{lemma}
        If there exists polynomially bounded proof system for
        $GNI_{pad}$, then $GNI \in NP$.
    \end{lemma}
\end{frame}

\begin{frame}
    \frametitle{Results for nonisomorphism}

    \begin{theorem}
        There exists polynomially bounded proof system for $GNI_{pad}$.
    \end{theorem}

    \begin{proof}
        \begin{enumerate}
	        \item If $G_1 \sim G_2$ then there exists $\frac{n!}{Aut(G_1)}$ different
		        graphs that can be received by permutation.
            \item If $G_1 \nsim G_2$ then there exists $\frac{n!}{Aut(G_1)} +
		        \frac{n!}{Aut(G_2)}$ different graphs that can be received by permutation.
            \item Increase gap between cases.
        	\item Compute average value of different graphs (use padding as random bits).
        \end{enumerate}
    \end{proof}
\end{frame}

\begin{frame}
    \frametitle{Auxiliary problem. Computation the average value}

    $f:\{0, 1\}^{n} \rightarrow \{0, 1\}$
    
    $\overline{f} = \frac{\sum\limits_{i = 0}^{2^n - 1}f(i)}{2^n}$

    $L_f = \bigcup\limits_{n \in \mathbb{N}}\{r \in \{0, 1\}^n \mid 0.r < \overline{f}\}$

    \begin{lemma}
        If $f$ can be computed in polynomial time than $L_f \in HeurP$.
    \end{lemma}

    \begin{statement}
        There exists algorithm $A$ that:
        \begin{itemize}
	        \item $A$ is polynomially bounded in $\frac{n}{\epsilon\delta}$
        	\item $A$ uses $n$ random bits.
        	\item $\Pr[|A^{f}(n, \epsilon, \delta) - \overline{f}| \ge \epsilon] <
		        \delta$
        \end{itemize}
    \end{statement}
\end{frame}


\begin{frame}
    \frametitle{Algorithm for auxiliary problem}

    \begin{statement}
        There exists algorithm $A$ that:
        \begin{itemize}
	        \item $A$ is polynomially bounded in $\frac{n}{\epsilon\delta}$
        	\item $A$ uses $n$ random bits.
        	\item $\Pr[|A^{f}(n, \epsilon, \delta) - \overline{f}| \ge \epsilon] <
		        \delta$
        \end{itemize}
    \end{statement}

    \begin{enumerate}
 		\item Create an expander ({\it ``random''}) graph on $2^n$ vertexes.
    	\item Pick a random vertex $v$ (we use $n$ random bits).
    	\item Average out value of function in points on distance $t$ from $v$.
    \end{enumerate}
    
\end{frame}

\begin{frame}
    \frametitle{Nondeterministic case}

    $L_f = \bigcup\limits_{n \in \mathbb{N}}\{r \in \{0, 1\}^n \mid 0.r <
    \overline{f}\}$

    \begin{lemma}
        If $f$ can be computed in polynomial time by nondeterministic algorithm than
        $L_f \in HeurNP$.
    \end{lemma}
    
	$Y = \{(C, x) \mid C$ is a circuit $x \in \{0, 1\}^n, x \in L_{C}\}$
    
    \begin{lemma}
        $Y \in HeurNP$, $Y \in \#P-hard$
    \end{lemma}

    \begin{itemize}
	    \item $Perm \in \#P$
    \end{itemize}
    
\end{frame}
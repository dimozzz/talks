\begin{frame}
    \frametitle{Plan}

    \begin{enumerate}
	    \item Proof systems. Classical and heuristic case.
    	\item \textit{Hard} languages in $HeurP$ and $HeurNP$.
    	\item Graph nonisomorphism (GNI) problem and its modification. Short proofs.
    \end{enumerate}
\end{frame}


\begin{frame}
    \frametitle{Heuristic classes}

    $L$ is a language, $D = {D_n}$ is a family of distributions.

    Allow algorithm to have a mistakes.

    \pause
    \begin{definition}
        $L \in Heur\only<3->{\alert{N}}P$:
        \begin{itemize}
            \item \only<3->{\alert{Non}deterministic} \only<-2>{Deterministic}
		        algorithm $\Pi(x, 1^n, \delta)$ is polynomially bounded in
                $\frac{n}{\delta}$;
            \item $\Pr\limits_{x \gets D_n}[\Pi(x, 1^n, \delta)] \neq
        		L(x)] < \delta$.
        \end{itemize}
    \end{definition}

    \pause

%    \vspace{0.5cm}
%    $\forall D~ (P, D) \subset HeurP \subseteq (EXP, U)$
    
%    $\forall D~ (NP, D) \subset HeurNP \subseteq (NEXP, U)$
\end{frame}


\begin{frame}
	\frametitle{Proof systems}

    \begin{definition}
        Proof system for language $L$:
        \begin{itemize}
            \item Polynomial time algorithm $\Pi$;
            \item $\forall x \in L~ \exists \omega \in \{0, 1\}^{*}, \Pi(x, \omega) = 1$;
        	\item $\forall x \notin L~ \forall \omega \in \{0, 1\}^{*}, 
        		\Pi(x, \omega) = 0$.
        \end{itemize}
    \end{definition}

    \pause
    Proof system is polynomially bounded is there exists such polynomial $p$ that:
    $\forall x \in L~ \exists \omega \in \{0, 1\}^{p(|x|)}, \Pi(x, \omega) = 1$.

\end{frame}


\begin{frame}
    \frametitle{Heuristic proof system}

    $L$ is a language, $D = {D_n}$ is a family of distributions on
    complement $L$.

    \pause
    \begin{definition}
        $\Pi$ is a heuristic proof system for $(L, D)$:
        \begin{itemize}
            \item $x \in L \Rightarrow \exists \omega~\Pi(x, \omega,
        		1^n, \delta) = 1$;
            \item $x \notin L \Rightarrow
        		\Pr\limits_{x \gets D_n}[\exists \omega~
                \Pi(x, \omega, 1^n, \delta) = 1] < \delta$;
             \item $\Pi$ is polynomially bounded by $\frac{n}{\delta}$.
        \end{itemize}
    \end{definition}

    $L \in HeurNP \Rightarrow L$ has a heuristic proof system which polynomially
    bounded on \textit{almost all} elements.
\end{frame}


\begin{frame}
    \frametitle{\textit{Hard} languages}

    $L$ is a language.
    $L_{pad} = \{(x, r) \mid x \in L, |r| \in \{0, 1 \}^{p(|x|)}\}$.

    \begin{theorem}
        \begin{enumerate}
            \item $\exists X \in PP$-complete, $\forall D~
	    		(X_{pad}, D \times U) \in HeurP$.
    		\pause
    		\item $\exists Y \in (PP \cdot NP)$-complete:
		        \begin{itemize}
        	        \item $\forall D~ (Y_{pad}, D \times U) \in HeurNP$;
                	\item $(NP, PSamp) \notin HeurBPP \Rightarrow \exists D'~ (Y, D')
                		\notin HeurBPP$.
		        \end{itemize}
        		
        \end{enumerate}
    \end{theorem}

    $L \in PP \cdot NP$ iff there exists a $PP$-reduction $L$ to some $NP$ language.
    
    \pause

    \vspace{0.5cm}
    $PP$ is closed under complement thus if $NP \neq co-NP$ then:
    \begin{itemize}
    	\item $X, X_{pad} \notin NP$;
    	\item $Y, Y_{pad} \notin NP$.
    \end{itemize}
 
\end{frame}



\begin{frame}
    \frametitle{Auxiliary problem. Computation the average value}

    $f:\{0, 1\}^{n} \rightarrow \{0, 1\}$
    
    $\overline{f} = \frac{\sum\limits_{i = 0}^{2^n - 1}f(i)}{2^n}$

    \pause
    
    \begin{statement}[Goldreich]
        There exists algorithm $A$ that:
        \begin{itemize}
	        \item $A$ is polynomially bounded in $\frac{n}{\epsilon\delta}$
        	\item {\color{blue}$A$ uses $n$ random bits}.
        	\item $\Pr[|A^{f}(n, \epsilon, \delta) - \overline{f}| \ge \epsilon] <
		        \delta$
        \end{itemize}
    \end{statement}

    \pause
    $L_f = \bigcup\limits_{n \in \mathbb{N}}\{r \in \{0, 1\}^n \mid 0.r < \overline{f}\}$

    \begin{lemma}
        If $f$ can be computed in (nondeterministic)polynomial time then $L_f \in Heur(N)P$.
    \end{lemma}

\end{frame}


\begin{frame}
    \frametitle{\textit{Hard} languages. Part 2}
    \begin{theorem}
        \begin{enumerate}
            \item $\exists X \in PP$-complete, $\forall D~
	    		(X_{pad}, D \times U) \in HeurP$.
    		\pause
    		\item $\exists Y \in (PP \cdot NP)$-complete:
		        \begin{itemize}
        	        \item $\forall D~ (Y_{pad}, D \times U) \in HeurNP$;
                	\item $(NP, PSamp) \notin HeurBPP \Rightarrow \exists D'~ (Y, D')
                		\notin HeurBPP$.
		        \end{itemize}
        		
        \end{enumerate}
    \end{theorem}

    \pause
    $X = \{(C, r) \mid C$ is a deterministic circuit $r \in \{0, 1\}^n, \#C > r\}$
    
    $Y = \{(C, r) \mid C$ is a nondeterministic circuit $r \in \{0, 1\}^n, \#C >
    r\}$.

    \pause

    \vspace{0.2cm}
    For any fixed circuit we can use algorithm for auxilary problem.
\end{frame}



\begin{frame}
	\frametitle{Weakness}

    \begin{itemize}
    	\item $X$ has a heuristic algorithm (we don't need any proofs);
    	\item polynomial proofs not for all words of $Y$;
    	\item hard distribution not on complement of $Y$;
    	\item $X$ and $Y$ are complete problems under {\color{blue}worst-case}
		    reduction.
    \end{itemize}
\end{frame}


\begin{frame}
    \frametitle{AM}
    
    \begin{theorem}
		$\forall L \in AM$ there exists a polynomially bounded heuristic proof system
        for $L_{pad}$.
    \end{theorem}


    \begin{definition}
    	$L \in AM$ iff there exists polynomial randomized algorithm
        $M$:
        \begin{itemize}
			\item $x \in L \Leftrightarrow
        		\Pr\limits_{r}[\exists z~ M(x, z) = 1] 
        		> \frac{2}{3}$;
            \item $x \notin L \Leftrightarrow
        		\Pr\limits_{r}[\exists z~ M(x, z) = 1] 
        		< \frac{1}{3}$;
        \end{itemize}
    \end{definition}
    
\end{frame}

\begin{frame}
    \frametitle{AM. Part 2}
    
    \begin{definition}
    	$L \in AM$ iff there exists polynomial randomized algorithm
        $M$:
        \begin{itemize}
			\item $x \in L \Leftrightarrow
        		\Pr\limits_{r}[\exists z~ M(x, z) = 1] 
        		> \frac{2}{3}$;
            \item $x \notin L \Leftrightarrow
        		\Pr\limits_{r}[\exists z~ M(x, z) = 1] 
        		< \frac{1}{3}$;
        \end{itemize}
    \end{definition}

    \pause
    Amplification theorem:
    \begin{theorem}
        $L \in AM$ iff there exists polynomial \alert{$p$}, constant $c$ and
        polynomial randomized algorithm $M'$ with \alert{$c p(|x|)$} random
        bits:
        \begin{itemize}
			\item $x \in L \Leftrightarrow
        		\Pr\limits_{r}[\exists z~ M'(x, z) = 1] = 1$;
            \item $x \notin L \Leftrightarrow
        		\Pr\limits_{r}[\exists z~ M'(x, z) = 1] 
        		< \frac{1}{2^{\alert{p(|x|)}}}$;
        \end{itemize}
    \end{theorem}

    \pause
    $\Pr\limits_{r}[\exists z M(x, z) = 1]$ is an average value of function that can
    be computed by nondeterministic algorithm.
\end{frame}

\begin{frame}
    \frametitle{Graph nonisomorphism}

    \begin{theorem}[Goldwasser-Sipser, 89]
    	$GNI \in AM$    
    \end{theorem}
    
	\pause
    \begin{definition}
        $GNI_{pad} = \{(G_1, G_2, r) \mid (G_1, G_2) \in GNI,
        r \in \{0, 1\}^{q(n)}\}$, $q$ is a fixed polynomial.
    \end{definition}

    \pause
    \begin{lemma}
        If there exists polynomially bounded proof system for
        $GNI_{pad}$, then $GNI \in NP$.
    \end{lemma}

    \begin{proof}
        \begin{enumerate}
        	\item $\delta > frac{1}{2^{q(n)}}$. $Pi[(G_1, G_2, r), 1^n, w, \delta] =
        		M'_r(x, w)$. Where $M'_r$ is a machine from amplification theorem
                which uses $r$ as random bits.
            \item $\delta < frac{1}{2^{q(n)}}$. Brute force (proof is any string).
        \end{enumerate}
    \end{proof}
\end{frame}
\begin{frame}{Notation}
    $(\Fs, \Hs) =
    \begin{cases}
        f_1(x_1, \dots, x_n) = 0 \\
        f_2(x_1, \dots, x_n) = 0 \\
        \dots \\
        f_a(x_1, \dots, x_n) = 0 \\
        \hline
        h_1(x_1, \dots, x_n) \ge 0 \\
        h_2(x_1, \dots, x_n) \ge 0 \\
        \dots \\
        h_s(x_1, \dots, x_n) \ge 0 \\
    \end{cases}
    $

    \vspace{0.1cm}
    $f_i, h_j$ are polynomials.

    \vspace{0.1cm}
    \pause

    \begin{block}{Range axiom $R_i$ for a variable $x_i$:}
        \begin{itemize}
            \item $\mathbf{\{0, 1\}}$ basis: $x_i^2 - x_i$;
            \item $\mathbf{\{\pm 1\}}$ basis: $x_i^2 - 1$.
        \end{itemize}
    \end{block}

   
\end{frame}


\begin{frame}{Proof systems}

    The \deftext{Sum-of-Squares} ($\SOS$) proof of $(\Fs, \Hs)$:
    $$
        \sum_{u = 1}^{a} p_u f_u + \sum_{j = 1}^{n} r_j R_j + \sum_{v = 1}^{b} q_v^2 h_v = -1
    $$
    $f_u \in \Fs, h_v \in \Hs \cup {1}$

    \pause
    \vspace{0.4cm}

    The \deftext{Polynomial Calculus} ($\PCR[\field]$) proof of $\Fs$ is a sequence
    $(p_1, p_2, p_3, \dots, p_{\ell})$:
    \pause
    \begin{itemize}
        \item $p_i \in \Fs \cup \bigcup\limits_{j = 1}^{n} \{R_j\}$;
        \pause
        \item $p_i = x_j p_k$ for some $j$ and $k < i$;
        \pause    
        \item $p_i = \alpha p_k + \beta p_s$ for some $k, s < i$ and $\alpha, \beta \in \field$;
        \pause
            \item $p_{\ell} = 1$.
    \end{itemize}

    \pause
    \vspace{-0.2cm}
    \begin{minipage}{0.4\linewidth}
        $\Fs =
        \begin{cases}
            xy - 1 = 0 \\
            yz + 1 = 0 \\
            x + z - 2 = 0
        \end{cases}$
    \end{minipage}
    \pause
    \begin{minipage}{0.58\linewidth}
        \begin{prooftree}
            \AxiomC{$x + z - 2$}
            \UnaryInfC{$x y + y z - 2y$}
            \AxiomC{$xy - 1$}
            \AxiomC{$yz + 1$}
            \BinaryInfC{$xy + yz$}
            \BinaryInfC{$2y$}
            \UnaryInfC{$2 y^2$}
            \AxiomC{$y^2 - 1$}
            \BinaryInfC{$1$}
        \end{prooftree}
    \end{minipage}
\end{frame}


\begin{frame}{Motivation}

    \pause
    \begin{itemize}
        \item Lower bounds on the length of proofs $\Rightarrow$ lower bounds on the running time of
            algorithms.
            \pause
        \item Cook's program: $\NP \stackrel{?}{=} \coNP$.
            \pause
        \item Lower bounds on the length of proofs $\Rightarrow$ lower bounds on some computational
            models.
            \pause
        \item There are unconditional lower bounds!
    \end{itemize}

    \pause
    \vspace{0.3cm}
    Why do we pay attention to different encodings: $x \in \{0, 1\}$ and $x \in \{\pm 1\}$?
    \pause
    \begin{itemize}
        \item Sizes may be really different in these encodings.
            \pause
        \item It shows the limitations of current techniques for proving lower bounds.
    \end{itemize}
    
\end{frame}

\begin{frame}{It really helps!}

    \begin{block}{Grigoriev 98; Buss, Grigoriev, Impagliazzo, Pitassi 01; Grigoriev 01}
        \begin{enumerate}
            \item Tseitin formulas has small $\PCRf[\field]$ and $\SOSf$-proofs.
            \item There are Tseitin formulas that has $\PCR[\field]$ or $\SOS$-degree $\Omega(n)$.
        \end{enumerate}
    \end{block}

    \begin{minipage}{0.5 \linewidth}
        \tikzstyle{undir} = [thick]
\tikzstyle{dir} = [thick, ->, bend left = 12]
\tikzstyle{ver} = [thick, ->, draw, circle]

\begin{tikzpicture}[black, >=stealth']

    \node[ver] (A) at (0, 0) {};
    \node[ver] (B) at (2, 2) {};
    \node[ver] (C) at (3, 0) {};
    \node[ver] (D) at (4, 1) {};
    \node[ver] (E) at (3.5, 3.5) {};
    \node[ver] (F) at (-0.5, 3) {};

    \only<1>{
        \draw[undir] (A) to (B);
        \draw[undir] (A) to (C);
        \draw[undir] (B) to (C);
        \draw[undir] (C) to (D);
        \draw[undir] (B) to (D);
        \draw[undir] (D) to (E);
        \draw[undir] (E) to (F);
        \draw[undir] (F) to (A);
        \draw[undir] (B) to (F);
    }
    
	\only<2->{
        \draw[dir] (A) to (B);
        \draw[dir] (A) to (C);
        \draw[dir] (B) to (C);
        \draw[dir] (C) to (D);
        \draw[dir] (B) to (D);
        \draw[dir] (D) to (E);
        \draw[dir] (E) to (F);
        \draw[dir] (F) to (A);
        \draw[dir] (B) to (F);

        \draw[dir] (B) to (A);
        \draw[dir] (C) to (A);
        \draw[dir] (C) to (B);
        \draw[dir] (D) to (C);
        \draw[dir] (D) to (B);
        \draw[dir] (E) to (D);
        \draw[dir] (F) to (E);
        \draw[dir] (A) to (F);
        \draw[dir] (F) to (B);
    }
\end{tikzpicture}

    \end{minipage}%
    \begin{minipage}{0.5 \linewidth}
        \pause
        \begin{itemize}
            \item $v\colon ~ f_v \coloneqq \sum\limits_{e \in E_v} x_{e} + c(v) \pmod 2$;
            \item $\sum\limits_{v} c(v) \neq 0 \Rightarrow$ system is unsat. 
        \end{itemize}
    \end{minipage}

    \pause

    System is unsat $\Leftrightarrow$ $\exists \lambda_v \in \{0, 1\}, \sum\limits_{v \in V} \lambda_v
    f_v = 1$ and $\PCR[\field_2]$ has a short proof.
\end{frame}

\begin{frame}{It really helps!}

    Representation of parity over $\field$, $\mathrm{char}(\field) \neq 2$:
    \begin{itemize}
        \item has exponential size in $\{0, 1\}$ variables;
        \item is extremely short in $\{\pm 1\}$ variables: $\prod\limits_{e \in E_v} y_{e} - (-1)^{c(v)}
            = 0$.
    \end{itemize}

    \pause
    \begin{center}
        \begin{minipage}{0.4\linewidth}
            \centering
            \textcolor{blue}{$\prod x_i \prod y_{j} - (-1)^{a}$}
        \end{minipage}
        \begin{minipage}{0.4\linewidth}
            \centering
            \textcolor{blue}{$\prod y_j \prod z_{k} - (-1)^{b}$}
        \end{minipage}

        \pause
        \vspace{0.1cm}

        \begin{prooftree}
            \AxiomC{\textcolor{blue}{$\prod x_i \prod y_{j} - (-1)^{a}$}}
            \UnaryInfC{$\prod x_i \prod y^2_{j} - (-1)^{a} \prod y_{j}$}
            \AxiomC{$y_1^2 - 1$}
            \UnaryInfC{$y_1^2y_2^2 - y_2^2$}
            \AxiomC{$y_2^2 - 1$}
            \BinaryInfC{$y_1^2y_2^2 - 1$}
            \UnaryInfC{$\vdots$}
            \UnaryInfC{$\prod y_j^2 - 1$}
            \UnaryInfC{$\prod x_i \prod y_j^2 - \prod x_i$}
            \BinaryInfC{$\prod x_i - (-1)^{a} \prod y_j$}
            \AxiomC{\textcolor{blue}{$\prod y_j \prod z_k - (-1)^{b}$}}
            \UnaryInfC{$\vdots$}
            \UnaryInfC{$\prod z_k - (-1)^{b} \prod y_j$}
            \BinaryInfC{$\prod x_i - (-1)^{a + b} \prod z_k$}
            \UnaryInfC{$\vdots$}
            \UnaryInfC{$\prod x_i \prod z_k - (-1)^{a + b}$}
        \end{prooftree}
    \end{center}
\end{frame}


\begin{frame}{Hierarchy}

    
\tikzset{
    vert/.style = {
        draw,
        ellipse
    },
    tikzart-fire/.pic = {
        \draw[fill = red!60] (0, 0) .. controls (0.3, 0) and (0.6, 0.1) .. (0.7, 0.3)
            .. controls (0.8, 0.5) and (0.85, 0.6) .. (0.8, 0.9)
            .. controls (0.75, 1.1) and (0.7, 1.2) .. (0.6, 1.4)
            .. controls (0.65, 1.2) and (0.6, 1.05) .. (0.5, 0.9)
            .. controls (0.5, 1.2) and (0.2, 1.3) .. (0.1, 1.6)
            .. controls (0.05, 1.75) and (0.1, 2) .. (0.2, 2.1)
            .. controls (-0.1, 2) and (-0.2, 1.85) .. (-0.3, 1.7)
            .. controls (-0.4, 1.5) and (-0.45, 1.3) .. (-0.4, 1.1)
            .. controls (-0.5, 1.2) and (-0.51, 1.4) .. (-0.5, 1.5)
            .. controls (-0.75, 1.2) and (-0.8, 0.7) .. (-0.7, 0.5)
            .. controls (-0.6, 0.28) and (-0.4, 0) .. (0, 0);
            \fill[white] (0, 0) .. controls (0.3, 0) and (0.52, 0.34) .. (0.37, 0.61)
            .. controls (0.4, 0.54) and (0.32, 0.32) .. (0.25, 0.25)
            .. controls (0.3, 0.35) and (0.25, 0.5) .. (0.2, 0.6)
            .. controls (0.1, 0.8) and (-0.05, 1) .. (0, 1.2)
            .. controls (-0.32, 1) and (-0.3, 0.72) .. (-0.2, 0.47)
            .. controls (-0.3, 0.51) and (-0.31, 0.6) .. (-0.33, 0.7)
            .. controls (-0.4, 0.6) and (-0.4, 0.5) .. (-0.4, 0.4)
            .. controls (-0.35, 0.18) and (-0.2, 0) .. (0, 0);
    },
    semisim/.style = {
        ->,
        blue,
        dashed,
        decorate,
        decoration = {
            snake,
            amplitude = 0.5,
            segment length = 2
        }
    },        
}


    
\begin{tikzpicture}[xscale = 1.3, xshift = -1]
    \node[vert] (res) at (1, 0) {$\Res$};
    \node[vert] (ns) at (-3, 0) {$\NS$};
    \node[vert] (cp) at (3, 1) {$\CP$};
    \node[vert] (resk) at (1.2, 1.4) {$\Res(k)$};
    \node[vert] (acf) at (1.3, 2.4) {$\AC_0$-Frege};
    \node[vert] (resl) at (-1.3, 2.7) {$\ResL$};
    \node[vert] (acfp) at (0.5, 3.8) {$\AC_0[p]$-Frege};
    \node[vert] (fre) at (0.5, 5) {Frege};
    \node[vert] (ips) at (-2, 6) {$\PrSys{IPS}$};
    \node[vert] (pcr) at (-2.8, 1.9) {$\PCR[]$};
    \node[vert] (sos) at (-4, 2.5) {$\SOS$};
    
    \node[vert] (cps) at (-4, 6.5) {$\PrSys{CPS}$};

    

    \draw[->, semisim] (pcr) -- (sos);
    \draw[->] (res) -- (cp);
    \draw[->] (cp) to[out = 90, in = -20] (fre);
    \draw[->] (res) -- (resl);
    \draw[->] (res) -- (resk);
    \draw[->] (resk) -- (acf);
    \draw[->] (res) to[out = 138, in = -30] (pcr);
    \draw[->] (ns) -- (pcr);
    \draw[->] (resl) -- (acfp);
    \draw[->] (acf) -- (acfp);
    \draw[->] (acfp) -- (fre);
    \draw[->] (fre) -- (ips);
    \draw[->, semisim] (ips) -- (cps);

    \draw[->] (pcr) -- (ips);
    \draw[->] (sos) -- (cps);

    \node at (0, 6.9) {};
    

    \begin{scope}[on background layer]
        \draw[ultra thick, fill = black!5] (-4, -1) to[out = 110, in = 220] (-4.4, 3)
            to[out = 40, in = 180] (-1, 2) to[out = 0, in = 125] (2.3, 3) to[out = -55, in = 90]
            (1.5, -1);
    \end{scope}
    \node[blue] at (-1.5, 0.95) {Restriction};


    \begin{scope}[on background layer]
        \fill[orange!5] (-4, -1) to[out = 90, in = 180] (-2.3, 0.75) to[out = 0, in = 180] (0.5, 0.7)
            to[out = 0, in = 200] (3.5, 2) -- (3.5, -1) -- cycle;
        \draw[ultra thick, orange] (-4, -1) to[out = 90, in = 180] (-2.3, 0.75) to[out = 0, in = 180]
            (0.5, 0.7) to[out = 0, in = 200] (3.5, 2);
        \draw[ultra thick] (-4, -1) to[out = 110, in = 220] (-4.4, 3) to[out = 40, in = 180] (-1, 2)
            to[out = 0, in = 125] (2.3, 3) to[out = -55, in = 90] (1.5, -1);
    \end{scope}
    \node[blue] at (0, -0.7) {Mon. Interpolation};
\end{tikzpicture}
    
\end{frame}



\begin{frame}{Results}

    $d_0$ is the degree of $(\Fs, \Hs)$. $n$ is the number of variables of $(\Fs, \Hs)$.
    
    \begin{theorem}
        Any $\SOSf$-proof of $(\Fs, \Hs) \circ \Maj(z_1, z_2, z_3)$ has size
        $\exp\left[\Omega\left(\frac{(d - d_0)^2}{n}\right)\right]$. There $d$ is an
        $\SOS$-degree of $(\Fs, \Hs)$.
    \end{theorem}

    \pause

    \begin{theorem}
        If $\varphi$ is a random $11$-CNF formula then whp any $\SOSf$-proof or $\PCRf[\field]$-proof of
        $\varphi$ has size $\exp(\Omega(n))$.
    \end{theorem}

    \pause
    \begin{theorem}
        Any $\PCRf[\field]$-proof of Pigeonhole Principle has size $\exp(\Omega(n))$.
    \end{theorem}

    $\SOSf$ is strictly stronger than $\PCRf[\mathbb{R}]$.

\end{frame}
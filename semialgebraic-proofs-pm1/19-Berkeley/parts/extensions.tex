\begin{frame}{Extensions}

    \pause
    \begin{block}{Open problem}
        Can we simulate Resolution in $\PCRf[]$?
    \end{block}

    \pause

    \begin{minipage}{0.48\linewidth}
        Variables:
        \begin{itemize}
            \item $x_i \in \{0, 1\}$
            \item $\bar{x}_i \coloneqq 1 - x_i$;
            \item $y_i \coloneqq 1 - 2 x_i$;
        \end{itemize}
    \end{minipage}
    \begin{minipage}{0.48\linewidth}
        Axioms:
        \begin{itemize}
            \item $x^2_i - x_i$
            \item $\bar{x}_i + x_i - 1$;
            \item $y_i + 2 x_i - 1$;
        \end{itemize}
    \end{minipage}

    \vspace{0.4cm}
    \pause

    The $\PCR[\field]_*$-proof of $\Fs$ is a sequence $(p_1, p_2, p_3, \dots, p_{\ell})$:
    \pause
    \begin{itemize}
        \item $p_i \in \Fs \cup \{\text{Axioms}\}$;
        \pause
        \item $p_i = x_j p_k$ for some $j$ and $k < i$;
        \pause    
        \item $p_i = \alpha p_k + \beta p_s$ for some $k, s < i$ and $\alpha, \beta \in \field$;
        \pause
            \item $p_{\ell} = 1$.
    \end{itemize}

    \pause
    
    $\PCR[\field]_*$ simulates $\PCRf[\field]$ and $\PCRb[\field]$.
\end{frame}


\begin{frame}{Lower bounds}
    
    \begin{theorem}
        If $\varphi$ is a random $11$-CNF formula then whp any $\PCR[\field]_*$-proof of $\varphi$ has
        size $\exp(\Omega(n))$.
    \end{theorem}


    \pause

    Can we compose techniques for the $\{0, 1\}$ and $\{\pm 1\}$ bases?

    \pause
    
    \begin{blockpic}{$\Split{x}$}{
        \node at (3, 1.8) {};
        \node[inner sep = 0pt, xscale = -1] at (0, 0)
        {\includegraphics[width = 1.05cm]{pics/hammer.png}};
    }
    
    $p_i \coloneqq r_i + x q_i$.

    $\Split{x}(\pi) \coloneqq (r_1, q_1, r_2, q_2, r_3, q_3, \dots, r_{\ell}, q_{\ell})$.
\end{blockpic}

    $\Split{y}(y_j + 2 x_j - 1)$ \hspace{1.5cm} $\Rightarrow$ \hspace{1.5cm} $r \coloneqq 2 x_i - 1$, $q
    \coloneqq 1$.

\end{frame}


\begin{frame}{Strategy for the $\PCR[\field]_*$}

    $\pi \coloneqq (p_1, \dots, p_{\ell})$ is a proof of $\Fs$. $H \coloneqq \{t \mid t \in \text{ QR of }
    \pi, \deg(t) \text{ is big}\}$.

    \begin{enumerate}
        \item $\pi$ is small $\Rightarrow$ size of $H$ is small.
        \pause
        \item Pick the most frequent \deftext{literal} $x_i$ in $H$.
        \pause
        \item If $x_i$ is frequent then:
            \begin{itemize}
                \item set it to $0$;
                \item set $y_i$ to the right value.
            \end{itemize}
        \pause
        \item If all $x_i$ are \deftext{not} frequent then:
            \begin{itemize}
                \item pick the most frequent variable $y_j$ in $H$;
                \item replace $x_j$ by $\frac{1 - y_j}{2}$;
                \item replace $\bar{x}_j$ by $\frac{1 + y_j}{2}$;
                \item expand brackets ($x_j$ and $\bar{x}_j$ are \deftext{not} frequent);
                \item apply $\Split{y_j}$ to $\pi$ set it to $0$;
            \end{itemize}
        \item Resulting proof is still a proof of \deftext{damaged} $\Fs$.
        \pause
        \item Try to avoid local contradictions in the resulting proof. \begin{tikzpicture}
    \node[inner sep = 0pt, xscale = -1] at (0, 0) {\includegraphics[width =
        0.08\textwidth]{pics/cat.png}};
\end{tikzpicture}

        \pause
        \item Repeat until we have terms of big degree in the QR.
        \vspace{0.3cm}
        \pause
        \item Try to satisfy all \deftext{broken} constraints.
    \end{enumerate}
\end{frame}
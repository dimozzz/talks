\begin{frame}{Complexity theory}

    Let $f\colon X^n \to Y$ be a function. How much resources we have to spend to compute this function?
    \pause
    Examples:
    \begin{itemize}
        \item Is there a road on the map between two points?
        \item Is there a solution of equation $a x^2 + b x + c = 0$ for given $a, b, c$?
    \end{itemize}

    \pause

    Applications in CS:
    \begin{itemize}
        \item Algorithm design
        \item Secure and efficient cryptography
        \item etc.
    \end{itemize}


    \pause
    Applications in Math:
    \begin{itemize}
        \item{} [13-th Hilbert's Problem] Can we represent a root of polynomial as a composition of
            ``simple'' functions?
            \begin{itemize}
                \item How many ``simple'' functions in this composition do we need to find a root of
                    polynomial?
            \end{itemize}
        \item Math Logic
        \item etc.
    \end{itemize}
\end{frame}



\begin{frame}{Proof complexity}

    Language $L$ is a subset of $\{0, 1\}^*$.

    \begin{definition}[Cook, Reckhow 79]
        Proof system for a language $L$ $\Leftrightarrow$ polynomial-time algorithm $\Pi\colon \{0,
        1\}^* \times \{0, 1\}^* \rightarrow \{0, 1\}$ such that:
        \begin{itemize}
            \item $x \in L \Rightarrow \exists w ~ \Pi(x, w) = 1$;
            \item $\exists w ~ \Pi(x, w) = 1 \Rightarrow x \in L$.
        \end{itemize}
    \end{definition}

    \pause

    Complexity measure is a length of $w$.

    \pause

    Basic language: unsatisfiable propositional formulas in CNF ($\UNSAT$).

    In other words: $L$ consists of systems of equations without solution.

\end{frame}


\begin{frame}{Example and motivation}

    Proof system \textit{Nullstellensatz}.
    \begin{itemize}
        \item Input: system of polynomial equations $\{p_1(x_1, x_2, \dots) = 0, p_2(x_1, x_2, \dots) =
            0, \dots\}$.
            \pause
        \item Proof: collection of polynomials $h_i$, such that $\sum\limits_{i} h_i p_i = 1$.
    \end{itemize}

    Motivation:
    \begin{itemize}
        \item Cook's program: prove lower bounds for stronger and stronger proof systems (if we will
            prove lower bounds for all of them then $\P \neq \NP$);
        \pause
        \item Lower bounds on \textcolor{blue}{weak} proof systems
            $\Rightarrow$:
            \begin{itemize}
                \item lower bounds for algorithms for important problems;
                \item lower bounds for \textcolor{blue}{monotone} algorithms.
            \end{itemize}
        \pause
        \item Lower bounds for \textcolor{blue}{strong} proof systems $\Rightarrow$ unconditional lower
            bounds for various computational models.
    \end{itemize}

\end{frame}


\begin{frame}{History}

    \begin{itemize}
        \item First lower bounds (Tseitin 68).
        \pause
        \item Lower bounds on \textit{Resolution} (Haken 85; Kraj{\'{\i}}{\v{c}}ek 97; Ben-Sasson, Wigderson 02; ...;
            Razborov 16).
        \pause
        \item Lower bounds on \textit{Cutting Planes} (Pudl{\'{a}}k 95; Haken, Cook 98).
        \item Lower bounds on \textit{Nullstellensatz} and generalizations (Beame, Impagliazzo,
            Kraj{\'{\i}}{\v{c}}ek, Pitassi, Pudl{\'{a}}k 94; Razborov 98; Alekhovich, Razborov 2003;
            Nordstr{\"{o}}m, Mik{\v{s}}a 15).
        \pause
        \item Lower bounds via communication complexity (Beame, Impagliazzo,
            Pitassi 94; Beame et al. 08; Huynh, Nordstr{\"{o}}m 12; G{\"{o}}{\"{o}}s, Pitassi 14).
    \end{itemize}

\end{frame}

\begin{frame}{Communication protocols. $f\colon U \times V \to T$}
    \begin{center}
    	\onslide<1->{
    \tikzstyle{op1} = [opacity = 0]
    \tikzstyle{op2} = [opacity = 0]
    \tikzstyle{op3} = [opacity = 0]
    \tikzstyle{op4} = [opacity = 0]
}
\only<2->{\tikzstyle{op2} = [opacity = 1]}
\only<3->{\tikzstyle{op3} = [opacity = 1]}
\only<4->{\tikzstyle{op4} = [opacity = 1]}

\begin{tikzpicture}[black]
    \node[police, female, minimum size = 1.5cm] (alice) at (0, 0) {};
    \node[jester, mirrored, minimum size = 1.5cm] (bob) at (7, 0) {};
    \node[above = 0.3 of alice] {$x \in U$};
    \node[above = 0.3 of bob] {$y \in V$};

    \path (alice.east) -- (bob.west) node[midway, above = 2.3] {\Large $f(x, y) = ?$};
    \draw[op2, ->, thick] ($(alice.east) + (0.3, 1)$) -- ($(bob.west) + (-0.3, 1)$) node[midway, above] {$r_1 = a(x)$};
    \draw[op3, <-, thick] ($(alice.east) + (0.3, 0.2)$) -- ($(bob.west) + (-0.3, 0.2)$) node[midway, above] {$r_2 = b(y,
        r_1)$};
    \draw[op4, ->, thick] ($(alice.east) + (0.3, -0.2)$) -- ($(bob.west) + (-0.3, -0.2)$);
    \draw[op4, ->, thick] ($(alice.east) + (0.3, -0.6)$) -- ($(bob.west) + (-0.3, -0.6)$) node[midway, below] {$\vdots$};
\end{tikzpicture}    
    \end{center}

    \pause
    \pause
    \pause
\end{frame}

\begin{frame}{My results. Dag-like communication}
    \begin{itemize}   
        \item ``Communication'' model that related to proof complexity.
        \pause
        \item New lower bounds on the communication model (and on the proof systems as a corollary).
        \pause
        \item New lower bound on the running time of ``monotone'' algorithms. And new separations between
            monotone models of computation.
    \end{itemize}

    \pause
    \begin{enumerate}
        \item[] [\textcolor{blue}{S}]. Dag-Like Communication and Its Applications. CSR 2017.
        \item[] [Garg, G\"{o}\"{o}s, Kamath, \textcolor{blue}{S}]. Monotone Circuit Lower Bounds from
            Resolution. STOC 2018.
        \item[] [G\"{o}\"{o}s, Kamath, Robere, \textcolor{blue}{S}]. Adventures in Monotone Complexity and
            TFNP. ITCS 2019.
    \end{enumerate}
\end{frame}

\begin{frame}{My results. Beyond ``classical'' techniques}
        
    \begin{itemize}
        \item Lower bounds for Nullstellensatz (and generalizations) proofs over $\pm 1$
            variables. Solution to an open problem posted by Impagliazzo, Pitassi and Mouli 2019.
        \pause
        \item Lower bound for Resolution proofs of systems of equations that encodes Pseudorandom
            generators. Solution to an open problem posted by:
            \begin{itemize}
                \item Alekhnovich, Ben-Sasson, Razborov, Wigderson 2004;
                \item Razborov 2015.
            \end{itemize}
        \pause
        \item Lower bound for Resolution systems of equations that encodes \textit{Sparse Weak Pigeonhole
            Principle}. Solution to an open problem posted by:
            \begin{itemize}
                \item Razborov 2004;
                \item Urquhart 2008.
            \end{itemize}
    \end{itemize}

    \begin{enumerate}
        \item[] [\textcolor{blue}{S}]. (Semi) Algebraic proofs over $\pm 1$ variables. STOC 2020.
        \item[] [\textcolor{blue}{S}]. Pseudorandom generators, resolution and
            heavy width. CCC 2022.
        \item[] [F. de Rezende, Nordstr{\"{o}}m, Risse, \textcolor{blue}{S}] Exponential resolution lower
            bounds for weak pigeonhole principle and perfect matching formulas over sparse graphs. CCC
            2020.
    \end{enumerate}
\end{frame}

\begin{frame}{Other areas}

    \begin{itemize}
        \item Lower bounds for DPLL algorithms on satisfiable instances. [Itsykson, \textcolor{blue}{S} 11, 12, 14]
        \item Hierarchy of heuristic computations. [Itsykson, Knop, \textcolor{blue}{S} 15, 16]
        \item Hierarchy of computable distributions. [Itsykson, Knop, \textcolor{blue}{S} 15, 16]
        \item Unambiguous hierarchy and Toda's Theorem. [Hirsch, \textcolor{blue}{S} 15]
    \end{itemize}
    
\end{frame}


\begin{frame}{Future research plans}

    \begin{itemize}
        \item More on new techniques for proving lower bounds in proof complexity.
            \begin{itemize}
                \item Lower bounds for more powerful algebraic proof systems (generalizations of
                    Nullstellsatz).
                \item Apply recent development for circuit complexity in proof complexity (information
                    theory, notion of unpredictability, etc.).
            \end{itemize}
        \item More on ``classical'' communication complexity.
            \begin{itemize}
                \item Lower bounds on parallel algorithms (aka $\AC[0]$-circuits) via communication
                    complexity.
                \item Lower and upper bounds on some generalizations of dag-like communication. 
            \end{itemize}
    \end{itemize}
    
\end{frame}


\begin{frame}{Students}

    Anastasia Sofronova (2016-- )

    Joint results:
    \begin{itemize}
        \item[] [Sofronova, S] $k$-DNF Resolution and Pseudorandom Generators. (In preparation)
        \item[] [Sofronova, S] A Lower Bound for $k$-DNF Resolution on Random CNF Formulas via
            Expansion. (\alert{Submitted} STOC 2023)
        \item[] [Sofronova, S] Branching Programs with Bounded Repetitions and Flow Formulas
            Expansion. (CCC 2021)
    \end{itemize}

    \pause
    Other results:
    \begin{itemize}
        \item[] [Mihajlin, Sofronova] A better-than-3 log n depth lower bound for De Morgan formulas with
            restrictions on top gates. (CCC 2022) 
        \item[] [Galesi, Itsykson, Riazanov, Sofronova] Bounded-depth Frege
            complexity of Tseitin formulas for all graphs. (MFCS 2021)
    \end{itemize}
\end{frame}

\begin{frame}{Teaching experience}

    \begin{itemize}
        \item Summer School in Physics (high-school students, 2005--2019)
            \pause
        \item Academic University (bachelor, master students, 2011--2017)
            \begin{itemize}
                \item Discrete Math;
                \item Complexity Theory;
                \item Math Logic
                \item etc.
            \end{itemize}
        \item St. Petersburg State University (bachelor, master students, 2020--2022)
            \begin{itemize}
                \item Information Theory;
                \item Complexity Theory;
                \item Math Logic
                \item etc.
            \end{itemize}
    \end{itemize}
\end{frame}
\begin{frame}{Complexity theory}

    Let $f\colon X^n \to Y$ be a function. How much resources we have to spend to compute this function?
    \pause
    Examples:
    \begin{itemize}
        \item Is there a road on the map between two points?
        \item Is there a solution of equation $a x^2 + b x + c = 0$ for given $a, b, c$?
    \end{itemize}

    \pause

    Applications in CS:
    \begin{itemize}
        \item Algorithm design
        \item Secure and efficient cryptography
        \item etc.
    \end{itemize}


    \pause
    Applications in Math:
    \begin{itemize}
        \item{} [13-th Hilbert's Problem] Can we represent a root of polynomial as a composition of
            ``simple'' functions?
            \begin{itemize}
                \item How many ``simple'' functions in this composition do we need to find a root of
                    polynomial?
            \end{itemize}
        \item Math Logic
        \item etc.
    \end{itemize}
\end{frame}

\begin{frame}{My research}

    \begin{itemize}
        \item Circuit complexity (how hard it is to compute function via boolean circuits/Turing Machines).
            \pause
        \item Proof complexity (complexity of \alert{nondeterministic computations}).
            \pause
         \item Distribution generation (how hard it is to generate distribution with required
            properties?).
            \pause
        \item Lower bounds for algorithms, etc...
    \end{itemize}
\end{frame}

\begin{frame}{Circuits}

        \begin{minipage}{0.33\linewidth}
        \centering
        Formulas
        \vspace{0.2cm}
        
        \input{pics/mon-form.tex}
    \end{minipage}
    \begin{minipage}{0.33\linewidth}
        \centering
        Circuits
        \vspace{0.2cm}
        
        \begin{tikzpicture}[>=latex]
    \node[circle, minimum size = 0.5cm, inner sep = 0pt, draw, fill = LEIorange!5] (a) at (5, 2)
        {$x$};
    \node[circle, minimum size = 0.5cm, inner sep = 0pt, draw, fill = LEIorange!5] (b) at (3.5, 2)
        {$y$};
    \node[circle, minimum size = 0.5cm, inner sep = 0pt, draw, fill = LEIorange!5] (c) at (4.5, 1)
        {$\lor$};
    \node[circle, minimum size = 0.5cm, inner sep = 0pt, draw, fill = LEIorange!5] (d) at (2.7, 1)
        {$z$};
    \node[circle, minimum size = 0.5cm, inner sep = 0pt, draw, fill = LEIorange!5] (e) at (3.8, 0.3)
        {$\land$};
    %\node[circle, minimum size = 0.5cm, inner sep = 0pt, draw, fill = LEIorange!5] (f) at (5.2, 0.6)
     %   {$y$};
    \node[circle, minimum size = 0.5cm, inner sep = 0pt, draw, fill = LEIorange!5] (g) at (4, -0.5)
        {$\lor$};

    \draw[->] (a) -- (c);
    \draw[->] (b) -- (c);
    \draw[->] (c) -- (e);
    \draw[->] (d) -- (e);
    \draw[->] (e) -- (g);
    \draw[->] (c) -- (g);
    \draw[->] (g) -- ++(0, -0.5);
\end{tikzpicture}
    \end{minipage}
    \begin{minipage}{0.32\linewidth}
        \centering
        More circuits
        \vspace{0.2cm}
        
        \input{pics/mon-r-ckt.tex}
    \end{minipage}

    \pause

    \vspace{1cm}
    \begin{center}
        Circuits are stronger than Turing Machines!
    \end{center}

\end{frame}

\begin{frame}{Results}

    \begin{itemize}   
        \item New technique for proving lower bounds for depth-4 circuits.
            \begin{itemize}
                \item Open problem:
                    \begin{itemize}
                        \item H{\aa}stad, Jukna, Pudlak 1995;
                        \item Meir, Wigderson 2019.
                    \end{itemize}
            \end{itemize}
        \pause
        \item New communication model and techniques for proving lower bounds.
            \begin{itemize}
                \item Corollaries:
                    \begin{itemize}
                        \item Lower bounds for monotone circuit (\alert{general technique}).
                        \item Connection between: Proof Complexity, Circuit Complexity, TFNP.
                    \end{itemize}
            \end{itemize}
    \end{itemize}


    \begin{enumerate}
        \item[] [G{\"{o}}{\"{o}}s, Riazanov, Sofronova, \textcolor{blue}{S} 23] Top-Down Lower Bounds for
            Depth-Four Circuits. FOCS 2023.
        \item[] [Garg, G\"{o}\"{o}s, Kamath, \textcolor{blue}{S} 18]. Monotone Circuit Lower Bounds from
            Resolution. STOC 2018.
        \item[] [G\"{o}\"{o}s, Kamath, Robere, \textcolor{blue}{S} 19]. Adventures in Monotone Complexity and
            TFNP. ITCS 2019.
        \item[] [\textcolor{blue}{S} 17]. Dag-like Communication and Its Applications. CSR 2017.
    \end{enumerate}
    
\end{frame}



\begin{frame}{Proof complexity}

    Language $L$ is a subset of $\{0, 1\}^*$.

    \begin{definition}[Cook, Reckhow 79]
        Proof system for a language $L$ $\Leftrightarrow$ polynomial-time algorithm $\Pi\colon \{0,
        1\}^* \times \{0, 1\}^* \rightarrow \{0, 1\}$ such that:
        \begin{itemize}
            \item $x \in L \Rightarrow \exists w ~ \Pi(x, w) = 1$;
            \item $\exists w ~ \Pi(x, w) = 1 \Rightarrow x \in L$.
        \end{itemize}
    \end{definition}

    \pause
    \begin{itemize}
        \item Complexity measure is a length of $w$.
        \item Basic language: unsatisfiable propositional formulas in CNF ($\UNSAT$).
        \item In other words: $L$ consists of systems of equations without solution.
    \end{itemize}
 
\end{frame}


\begin{frame}{Example and motivation}

    Proof system \textit{Nullstellensatz}.
    \begin{itemize}
        \item Input: system of polynomial equations $\{p_1(x_1, x_2, \dots) = 0, p_2(x_1, x_2, \dots) =
            0, \dots\}$.
            \pause
        \item Proof: collection of polynomials $h_i$, such that $\sum\limits_{i} h_i p_i = 1$.
    \end{itemize}

    Motivation:
    \begin{itemize}
        \item Cook's program: prove lower bounds for stronger and stronger proof systems (if we will
            prove lower bounds for all of them then $\P \neq \NP$).
        \pause
        \item Algorithm design for $\NP$-complete problems.
        \pause
        \item Cryptography.
        \item etc.
    \end{itemize}
\end{frame}


\begin{frame}{My results. Beyond ``classical'' techniques}
        
    \begin{itemize}
        \item Lower bounds for (Semi)Algebraic proofs over $\pm 1$ variables.
            \begin{itemize}
                \item Open problem: Impagliazzo, Pitassi and Mouli 2019.
            \end{itemize}
        \pause
        \item Hardness of inversion of Pseudorandom generators in Resolution.
            \begin{itemize}
                \item Open problem:
                    \begin{itemize}
                        \item Alekhnovich, Ben-Sasson, Razborov, Wigderson 2004;
                        \item Razborov 2015.
                    \end{itemize}
            \end{itemize}
        \pause
        \item Lower bound for Resolution systems of equations that encodes \textit{Sparse Weak Pigeonhole
            Principle} (\alert{$\NP$ vs. $\Ppoly$}).
            \begin{itemize}
                \item Open problem:
                    \begin{itemize}
                        \item Razborov 2004;
                        \item Urquhart 2008.
                    \end{itemize}
            \end{itemize}
    \end{itemize}

    \begin{enumerate}
        \item[] [\textcolor{blue}{S} 24] Random $(\log n)$-CNF are Hard for Cutting Planes. STOC 2024.
        \item[] [\textcolor{blue}{S} 20] (Semi)Algebraic proofs over $\pm 1$ variables. STOC 2020.
        \item[] [\textcolor{blue}{S} 22] Pseudorandom generators, resolution and
            heavy width. CCC 2022.
        \item[] [F. de Rezende, Nordstr{\"{o}}m, Risse, \textcolor{blue}{S} 20] Exponential resolution lower
            bounds for weak pigeonhole principle and perfect matching formulas over sparse graphs. CCC
            2020.
    \end{enumerate}
\end{frame}

\begin{frame}{Other areas}

    \begin{itemize}
        \item Sampling and Certifying Symmetric Functions. [Filmus, Leigh, Riazanov, \textcolor{blue}{S} 23]
        \item Lower bounds for DPLL algorithms on satisfiable instances. [Itsykson, \textcolor{blue}{S} 11, 12, 14]
        \item Lower bounds for DPLL algorithms on satisfiable instances. [Itsykson, \textcolor{blue}{S} 11, 12, 14]
        \item Hierarchy of heuristic computations. [Itsykson, Knop, \textcolor{blue}{S} 15, 16]
        \item Hierarchy of computable distributions. [Itsykson, Knop, \textcolor{blue}{S} 15, 16]
        \item Unambiguous hierarchy and Toda's Theorem. [Hirsch, \textcolor{blue}{S} 15]
    \end{itemize}
    
\end{frame}


\begin{frame}{Future research plans}

    Ultimate goals.
    \begin{itemize}
        \item General circuit lower bounds.
        \item Frege proof system lower bounds.
        \item Connection with new areas (more on foundations of cryptography and analysis of algorithms).
    \end{itemize}

    \pause
    Reachable (but hard) goals.
    \begin{itemize}
        \item Combinatorial lower bounds for constant depth circuits.
        \item Lower bounds for resolution over parities.
        \item New techniques for $\AC_0[p]$-circuits.
    \end{itemize}

    \pause 
    For grant application (doable problems):
    \begin{itemize}
        \item New lower bounds for $k$-DNF resolution (partial progress [Sofronova, \textcolor{blue}{S} 23]).
        \item Combinatorial proof analog of Bazzi's Theorem ($r$-wise independence fools
            $\AC_0$-circuits).
        \item Lower bounds for more powerful algebraic proof systems (generalizations of Nullstellsatz).
    \end{itemize}
    
\end{frame}


\begin{frame}{Students}

    \begin{itemize}
        \item Sofronova Anastasia (PhD, 2021--Present) (now joint adv. with Mika G{\"{o}}{\"{o}}s)
        \item Artur Riazanov (PhD, 2022--Present) (joint adv. with Mika G{\"{o}}{\"{o}}s)
        \item Oskar Dabkowski (Bachelor Project, 2023)
        \item Tiberiu Mosnoi (Master Project, 2023)
        \item Sofronova Anastasia (master, 2019--2021)
        \item Sofronova Anastasia (undergraduate, 2016--2019)
    \end{itemize}
\end{frame}

\begin{frame}{Teaching experience}

    \begin{itemize}
        \item Summer School in Physics (high-school students, 2005--2019)
            \pause
        \item Academic University (bachelor, master students, 2011--2017)
            \begin{itemize}
                \item Discrete Math;
                \item Complexity Theory;
                \item Math Logic;
                \item etc.
            \end{itemize}
        \item St. Petersburg State University (bachelor, master students, 2020--2022)
            \begin{itemize}
                \item Information Theory;
                \item Complexity Theory;
                \item Math Logic;
                \item etc.
            \end{itemize}
        \item EPFL (master students, 2023)
            \begin{itemize}
                \item Computational Complexity.
            \end{itemize}
    \end{itemize}
\end{frame}
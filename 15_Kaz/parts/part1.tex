\begin{frame}
	\frametitle{$\DPLL$-алгоритмы}

   	\input{pics/tree.tex}
    
	\pause
    \pause
    \pause
    \pause
    \pause
    \begin{itemize}
        \item Процедура $\mathbf{A}$ выбирает переменную для расщепления.
    	\pause
	    \item Процедура $\mathbf{B}$ выбирает первое значение.
    	\pause
    	\item Правила упрощения:
	    \begin{itemize}
            \item удаление единичного дизъюнкта;
        	\item чистые литералы.
    	\end{itemize}
    \end{itemize}

\end{frame}


\begin{frame}
    \frametitle{$\DPLL$-алгоритмы}

    \begin{itemize}
        \item $\SAT$-солверы основаны на алгоритмах расщепления ($CDCL$ не укладываются в данную модель).
        \item Оптимальность на выполнимых формулах.
	\end{itemize}

    \pause
    \begin{block}{Утверждение}
   		Если $\P = \NP$, то существует верхняя полиномиальная оценка на время работы $\DPLL$-алгоритма на \alert{выполнимых формулах}.        
    \end{block}

\end{frame}

\begin{frame}
    \frametitle{Нижние оценки}

    \pause
	\begin{itemize}
		\item Невыполнимые формулы
		\begin{itemize}
            \item{} Экспоненциальный нижние из оценки следуют из нижних оценок для резолюции.
			\item{} [Tseitin, 1968] ... [Pudlak, Implagliazzo, 2000].
		\end{itemize}
        \pause
		\item Выполнимые формулы
		\begin{itemize}
			\item Обращение функции соответствует выполнимой формуле..
            \pause
            \item{} [Nikolenko~2002], [Achilioptas, Beame, Molloy~2003-2004] экспоненциальные нижние оценки на специфические алгоритмы.
            \item{} [Alekhnovich, Hirsch, Itsykson~2005] экспоненциальные нижние оценки на ``близорукие'' и ``пьяные'' алгоритмы.
            \pause
            \item{}  Экспоненциальные нижние оценки на время обращения функции Голдрейха для близоруких [J. Cook et al.~2009,
				2014] и пьяных [Itsykson~2010] алгоритмов.
		\end{itemize}
	\end{itemize}
\end{frame}

\begin{frame}
	\frametitle{Пьяные и близорукие алгоритмы}
    \pause

    \begin{definition}
        Пьяный алгоритм:
        \begin{itemize}
	        \item $\mathbf{A}$~--- произвольная;
	        \item $\mathbf{B}$~--- случайный бит.
        \end{itemize}
    \end{definition}

    \pause
    \begin{definition}
        Близорукая процедура:
        \pause
        \begin{itemize}
	        \item видит структуру формулы;
        	\pause
        	\item не видит отрицаний;
        	\item<6-> запрашивает отрицания в $K = n^{1 - \epsilon}$ клозах.
        \end{itemize}
    \end{definition}

    \pause
    $\begin{array}{l}
        (x_1 \vee x_3 \vee x_5) \\
        \alert<7->{(x_2 \vee x_3)} \\
        (x_2 \vee x_4 \vee x_5) \\
        \alert<7->{(x_1 \vee x_4 \vee x_6)} \\
    \end{array}
    \pause
    \pause
    \pause
    \Rightarrow
    \begin{array}{l}
        (x_1 \vee x_3 \vee x_5) \\
        (x_2 \vee \alert{\neg} x_3) \\
        (x_2 \vee x_4 \vee x_5) \\
        (x_1 \vee \alert{\neg} x_4 \vee x_6) \\
    \end{array}$
    
\end{frame}

\begin{frame}
	\frametitle{Функция Голдрейха}
	$f:\{0, 1\}^n \rightarrow \{0, 1\}^n$

    \pause

    \begin{columns}
    	\begin{column}{5.5cm}
            \input{pics/function_graph.tex}
        \end{column}

        \pause
        \pause
        \begin{column}{5.5cm}
            \begin{itemize}
	            \item $G(X, Y, E)$~--- двудольный граф;
            	\pause
                \item $\forall y \in Y ~~ deg(y) = d$
            	\pause
            	\item $d$~--- константа.
            \end{itemize}
        \end{column}
	\end{columns}
    

    \pause
    \begin{theorem}
		Существует такая \alert{явная} конструкция графа $G$ и такой нелинейный предикат $P$, что любой близорукий или пьяный
        $\DPLL$-алгоритм делает $2^{n^{\Omega(1)}}$ шагов на ``$f_{G, P}(x) = f_{G, P}(a)$'' для почти всех $a \in \{0, 1\}^n$.
	\end{theorem}
\end{frame}
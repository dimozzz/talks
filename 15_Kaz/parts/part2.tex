\begin{frame}
	\frametitle{$\DPLL$-алгоритмы с ошибкой}

   	\onslide<1->{
   	\tikzstyle{vertex2} = [opacity = 0]
   	\tikzstyle{vertex3} = [opacity = 0]
    \tikzstyle{vertex4} = [opacity = 0]
   	\tikzstyle{vertex5} = [opacity = 0]
    \tikzstyle{vertex9} = [opacity = 0]
    \tikzstyle{vertex11} = [opacity = 0]
}
\only<2->{\tikzstyle{vertex2} = [opacity = 1]}
\only<3->{\tikzstyle{vertex3} = [opacity = 1]}
\only<4->{\tikzstyle{vertex4} = [opacity = 1]}
\only<5->{
  	\tikzstyle{vertex5} = [opacity = 1]
    \tikzstyle{vertex9} = [opacity = 1]
}
\only<9->{\tikzstyle{vertex9} = [opacity = 0]}
\only<11->{\tikzstyle{vertex11} = [opacity = 1]}

\tikzstyle{end} = [circle, minimum size=0.6cm, draw, inner sep = 0.1pt]
            
\tikzstyle{level 1}=[level distance=1.5cm, sibling distance=5cm]
\tikzstyle{level 2}=[sibling distance=2cm]
    
\begin{tikzpicture}[label distance=8mm]
	\node [end] (z){$\phi$}
       	child [vertex2] {node [end] (b) {$\phi'$}
			child [vertex3]{
	           	node {$\vdots$}
                edge from parent
	  	        node[left] {$x_{j} := c_1$}
            }
		    child [vertex4]{
               	node {$\vdots$}
                edge from parent
	   	        node[right] {$x_{j} := 1 - c_1$}
            }
            node [vertex11, left = 0.25cm] {$\mathbf{C} = 1$}
           	edge from parent
            node[left] {$x_{i} := c$}
        }
        child [vertex5] {node [end] (c) {$\phi''$}
           	child [vertex9]{
               	node {$\vdots$}
                edge from parent
	            node[left] {$x_{k} := c_2$}
            }
		    child [vertex9]{
               	node {$\vdots$}
                edge from parent
	            node[right] {$x_{k} := 1 - c_2$}
            }
            node[vertex11, right = 0.25cm] {$\mathbf{C} = 0$}
            edge from parent
	   	    node[right] {$x_{i} := 1 - c$}
        }
        node[vertex11, above = 0.25cm] {$\mathbf{C} = 1$};
\end{tikzpicture}

    
	\pause
    \pause
    \pause
    \pause
    \pause
    \begin{itemize}
        \item Процедура $\mathbf{A}$ выбирает переменную для расщепления.
    	\pause
	    \item Процедура $\mathbf{B}$ выбирает первое значение.
    \end{itemize}

    \pause
    Алгоритм может иметь ошибку.

    \pause
    \pause
    \begin{itemize}
	    \item Процедура $\mathbf{C}$ отрезает ветви.
    \end{itemize}
\end{frame}


\begin{frame}
	\frametitle{Результат}
	\pause
	\begin{theorem}
        Для любой детерминированной процедуры $\mathbf{A}$ существует такой ансамбль $(\Phi_n, \mathcal{D}_n)$ и такое $\delta \ne
        \delta(n)$, что:
        \pause
		\begin{itemize}
            \item $\Phi_n$ невыполнимая формула.
        	\item $\Phi_n$ может быть построена за полиномиальное время.
            \pause
            \item $\mathcal{D}_n$ распределение на выполнимых формулах.
            \pause
			\item $\forall \mathbf{B}, \mathbf{C}$,
				$\Pr\limits_{x \gets \mathcal{D}_n}[\mathcal{M}(x)
                \ne 0] > 1 - \epsilon \Rightarrow
                time(\mathcal{M}(\Phi_n)) \ge (1 - \epsilon) 2^{\Omega(min(n^\delta, \frac{n}{K}))}$, где $\mathcal{M}$~---
                близорукий алгоритм $\mathcal{D}(\mathbf{A}, \mathbf{B},\mathbf{C})$.
		\end{itemize}
	\end{theorem}
\end{frame}


%%% Local Variables: 
%%% mode: latex
%%% TeX-master: t
%%% End: 

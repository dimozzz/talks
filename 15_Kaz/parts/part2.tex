\begin{frame}
	\frametitle{$\DPLL$-алгоритмы с ошибкой}

   	\input{pics/tree_2.tex}
    
	\pause
    \pause
    \pause
    \pause
    \pause
    \begin{itemize}
        \item Процедура $\mathbf{A}$ выбирает переменную для расщепления.
    	\pause
	    \item Процедура $\mathbf{B}$ выбирает первое значение.
    \end{itemize}

    \pause
    Алгоритм может иметь ошибку.

    \pause
    \pause
    \begin{itemize}
	    \item Процедура $\mathbf{C}$ отрезает ветви.
    \end{itemize}
\end{frame}


\begin{frame}
	\frametitle{Результат}
	\pause
	\begin{theorem}
        Для любой детерминированной процедуры $\mathbf{A}$ существует такой ансамбль $(\Phi_n, \mathcal{D}_n)$ и такое $\delta \ne
        \delta(n)$, что:
        \pause
		\begin{itemize}
            \item $\Phi_n$ невыполнимая формула.
        	\item $\Phi_n$ может быть построена за полиномиальное время.
            \pause
            \item $\mathcal{D}_n$ распределение на выполнимых формулах.
            \pause
			\item $\forall \mathbf{B}, \mathbf{C}$,
				$\Pr\limits_{x \gets \mathcal{D}_n}[\mathcal{M}(x)
                \ne 0] > 1 - \epsilon \Rightarrow
                time(\mathcal{M}(\Phi_n)) \ge (1 - \epsilon) 2^{\Omega(min(n^\delta, \frac{n}{K}))}$, где $\mathcal{M}$~---
                близорукий алгоритм $\mathcal{D}(\mathbf{A}, \mathbf{B},\mathbf{C})$.
		\end{itemize}
	\end{theorem}
\end{frame}


%%% Local Variables: 
%%% mode: latex
%%% TeX-master: t
%%% End: 
